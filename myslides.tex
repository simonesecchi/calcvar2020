\input ctuslides2


\def\R{{\bbchar R}}

\def\ge{{\varepsilon}}

\def\frac#1#2{{#1 \over #2}}

\slideshow

\tit Calcolo delle Variazioni\nl
     a.a. 2019-2020\nl

\subtit\bf Simone Secchi\nl simone.secchi@unimib.it

\subtit\rm \url{http://elearning.unimib.it}

\pg;

\sec Prerequisiti e strumenti

* Calcolo differenziale in spazi euclidei di dimensione finita
* Teoria della misura e dell'integrazione secondo Lebesgue
* Principi di Analisi Funzionale Lineare
* Teoria elementare degli spazi di Sobolev (almeno il caso hilbertiano $p=2$)


\pg;

\sec Strumenti: il calcolo differenziale in dimensione infinita

{\bf Notazione.} Se $X$ \`e uno spazio di Banach (reale), il suo duale
topologico sar\`a denotato con il simbolo $X^\*$. Se $A \in X^\*$, il
simbolo $A[v]$ indicher\`a il valore di $A$ nel punto $v$; talvolta
semplificheremo la notazione e scriveremo $Av$ al posto di $A[v]$.

\bigskip

{\bf Definizione.} Siano $X$ uno spazio di Banach, e $U \subset X$ un
suo aperto. Un funzionale su $U$ \`e un'applicazione $I \colon U \to
\R$. Si noti che i ``nostri'' funzionali {\bf non} sono
necessariamente {\bf lineari}!

\pg;

{\bf Definizione.} Sia $I \colon U \to \R$ un funzionale. Diremo che
$I$ \`e derivabile secondo Fr\'echet nel punto $u \in U$ se esiste un
elemento $A \in X^\*$ tale che
$$
\lim_{\|v\| \to 0} {I(u+v)-I(u)-Av \over \|v\|} =0, \eqmark
$$
o, equivalentemente, se
$$
I(u+v)=I(u)+Av+o(\|v\|) \quad\hbox{per $v \to 0$}.
$$

\medskip

Si osservi che questa \`e la definizione di funzione differenziabile
quando $X=\R^n$.

\pg;

{\bf Lemma.} Se $I$ \`e derivabile nel punto $u \in U$, allora
l'elemento $A$ che soddisfa (1) \`e univocamente determinato.

\smallskip

{\em Dim.} Infatti, supponiamo che $A$ e $B$ siano due elementi di
$X^\*$ che soddisfano (1). Per sottrazione,
$$
\lim_{\|v\| \to 0} {(A-B)v \over \|v\|} =0.
$$
Fissiamo $u \in X$ con $\|u\|=1$, e scegliamo $v=tu$, $t \to
0^+$. Allora
$$
(A-B)u=\lim_{t \to 0+} {t(A-B)u \over t\|u\|} =0.
$$
Per l'arbitrariet\`a di $u$, concludiamo che $A=B$.

\pg;

{\bf Definizione.} Se $I$ \`e un funzionale derivabile secondo
Fr\'echet nel punto $u \in U$, la derivata (talvolta: il
differenziale) di Fr\'echet di $I$ in $u$ \`e l'unico elemento $I'(u)
\in X^\*$ (talvolta: $dI(u)$) tale che
$$
I(u+v)=I(u)+I'(u)[v]+o(\|v\|)
$$
per $v \to 0$.

\bigskip

{\bf Definizione.} Se $I \colon U \to \R$ \`e derivabile secondo
Fr\'echet in ogni punto $u \in U$, diremo che $I$ \`e
Fr\'echet-derivabile in $U$. La derivata di Fr\'echet di $I$ \`e
allora la mappa $I' \colon U \to X^\*$ che ad $u \in U$ associa $I'(u)
\in X^\*$. Si tratta --- in generale --- di una mappa {\em non
  lineare}.

Se $I'$ \`e una mappa continua da $U$ in $X^\*$, diremo che $I \in C^1(U)$.

\pg;

\sec Il caso hilbertiano

Se $H$ \`e uno spazio di Hilbert (reale), \`e noto che gli elementi
del duale $H^\*$ sono isometricamente identificati con vettori di $H$
attraverso l'isomorfismo di Riesz. In particolare, un funzionale $I$
definito su $U \subset H$ \`e derivabile in $u \in U$ se e solo esiste
un vettore, detto d'ora in poi {\em gradiente} di $I$ in $u$ e
denotato $\nabla I(u)$, tale che
$$
I(u+v) = I(u) + \langle \nabla I(u) \mid v \rangle + o(\|v\|)
$$
per $v\to 0$.

\pg;

{\bf Proposizione.} Siano $I$ e $J$ due funzionali derivabili nel
punto $u \in X$. Allora valgono le seguenti affermazioni.
\begitems
\style n
* Se $a$ e $b$ sono numeri reali, allora $aI+bJ$ \`e derivabile in
$u$, e vale $(aI+bJ)'(u)=aI'(u)+bJ'(u)$.
* Il prodotto $IJ$ \`e derivabile in $u$, e vale $(IJ)'(u) =
J(u)I'(u)+I(u)J'(u)$.
* Se $\gamma \colon \R \to U$ \`e una curva derivabile in $t_0$ e
$u=\gamma(t_0)$, allora la composizione $\eta \colon \R \to \R$
definita da $\eta(t)=I(\gamma(t))$ \`e derivabile in $t_0$, e vale
$\eta'(t_0)=I'(u)[\gamma'(t_0)]$.
* Se $A \subset \R$ \`e un aperto, $f \colon A \to \R$ \`e derivabile in
$I(u) \in A$, allora la composizione $K(u)=f(I(u))$ \`e definita in un
intorno $V$ di $u$, \`e derivabile in $u$ e vale $K'(u)=f'(I(u))I'(u)$.
\enditems

\pg;

{\em Dim.} La prima affermazione \`e banale (esercizio!). Per quanto
riguarda la seconda, quando $v \to 0$ in $X$, abbiamo
$$
\eqalign{%
I(u+v)J(u+v) &= \left( I(u)+I'(u)[v]+o(\|v\|) \right) \left(
  J(u)+J'(u)[v]+o(\|v\|) \right) \cr
&= I(u)J(u) + J(u)I'(u)[v] + I(u)J'(u)[v] + I'(u)[v] J'(u)[v] \cr
&\quad {} + o(\|v\|) \left( I(u)+I'(u)[v]+J(u)+J'(u)[v]+o(\|v\|)
\right). \cr
}%
$$
Concludiamo osservando che
$$
I'(u)[v] J'(u)[v]
+ o(\|v\|) \left( I(u)+I'(u)[v]+J(u)+J'(u)[v]+o(\|v\|) \right)
$$
\`e $o(\|v\|)$ per $v \to 0$. La terza affermazione \`e simile,
infatti per $h \to 0$ in $\R$
$$
\eqalign{%
  \eta(t_0+h) &= I(\gamma (t_0+h)) = I(\gamma(t_0)+\gamma'(t_0)h +
  o(|h|)) \cr
  &= I(u) + I'(u)[\gamma'(t_0)h+o(|h|)] + o(\|\gamma'(t_0)h+o(|h|)\|)
  \cr
  &= \eta(t_0) +I'(u)[\gamma'(t_0)h] +I'(u)[o(|h|)] +
  o(\|\gamma'(t_0)h+o(|h|)\|). \cr
}%
$$
Poich\'e gli ultimi due addendi sono $o(|h|)$, otteniamo che
$$
\eta(t_0+h) = \eta(t_0)+I'(u)[\gamma'(t_0)h]+o(|h|).
$$

\pg;

Infine, quando $v \to 0$ in $X$, si verifica come prima che
$$
\eqalign{%
  K(u+v)&= f(I(u+v)) = f(I(u)+I'(u)[v]+o(\|v\|)) \cr
  &= f(I(u))+f'(I(u))(I'(u)[v]+o(\|v\|))+o(I'(u)[v]+o(\|v\|)) \cr
  &= f(I(u)) + f'(I(u))I'(u)[v]+o(\|v\|).
}%
$$

\bigskip

{\bf Osservazione.} \`E possibile introdurre il concetto di derivata
per applicazioni tra due spazi di Banach $X$ e $Y$. Solo in questo
contesto pu\`o essere enunciata una formulazione completa della regola
di derivazione delle funzioni composte.

Poich\'e non ne faremo uso in queste lezioni, rimandiamo al testo di
Ambrosetti e Prodi per ulteriori approfondimenti.

\pg;

{\bf Definizione.} Sia $I$ un funzionale definito nell'aperto $U$ di
$X$, e sia $u \in U$. Diremo che $I$ \`e derivabile secondo G\^ateaux
in $u$ se esiste un elemento $A \in X^\*$ tale che
$$
\lim_{t \to 0} {I(u+tv)-I(u) \over t} = Av \eqmark
$$
per ogni $v \in X$. In tal caso, l'unico (esercizio!) elemento
siffatto prende il nome di derivata secondo G\^ateaux di $I$ in $u$, e
si denota con $I'_G(u)$ o con $d_GI(u)$.

\bigskip

Osserviamo che questa nuova derivata riprende la cosiddetta {\em
  derivata direzionale} gi\`a nota nell'ambito del calcolo
differenziale in dimensione finita.

In particolare, ricordando i ``soliti'' esempi in $\R^2$, deduciamo
che esistono funzionali (non lineari) derivabili secondo G\^ateaux ma
non derivabili secondo Fr\'echet.

\pg;

\sec Condizione sufficiente per la derivabilit\`a secondo Fr\'echet

{\bf Proposizione.} Supponiamo che $U \subset X$ sia un aperto, che
$I$ sia G\^ateaux-derivabile in $U$, e che $I'_G$ sia continua in un
punto $u \in U$. Allora $I$ \`e Fr\'echet-derivabile in $u$, e
(ovviamente) $I'(u)=I'_G(u)$.

\bigskip

Omettiamo la dimostrazione, che \`e probabilmente stata proposta nel
caso $X=\R^2$ nel corso di Analisi Matematica 2.

\pg;

\sec Punti critici

{\bf Definizione.} Siano $X$ uno spazio di Banach, $U$ un aperto di
$X$, e $I$ un funzionale definito su $U$. Diremo che $u \in U$ \`e un
punto critico di $I$ se $I$ \`e derivabile in $u$ e
$$
I'(u)=0.
$$
Pi\`u esplicitamente, questo significa che $I'(u)[v]=0$ per ogni $v
\in  X$.

Se $u$ \`e un punto critico di $I$ e $I(u)=c$, diremo che $u$ \`e un
punto critico (di $I$) al livello $c$. Se, per qualche $c \in \R$,
l'insieme $I^{-1}(\{c\})\subset X$ contiene almeno un elemento, diremo
che $c$ \`e un valore critico per $I$.

\bigskip

L'equazione $I'(u)=0$ \`e nota come equazione di Eulero (o di
Eulero-Lagrange) associata al funzionale $I$.

\pg;

\sec Esempi

{\bf Esempio 1.} Ogni $A \in X^\*$ \`e derivabile. Infatti, basta
scrivere
$$
A[u+v]=Au+Av
$$
per dedurre che $A'(u)=A$ per qualsiasi $u \in X$.

\medskip

{\bf Esempio 2.} Sia $X$ uno spazio di Banach, e sia $a \colon X
\times X \to \R$ una forma bilineare continua. Denotiamo con $J \colon
X \to \R$ il funzionale definito da $J(u)=a(u,u)$ per ogni $u \in
X$. Allora $J$ \`e derivabile in $X$. Infatti
$$
\eqalign{%
J(u+v) &=a(u+v,u+v) = a(u,u)+a(u,v)+a(v,u)+a(v,v) \cr
&= J(u) +a(u,v)+a(v,u)+a(v,v).
}%
$$
Poich\'e $|a(v,v)| \leq M \|v\|^2$ per l'ipotesi di continuit\`a di
$a$ come forma bilineare, deduciamo che $a(v,v)=o(\|v\|)$ per $v \to
0$, e dunque che
$$
J'(u)[v]=a(u,v)+a(v,u).
$$

\pg;

{\bf Esempio 3.} (esercizio) Sia $H$ uno spazio di Hilbert con norma $\| \cdot
\|$. Il funzionale $J(u)=\|u\|$ \`e derivabile in ogni punto $u \neq
0$, e risulta
$$
\nabla J(u) = {u \over \|u\|}.
$$

\medskip

{\bf Esempio 4.} Sia $X$ uno spazio di Banach, e siano $I$, $J$ due
funzionali derivabili in $X$. Definiamo
$$
Q(u) = {I(u) \over J(u)}
$$
sul sottoinsieme (aperto) $\{u \in  X \mid J(u) \neq 0\}$. Per la
Proposizione sulle regole di calcolo dimostrata sopra, possiamo
affermare che $Q$ \`e derivabile e che
$$
Q'(u) = {J(u) I'(u)[v] - I(u) J'(u)[v] \over J(u)^2}
$$
per ogni $u\ in X$ tale che $J(u) \neq 0$.

\pg;

\sec Esempi in spazi concreti

{\bf Esempio 5.} Sia $\Omega \subset \R^N$, $N \geq 1$, un insieme
aperto e limitato. Definiamo i funzionali
$$
I \colon L^2(\Omega) \to \R, \quad I(u) = \int_\Omega |u(x)|^2 \, dx,
$$
$$
J \colon H_0^1(\Omega) \to \R, \quad J(u) = \int_\Omega |\nabla
u(x)|^2 \, dx,
$$
$$
K \colon H^1(\Omega) \to \R, \quad K(u) = \int_\Omega |\nabla
u(x)|^2 \, dx,
$$
$$
L \colon H^1(\Omega) \to \R, \quad L(u) = \int_\Omega |\nabla
u(x)|^2 \, dx + \int_\Omega |u(x)|^2 \, dx.
$$
Trattandosi di forme quadratiche associate a forme bilineari continue,
sappiamo gi\`a che i quattro funzionali sono derivabili.

\pg;

Esplicitamente, valgono le relazioni
$$
\eqalign{%
\nabla I(u) &= 2u \cr
\nabla L(u) &= 2u \cr
\nabla J(u) &= 2u.
}
$$
Un
calcolo diretto mostra che
$$
K'(u)[v] = 2 \int_\Omega \nabla u(x) \cdot \nabla v(x) \, dx
$$
per ogni $u$, $v \in H^1(\Omega)$, ma non siamo autorizzati ad
affermare che $\nabla K(u)=2u$ (perch\'e?)

\pg;

\sec Inversione della Convergenza Dominata

{\bf Teorema di Lebesgue.} Sia $\Omega$ un aperto di $\R^N$, e sia
$\{u_k\}_k$ una successione in $L^1(\Omega)$ tale che
\begitems
\style n
* $u_k(x) \to u(x)$ per q.o $x \in \Omega$;
* esiste $v \in L^1(\Omega)$ tale che $|u_k(x)| \leq v(x)$ per q.o. $x
\in \Omega$ e ogni $k$.
\enditems
Allora $u \in L^1(\Omega)$ e $u_k \to u$ nella norma di $L^1(\Omega)$.

\bigskip

Questo risultato fondamentale di Teoria della Misura pu\`o essere {\em
parzialmente} invertito, come mostra il seguente teorema. Per la
dimostrazione, rimandiamo al libro di H. Brezis, Analisi funzionale.

\pg;

{\bf Teorema.} Sia $\Omega$ un aperto di $\R^N$, e sia $\{u_k\}_k$ una
successione di $L^p(\Omega)$, $p \in [1,+\infty]$, tale che $u_k \to
u$ in $L^p(\Omega)$. Allora esistono una sottosuccessione
$\{u_{k_j}\}_j$ ed una funzione $v \in L^p(\Omega)$ tali che
\begitems
\style n
* $u_{k_j}(x) \to u(x)$ per q.o. $x \in \Omega$;
* per ogni $j$, $|u_{k_j}(x)| \leq v(x)$ per q.o. $x \in \Omega$.
\enditems

\bigskip

Questo teorema mostra che la convergenza forte in $L^p$ implica --- a
meno di sottosuccessioni --- l'esistenza di una funzione dominante.

\pg;

\sec Operatori di Nemitskii

Siano $\Omega$ un aperto limitato di $\R^N$, $N \geq 3$, con frontiera
regolare, e sia $f \colon \R \to \R$ una funzione continua. Supponiamo
che esistano $a>0$ e $b>0$ tali che
$$
|f(t)| \leq a+b|t|^{2^\*-1},
$$
dove $2^\* = 2N/(N-2)$ \`e l'esponente critico di Sobolev. Definiamo
$$
F(t) = \int_0^t f(x) \, dx
$$
e consideriamo il funzionale $J \colon H^1(\Omega) \to \R$ dato da
$$
J(u) = \int_\Omega F(u(x))\, dx.
$$

\pg;

{\bf Proposizione.} Sotto le ipotesi precedenti, $J$ \`e un funzionale
derivabile in $H^1(\Omega)$, e vale
$$
J'(u)[v] = \int_\Omega f(u(x))v(x)\, dx
$$
per ogni $u$, $v \in H^1(\Omega)$.

\bigskip

La dimostrazione non \`e immediata: mostriamo prima che $J$ \`e
G\^a\-te\-aux-derivabile, e poi che la derivata di G\^ateaux \`e
continua. Come abbiamo visto sopra, ci\`o implica che $J$ \`e
Fr\'echet-derivabile.

\pg;

* Derivata di G\^ateaux

Per q.o. $x\in\Omega$, risulta
$$
\lim_{t \to 0} {F(u(x)+t(v(x)) - F(u(x)) \over t} = f(u(x))v(x).
$$
Per il teorema di Lagrange, esiste un numero reale $\theta$ tale che
$|\theta| \leq |t|$ e
$$
\eqalign{%
\left| {F(u(x)+t(v(x)) - F(u(x)) \over t} \right| &= \left|
f(u(x)+\theta v(x))v(x) \right| \cr
&\leq \left( a+b|u(x)+\theta v(x)|^{2^\*-1} \right) |v(x)| \cr
&\leq C \left( |v(x)|+|u(x)|^{2^\*-1}|v(x)|+|v(x)|^{2^\*} \right).
}
$$
Per Convergenza Dominata,
$$
\lim_{t \to 0} \int_\Omega {F(u(x)+t(v(x)) - F(u(x)) \over t}\, dx =
\int_\Omega f(u(x))v(x)\, dx.
$$

\pg;

Poich\'e $v \mapsto \int_\Omega f(u(x))v(x)\, dx$ \`e un operatore
lineare e continuo  in $H^1(\Omega)$ (disuguaglianza di H\"older e di
Sobolev), abbiamo individuato la derivata secondo G\^ateaux di $J$:
$$
J'_G(u)[v] = \int_\Omega f(u(x))v(x)\, dx.
$$

* Derivata di Fr\'echet

Mostriamo che $J'_G \colon H^1(\Omega) \to (H^1(\Omega))^\*$ \`e
un'applicazione continua. A tal fine, sia $\{u_k\}_k$ una successione
che converge a $u$ in $H^1(\Omega)$. Per il teorema di convergenza
dominata inversa, possiamo supporre che --- a meno di sottosuccessioni
---
\begitems
* $u_k \to u$ in $L^{2^\*}(\Omega)$;
* $u_k(x) \to u(x)$ per q.o. $x \in \Omega$;
* esiste $w \in L^{2^\*}(\Omega)$ tale che $|u_k(x)| \leq w(x)$ per
q.o. $x \in \Omega$ e ogni $k$.
\enditems

\pg;

Usiamo la disuguaglianza di H\"older:
$$
\eqalign{%
\left| (J'_G(u_k)-J'_G(u))[v] \right| &\leq \int_\Omega
|f(u_k(x))-f(u(x))| |v(x)| \, dx \cr
&\leq \left( \int_\Omega |f(u_k(x))-f(u(x))|^{2^\* \over 2^\*-1}\, dx
\right)^{2^\*-1 \over 2^\*} \times \cr
&\quad \times \left( \int_\Omega |v(x)|^{2^\*}\, dx
\right)^{1/2^\*}.
}
$$
La continuit\`a di $f$ implica $\lim_{k \to +\infty}
|f(u_k(x))-f(u(x))|=0$ per q.o. $x\in\Omega$, e inoltre
$$
\eqalign{%
|f(u_k(x))-f(u(x))|^{2^\* \over 2^\* -1} &\leq C \left(
1+|u_k(x)|^{2^\*-1} + |u(x)|^{2^\* -1} \right)^{2^\* \over 2^\* -1} \cr
&\leq C \left(
1+|w(x)|^{2^\*-1} + |w(x)|^{2^\* -1} \right)^{2^\* \over 2^\* -1} \cr
&\leq C \left(
1+|w(x)|^{2^\*} + |w(x)|^{2^\* } \right) \in L^1(\Omega).  \cr
}
$$
Per Convergenza Dominata,
$$
\lim_{k \to +\infty} \int_\Omega |f(u_k(x))-f(u(x))|^{2^\* \over 2^\*
-1} \, dx =0.
$$

\pg;

Perci\`o
$$
\eqalign{%
\|J'_G(u_k)-J'_G(u)\| &= \sup \{ (J'_G(u_k)-J'_G(u))[v] \mid v \in
H^1(\Omega),\ \|v\|=1 \} \cr
&\leq C \left( \int_\Omega |f(u_k(x))-f(u(x))|^{2^\* \over 2^\*
-1} \, dx \right)^{2^\* -1 \over 2^\*} \to 0.
}
$$
Riassumendo: abbiamo dimostrato che da ogni successione $\{u_k\}_k$
convergente a $u$ \`e possibile estrarre una sottosuccessione tale che
$J'_G(u_k) \to J'_G(u)$ in $(H^1(\Omega))^\*$. \`E ora un esercizio di
Topologia Generale dedurre che l'intera successione $\{u_k\}_k$ gode
di questa propriet\`a (perch\'e il limite \`e indipendente dalla
sottosuccessione scelta).

\pg;

\`E possibile estendere quanto dimostrato al caso in cui $\Omega$ sia
un aperto qualunque, anche illimitato. Il prezzo da pagare \`e un
rafforzamento delle ipotesi sulla funzione $f$

\pg+

Sia dunque $\Omega$ un aperto di $\R^N$ con frontiera regolare, e sia
$f \colon \R \to \R$ una funzione continua e tale che
$$
|f(t)| \leq a|t| + b |t|^{2^\*-1}.
$$
Dimostriamo che il funzionale $J(u)=\int_\Omega F(u(x))\, dx$ \`e
derivabile in $H^1(\Omega)$.

\pg+

* Derivata di G\^ateaux

Per q.o. $x \in \Omega$ e per ogni $v \in H^1(\Omega)$,
$$
\lim_{t \to 0} { F(u(x)+tv(x)) - F(u(x)) \over t} = f(u(x))v(x).
$$


\pg;

Per il teorema di Lagrange, esiste $\theta=\theta(x)$ tale che
$|\theta| \leq |t|$ e
$$
\eqalign{%
\left| { F(u(x)+tv(x)) - F(u(x)) \over t}\right| &= \left|
f(u(x)+\theta v(x))v(x) \right| \cr
&\leq C \left( |u(x)+\theta v(x)| + |u(x)+\theta v(x)|^{2^\* -1} \right)
\cr
&\leq C \left( |u(x)| |v(x)| + |v(x)|^{2^\*} + |u(x)|^{2^\* -1} |v(x)|
+ |v(x)|^{2^\*} \right) \cr
&\in L^{1}(\Omega). \cr
}
$$
Concludiamo ancora per Convergenza Dominata.

\pg+

Se poi $\{u_k\}_k$ \`e una successione che tende a $u$ in
$H^1(\Omega)$, a meno di sottosuccessioni possiamo anche supporre che
\begitems
* $u_k(x) \to u(x)$ per q.o. $x\in \Omega$
* $u_k \to u$ in $L^2(\Omega)$ e in $L^{2^\*}(\Omega)$
* esistono $w_1 \in L^{2^\*}(\Omega)$ e $w_2 \in L^2(\Omega)$ tali che
$|u_k(x)| \leq w_i(x)$, $i=1$, $2$, per q.o. $x \in \Omega$.
\enditems

\pg;

Sia $\varepsilon>0$, e scegliamo $R_\varepsilon>0$ tale che
$$
\|u\|_{L^2(\Omega_\ge)} + \|u\|_{L^{2^\*}(\Omega_\ge)}^{2^\* -1} +
\|w_1\|_{L^{2^\*}(\Omega_\ge)}^{2^\* -1} + \|w_2\|_{L^2(\Omega_\ge)}
\leq \ge,
$$
dove $\Omega_\ge = \{x \in \Omega \mid |x| > R_\ge \}$. Ora,
$$
\eqalign{%
\left| (J'_G(u_k)-J'_G(u))[v] \right| &\leq \int_\Omega |f(u_k)-f(u)|
|v| \, dx \cr
&= \int_{\Omega \cap B(0,R_\ge)} |f(u_k)-f(u)|
|v| \, dx + \int_{\Omega_\ge} |f(u_k)-f(u)|
|v| \, dx. \cr
}
$$
Trattiamo separatamente gli ultimi due integrali.

\pg;

Innanzitutto
$$
\eqalign{%
&\int_{\Omega_\ge} |f(u_k)-f(u)|
|v| \, dx \cr
&\leq C \int_{\Omega_\ge} \left( |u_k|+|u|+|u_k|^{2^\* -1} + |u|^{2^\*
-1} \right)|v| \, dx \cr
&\leq C \left( \int_{\Omega_\ge} |w_2| |v| \, dx + \int_{\Omega_\ge}
|u| |v| \, dx  + \int_{\Omega_\ge} |w_1|^{2^\* -1} |v| \, dx +
\int_{\Omega_\ge} |u|^{2^\* -1} |v| \, dx \right) \cr
&\leq C \|v\| \left( \|u\|_{L^2(\Omega_\ge)} + \|u\|_{L^{2^\*}(\Omega_\ge)}^{2^\* -1} +
\|w_1\|_{L^{2^\*}(\Omega_\ge)}^{2^\* -1} + \|w_2\|_{L^2(\Omega_\ge)}
\right) \cr
&\leq C \|v\| \ge.
}
$$

\pg;

D'altra parte,
$$
\int_{\Omega \cap B(0,R_\ge)} |f(u_k)-f(u)| |v| \, dx
\leq C \left( \int_{\Omega \cap B(0,R_\ge)} |f(u_k)-f(u)|^{2^\* \over
2^\* -1} \, dx \right)^{2^\* -1 \over 2^\*} \|v\|.
$$
Sui sottoinsiemi limitati di $\R$, la funzione $f$ soddisfa una
maggiorazione del tipo $|f(t)| \leq C (1+|t|^{2^\* -1} )$, e come
sopra concludiamo che
$$
\lim_{k \to +\infty} \int_{\Omega \cap B(0,R_\ge)} |f(u_k)-f(u)|^{2^\* \over
2^\* -1} \, dx =0.
$$
Ricapitolando,
$$
\eqalign{%
\| (J'_G(u_k)-J'_G(u))\| &= \sup \left\{ (J'_G(u_k)-J'_G(u))[v] \mid v \in H^1(\Omega),\ \|v\|=1
\right\} \cr
&\leq C \left( \int_{\Omega \cap B(0,R_\ge)} |f(u_k)-f(u)|^{2^\* \over
2^\* -1} \, dx \right)^{2^\* -1 \over 2^\*} + C \ge \cr
&= o(1) + C \ge.
}
$$
Per l'arbitrariet\`a di $\ge>0$, concludiamo che $J'_G(u_K) \to
J'_G(u)$.

\pg;

{\bf Osservazione.} La regolarit\`a della frontiera di $\Omega$ \`e
stata utilizzata solo {\em implicitamente} per garantire la validit\`a
di tutte le immersioni di Sobolev. Ne consegue che gli stessi
risultati sussistono, senza alcuna ipotesi su $\partial \Omega$, se
restringiamo il funzionale $J$ al sottospazio $H_0^1(\Omega)$.

\pg;

\sec Un problema lineare ellittico

Prenderemo a modello di applicazione
un'equazione alle derivate parziali del secondo ordine, avente la
forma
$$
\left\{
\matrix{ -\Delta u + q(x) u = h(x), \hfill &x \in \Omega \cr
u(x)=0, \hfill &x \in \partial \Omega \cr}
\right. \eqno(P)
$$
dove
\begitems
* $\Omega$ \`e un aperto limitato di $\R^N$
* $q\in C(\Omega)$, $h \in C(\Omega)$.
\enditems

\pg+

Il problema (P) prende il nome di {\em problema di Dirichlet
omogeneo}. L'aggettivo {\em omogeneo} si riferisce qui alla condizione
{\em al bordo} $u=0$ su $\partial\Omega$. Osserviamo che il problema
\`e {\em lineare}.

\pg;

* Una {\em soluzione classica} di (P) \`e una funzione $u \in
  C^2(\overline{\Omega})$ tale che (P) sia soddisfatto puntualmente in
  $\overline{\Omega}$.

\pg+

Fissiamo $v \in C_0^1(\Omega)$ e moltiplichiamo l'equazione in (P) per
$v$. Integrando su $\Omega$ con l'ausilio del Teorema di Stokes
(versione nota anche come {\em formula di Gauss-Green}), otteniamo che
$$
\int_\Omega \nabla u \cdot \nabla v \, dx + \int_\Omega q(x) u v \, dx
= \int_\Omega h(x)v \, dx.
$$
Questa uguaglianza ha senso sotto ipotesi ben pi\`u deboli di quelle
finora assunte. Ad esempio gli integrali sono finiti quando $u$, $v$
sono funzioni di $L^2(\Omega)$ tali che $\partial u / \partial x_i$ e
$\partial v / \partial x_i$ appartengano ad $L^2(\Omega)$ per ogni
indice $i$. La continuit\`a di $q$ e $h$ \`e allora eccessiva, e
possiamo sostituirla con $q \in L^\infty(\Omega)$, $h \in
L^2(\Omega)$.

\pg;

* Siano dunque $q \in L^\infty(\Omega)$, $h \in
L^2(\Omega)$. Una {\em soluzione debole} di (P) \`e una funzione $u
\in H_0^1(\Omega)$ tale che
$$
\int_\Omega \nabla u \cdot \nabla v \, dx + \int_\Omega q(x) u v \, dx
= \int_\Omega h(x) v \, dx
$$
per ogni $v \in H_0^1(\Omega)$.

\pg+

{\bf Osservazione.} Ogni soluzione classica \`e anche soluzione
debole.

Infatti, $u \in C^2(\overline{\Omega})$ implica $u \in
H^1(\Omega)$. Per una nota propriet\`a degli spazi di Sobolev,
poich\'e $u$ \`e continua in $\overline{\Omega}$ e $u=0$ su
$\partial\Omega$, abbiamo $u \in H_0^1(\Omega)$.

Sappiamo che, per ogni $v \in C_0^1(\Omega)$,
$$
\int_\Omega \nabla u \cdot \nabla v \, dx + \int_\Omega q(x) u v \, dx
= \int_\Omega h(x)v \, dx.
$$
Poich\'e $C_0^1(\Omega)$ \`e un sottospazio denso di $H_0^1(\Omega)$,
ad ogni $v \in H_0^1(\Omega)$ facciamo corrispondere una successione
$\{v_n\}_n \subset C_0^1(\Omega)$ tale che $v_n \to v$ in
$H_0^1(\Omega)$.

\pg;

Facendo tendere $n \to +\infty$ nella relazione
$$
\int_\Omega \nabla u \cdot \nabla v_n \, dx + \int_\Omega q(x) u v_n \, dx
= \int_\Omega h(x)v_n \, dx,
$$
deduciamo che $u$ \`e una soluzione debole di (P).

\pg+

\`E ragionevole chiedersi se ogni soluzione debole sia anche una
soluzione classica. Vediamo che cosa possiamo dire.

\pg+

Sia $u \in H_0^1(\Omega)$ una soluzione debole di (P), e supponiamo
che $h \in C(\overline{\Omega})$. {\em Se} \`e noto, per qualche
motivo, che $u \in C^2(\Omega)$, allora possiamo dedurre che $u=0$ su
$\partial\Omega$.

\pg;

Scegliendo in particolare $v \in C_0^1(\Omega)$ nella definizione di
soluzione debole, otteniamo che, per ogni $v \in C_0^1(\Omega)$,
$$
\int_\Omega \nabla u \cdot \nabla v \, dx + \int_\Omega q(x) u v \, dx
= \int_\Omega h(x)v \, dx.
$$

\pg+

Usando nel senso contrario la formula di Stokes, arriviamo alla
relazione
$$
\int_\Omega \left( -\Delta u + q(x) u -h(x) \right) v\, dx=0
$$
per ogni $v \in C_0^1(\Omega)$.

\pg+

Per densit\`a di $C_0^1(\Omega)$ in $L^2(\Omega)$, concludiamo che
$-\Delta u + q(x) u -h(x)=0$ quasi ovunque, e che $u=0$ quasi ovunque
in $\partial\Omega$.

\pg+

* Morale della favola: abbiamo bisogno di una {\em teoria della
regolarit\`a} per le soluzioni deboli di (P).

\pg;

\sec Soluzioni deboli e punti critici

Definiamo il funzionale $J \colon H_0^1(\Omega) \to \R$ mediante la formula
$$
J(u) = \frac{1}{2} \int_\Omega |\nabla u|^2 \, dx + \frac{1}{2}
\int_\Omega q(x) |u|^2 \, dx - \int_\Omega h(x) u \, dx.
$$

\pg+

Segue dagli esempi sulla derivabilit\`a che $J$ \`e derivabile secondo
Fr\'echet e che
$$
J'(u)[v] = \int_\Omega \nabla u \cdot \nabla v\, dx + \int_\Omega q(x)
u v \, dx - \int_\Omega h(x) v \, dx
$$
per ogni $u$, $v \in H_0^1(\Omega)$.

\pg+

Quindi le soluzioni deboli di (P) sono esattamente i punti critici del
funzionale $J$.

\pg+

Il funzionale $J$ \`e chiamato {\em funzionale dell'energia} associato
a (P), anche se dovremmo chiamarlo pi\`u propriamente funzionale di
azione o di Eulero-Lagrange.

\pg;

\sec Un problema non lineare

Molti modelli della Fisica Moderna conducono ad equazioni {\em non
lineari}. Vediamo come la discussione precedente possa essere estesa
ad un prototipo di equazione alle derivate parziali {\em semilineare}.

\pg+

Sia $\Omega$ un aperto limitato di $\R^N$. Supponiamo che $q \in
L^\infty(\Omega)$ e che $f \colon \R \to \R$ sia una funzione continua
e tale che
$$
|f(t)| \leq a + b |t|^{2^\* -1}.
$$
\pg+
Consideriamo il problema
$$
\left\{
\matrix{-\Delta u + q(x) u = f(u), \hfill &x \in \Omega \cr
u=0, \hfill &x \in \partial \Omega \cr}
\right. \eqno(SP)
$$

\pg;

{\bf Definizione.} Una soluzione debole di (SP) \`e una funzione $u
\in H_0^1(\Omega)$ tale che
$$
\int_\Omega \nabla u \cdot \nabla v \, dx + \int_\Omega q(x) uv \, dx
= \int_\Omega f(u)v \, dx
$$
per ogni $v \in H_0^1(\Omega)$.

\pg+

Sia $F(t) = \int_0^t f(x)\, dx$, e definiamo un funzionale $J \colon
H_0^1(\Omega) \to \R$ mediante la formula
$$
J(u) = \frac{1}{2} \int_\Omega |\nabla u|^2 \, dx + \frac12
\int_\Omega q(x)|u|^2 \, dx - \int_\Omega F(u)\, dx.
$$

\pg+

Sappiamo che $J$ \`e derivabile e che
$$
J'(u)[v] = \int_\Omega \nabla u \cdot \nabla v \, dx - \int_\Omega
q(x) uv\, dx - \int_\Omega f(u)v\, dx
$$
per ogni $v \in H_0^1(\Omega)$.

\pg+

* Ancora una volta, le soluzioni deboli di (SP) corrispondono ai punti
  critici del funzionale dell'energia $J$.

\pg;

\sec Riassunto

* Abbiamo visto che \`e possibile estendere il calcolo differenziale
elementare (cio\`e quello delle funzioni di pi\`u variabili) alle
funzioni di {\em infinite} variabili.

* Con questo linguaggio, abbiamo messo in corrispondenza biunivoca
  opportune soluzioni di equazioni differenziali con gli zeri della
  derivata di opportuni funzionali (non lineari).

\pg;

\sec Prospettive

* Ci prefiggiamo ora di... andare a caccia dei punti critici, al fine
  di {\em risolvere} equazioni differenziali.

* Per far ci\`o, vedremo che occorrono strumenti nuovi, e che la {\em
  topologia} dello spazio di riferimento avr\`a un ruolo fondamentale.

\pg;

\sec Problemi (in tutti i sensi) di minimizzazione

Uno dei pi\`u importanti teoremi dell'Analisi Matematica recita:

\medskip

{\bf Teorema.} Ogni funzione reale continua su un insieme compatto di $\R^N$
possiede massimi e minimi assoluti.

\pg+

Questo enunciato continua a sussistere per funzioni continue definite
su spazi metrici compatti, con dimostrazione sostanzialmente identica.

\pg+ Il ruolo della compattezza nel Teorema di Weierstrass \`e
fondamentale, come mostra il seguente controesempio, dovuto anch'esso
a Weierstrass.

\pg;

{\bf Esempio.} Sia
$$
I(u) = \int_{-1}^1 \left| x u'(x) \right|^2 \, dx
$$
definito per ogni funzione $u \in C^1([-1,1])$ a valori reali. Il
problema
$$
\min_{u \in X} I(u),
$$
dove $X = \left\{ u \in C^1([-1,1]) \mid u(\pm 1) = \pm 1 \right\}$
non ha soluzioni.

\pg+

Infatti, la famiglia di funzioni
$$
u_\ge (x) = \frac{\arctan(x/\ge)}{\arctan(1/\ge)}
$$
mostra con un calcolo diretto che $\inf_X I=0$. \`E poi evidente che
$I(u)=0$ implica $u'=0$ in $[-1,1]$, cio\`e $u$ \`e costante. Pertanto
$u \notin X$.

\pg;

\sec Weierstrass in astratto

{\bf Teorema.} Sia $M$ uno spazio topologico di Hausdorff, e
supponiamo che $I \colon M \to \R \cup \{+\infty\}$ soddisfi la
seguente condizione:

\smallskip

Per ogni $\alpha \in \R$, l'insieme $K_\alpha = \{ u \in M \mid I(u)
\leq \alpha \}$ \`e compatto.

\smallskip

Allora $I$ raggiunge il suo estremo inferiore $\inf_M I$.

\medskip

{\em Dim.} Possiamo evidentemente supporre che $I$ non sia
identicamente uguale a $+\infty$. Poniamo
$$
\alpha_0 = \inf_M I \geq -\infty,
$$
e consideriamo una successione $\{\alpha_m\}_m$ strettamente
decrescente verso $\alpha_0$. Poniamo per brevit\`a
$K_m=K_{\alpha_m}$.
\pg+
Per ipotesi, ogni $K_m$ \`e compatto e non-vuoto. Inoltre $K_m \supset
K_{m+1}$. Per la propriet\`a dell'intersezione finita, esiste
$$
u \in \bigcap_{m \in {\bbchar N}} K_m,
$$
cio\`e $I(u) \leq \alpha_m$ per ogni $m$. Facendo tendere $m \to
+\infty$, concludiamo che $I(u) \leq \alpha_0$, cio\`e $u$ \`e un
minimo assoluto di $I$ su $M$.

\pg;

* Nell'ipotesi del Teorema precedente, per ogni $\alpha \in \R$
  l'insieme
  $$
  \left\{ u \in M \mid I(u) > \alpha \right\} = M \setminus K_\alpha
  $$
  \`e aperto in $M$. Questo significa, per definizione, che $I$ \`e
  una funzione {\em semicontinua inferiormente} su $M$.

* Nei casi concreti, la struttura di $M$ pu\`o essere pi\`u ricca di
  quella di un mero spazio topologico. Di seguito un caso piuttosto
  frequente nell'Analisi Variazionale.

\pg;

{\bf Teorema.} Sia $V$ uno spazio di Banach riflessivo con norma $\|
\cdot \|$, e sia $M \subset V$ un sottospazio debolmente
chiuso. Supponiamo che $I \colon M \to \R \cup \{+\infty\}$ sia un
funzionale tale che
\begitems
* $I(u) \to +\infty$ se $\|u\| \to +\infty$;
* per ogni $u \in M$ ed ogni successione $\{u_m\}_k$ in $M$ tale che
$u_m \rightharpoonup u$, risulta: $I(u) \leq \liminf_{m \to +\infty}
I(u_m)$.
\enditems
Allora $I$ \`e limitato dal basso, e raggiunge il suo minimo assoluto.

\medskip

{\em Dim.} Sia $\alpha_0 = \inf_M I$ e sia $\{u_m\}_m$ una successione
in $M$ tale che $I(u_m) \to \alpha_0$ per $m \to +\infty$. Per la
prima ipotesi, $\{u_m\}_m$ \`e una successione limitata in $V$
(altrimenti...). Il teorema di Eberlein-Smulian garantisce la
convergenza debole di tale successione a qualche $u \in V$. Per
ipotesi $M$ \`e debolmente chiuso, sicch\'e $u \in M$. Infine,
$$
I(u) \leq \liminf_{m \to +\infty} I(u_m) = \alpha_0.
$$

\pg;

* La semicontinuit\`a inferiore debole del precedente teorema \`e
  sovente garantita dalla {\em convessit\`a} del funzionale.

\pg+

{\bf Lemma.} Siano $X$ uno spazio di Banach, $K$ un sottoinsieme
convesso e chiuso di $X$, e $I$ un funzionale convesso s.c.i. su
$K$. Allora $I$ \`e debolmente s.c.i.

\medskip

{\em Dim.} Per ogni $\alpha \in \R$, l'insieme $K_\alpha$ \`e convesso
e chiuso. Per un noto risultato di Analisi Funzionale Lineare, tale
insieme \`e anche debolmente chiuso, quindi $I$ \`e debolmente s.c.i.

\pg+

{\bf Notazione.} Per rendere pi\`u espressiva la simbologia,
utilizzeremo anche la scrittura
$$
[I \leq \alpha] = \left\{ u \in X \mid I(u) \leq \alpha \right\}.
$$

\pg;

\sec Punti critici e topologia

Quando $X$ \`e uno spazio di Banach riflessivo e $I$ \`e un funzionale
convesso, la strategia per dimostrare che $I$ raggiunge il suo minimo
assoluto $\alpha_0$ consiste in due passi:
\begitems
* $\alpha_0 \in \R$ \`e tale che $[I < \alpha_0]=\emptyset$;
* per $\ge>0$ abbastanza piccolo, l'insieme $[I \leq \alpha_0+\ge]$
\`e non-vuoto e debolmente compatto.
\enditems

\pg+

In realt\`a ci\`o indica la presenza di punti critici di $I$ \`e la
differenza topologica dei sottolivelli $[I \leq c]$ e $[I \leq
c+\ge]$.

\pg;

\sec Esempi

* Sia $I(x)=x^3-3x$ per ogni $x \in \R$. La derivata di $I$ si annulla
  in $\pm 1$, e se poniamo $c_1=I(1)=-2$, $c_2=I(-1)=2$, vediamo che
  \begitems
  \style n
  * se $a_1<c_1$, l'insieme $[I \leq a_1]$ \`e un intervallo del tipo
  $(-\infty,\alpha_1]$;
  * se $c_1<a_2<c_2$, risulta $[I \leq a_2] = (-\infty,\alpha_2] \cup
  [\beta_2,\gamma_2]$ con $\alpha_2<\beta_2<\gamma_2$;
  * se $a_3>c_2$, risulta $[I \leq a_3] = (-\infty,\alpha_3]$.
  \enditems
  Nell'attraversare i valori $c_1$ e $c_2$, i sottolivelli del
  funzionale sono topologicamente distinti.

* Sia $I(x,y) = x^2-y^2$ per ogni $(x,y) \in \R^2$. Sappiamo che $0$
  \`e l'unico valore critico di $I$. Per ogni $\ge>0$, l'insieme $[I
  \leq \ge]$ \`e connesso, mentre $[I \leq -\ge]$ ha due componenti
  connesse.

\pg;

* Sia $I(x,y) = (x^2+y^2)^2-2(x^2+y^2)$ per ogni $(x,y) \in
  \R^2$. Esistono due valori critici $c_1=-1$ e $c_2=0$. Si vede
  facilmente (tutta la geometria del funzionale \`e radiale!) che se
  $a_1<c_1$, l'insieme $[I \leq a_1]$ \`e vuoto, che se $c_1<a_2<c_2$
  l'insieme $[I \leq a_2]$ \`e un anello del tipo $r^2 \leq x^2+y^2
  \leq R^2$, e infine che se $a_3>c_2$ l'insieme $[I \leq a_3]$ \`e
  una palla $B(0,R)$. Quindi il numero di componenti connesse non
  cambia nell'attraversare il livello $c_2=0$, e tuttavia anello e
  palla hanno invarianti topologici diversi.

\pg+

Questa idea di collegare la topologia dei sottolivelli all'esistenza
di punti critici si rivela vincente, e da essa si sviluppa la
cosiddetta {\em Teoria di Morse}.

\pg+ A causa del forte legame con la Topologia Algebrica, questa
teoria non rientra nei limiti del nostro corso.

\pg;

\sec Principi variazionali

* Come osservato, non \`e chiaro che una funzione limitata e
  semicontinua inferiormente debba raggiungere il suo minimo assoluto:
  si pensi alla funzione {\em analitica} $f(x)=\arctan x$ per ogni $x
  \in \R$.

\pg+

* Con il termine di {\it principi variazionali} ci si riferisce a
  teoremi che permettano di costruire {\it quasi minimi}, cio\`e
  tipicamente successioni minimizzanti per funzionali limitati e
  s.c.i., aventi per\`o ulteriori propriet\`a.

\pg;

\sec Il principio di Ekeland

{\bf Teorema.} Sia $M$ uno spazio metrico completo con metrica $d$, e
sia $I \colon M \to \R \cup \{+\infty\}$ un funzionale limitato dal
basso, s.c.i. e non identicamente infinito. Allora per ogni $\ge>0$ e
ogni $u \in M$ con
$$
I(u) \leq \inf_M I + \ge,
$$
esiste un elemento $v \in M$ tale che
$$
d(u,v) \leq 1, \quad I(v) < I(u),
$$
e, per ogni $w \neq v$ in $M$,
$$
I(w) > I(v)-\ge d(v,w).
$$

\pg;

{\em Dim.} Definiamo un ordinamento su $M$ ponendo
$w \leq v$ se e solo se $I(w)+\ge d(v,w) \leq I(v)$. Poniamo $u_0=u$,
e supponiamo di aver definito $u_n$. Sia
$$
S_n = \left\{ w \in M \mid w \leq u_n \right\},
$$
e scegliamo $u_{n+1} \in S_n$ tale che
$$
I(u_{n+1}) \leq \inf_{S_n} I + \frac{1}{n+1}.
$$
\pg+
\`E chiaro che $S_{n+1} \subset S_n$ poich\'e $u_{n+1} \leq u_n$;
inoltre $S_n$ \`e chiuso perch\'e $I$ \`e s.c.i.

Se $w\in S_{n+1}$, allora $w \leq u_{n+1} \leq u_n$ e dunque
$$
\ge d(w,u_{n+1}) \leq I(u_{n+1})-I(u_n) \leq \inf_{S_{n}} I +
\frac{1}{n+1} - \inf_{S_n} I = {1 \over n+1}.
$$
Deduciamo che
$$
\mathop{\rm diam} S_{n+1} \leq \frac{2}{\ge (n+1)}.
$$

\pg;

Poich\'e $M$ \`e uno spazio completo, \`e noto che
$$
\bigcap_n S_n = \{v\}
$$
per qualche $v \in M$. In particolare $v \in S_0$, cio\`e $v \leq
u_0=u$. Quindi
$$
I(v) \leq I(u)-\ge d(u,v) \leq I(u)
$$
e
$$
d(u,v) \leq \ge^{-1} \left( I(u)-I(v) \right) \leq \ge^{-1} \left(
\inf_M I + \ge - \inf_M I \right) =1.
$$
Per concludere, dimostriamo che $w \leq v$ implica $w=v$. Infatti, $w
\leq v$ implica $w \in u_n$ per ogni $n$, e dunque $w \in S_n$ per
ogni $n$. Quindi $w=v$.

\pg;

* Il senso del principio di Ekeland \`e che ad ogni punto in cui il
funzionale raggiunge ``quasi'' il minimo, \`e possibile associare un
punto ancora ``migliore'', che realizza anche il minimo assoluto {\em
proprio} di
$$
w \mapsto I(w)+\ge d(v,w).
$$

\pg+

* Questo principio diventa ancora pi\`u suggestivo se si arricchisce
  la struttura dello spazio $M$ e del funzionale $I$.

\pg+

{\bf Teorema.} Siano $X$ uno spazio di Banach, $\varphi \colon X \to
\R$ una funzione derivabile e limitata dal basso su $X$. Allora, per
ogni $\ge>0$ e per ogni $u \in X$ tale che $\varphi (u) \leq \inf_X
\varphi + \ge$, esiste $v \in X$ tale che $\varphi(v) \leq
\varphi(u)$,
$$
|v-u| \leq \sqrt{\ge}, \quad |\varphi'(v)| \leq \sqrt{\ge}.
$$

\pg;

{\em Dim.} Scegliamo $M=X$, $I=\varphi$ e, per $\ge>0$ dato, scegliamo
$d(x,y)=\ge^{-1/2} |x-y|$ nel Teorema di Ekeland. Otteniamo un
elemento $v \in X$ tale che $\varphi(w)>\varphi(v)-\sqrt{\ge}|w-v|$
per ogni $w \neq v$.

Scriviamo $w=v+th$ con $t>0$, $h \in X$, $|h|=1$, per ottenere
$$
\varphi(v+th)-\varphi(v) > -\sqrt{\ge} t.
$$
\pg+
Dividendo per $t$ e prendendo il limite per $t \to 0$, deduciamo
$$
-\sqrt{\ge} \leq \varphi'(v)[h].
$$
Per l'arbitrariet\`a di $h$ sulla sfera unitaria di $X$, deve essere
$|\varphi'(v)| \leq \sqrt{\ge}$.

Le altre propriet\`a di $v$ sono ovvia conseguenza del Teorema di
Ekeland.

\pg+

* \`E ormai spontaneo ``discretizzare'' il parametro $\ge>0$, per
  costruire successioni minimizzanti con derivata ``quasi'' nulla.

\pg;

{\bf Corollario.} Siano $X$ uno spazio di Banach, $\varphi \colon X
\to \R$ un funzionale limitato dal basso e derivabile in $X$. Allora,
per ogni successione minimizzante $\{u_k\}_k$ di $\varphi$, esiste una
successione minimizzante $\{v_k\}_k$ di $\varphi$ tale che
$\varphi(v_k) \leq \varphi(u_k)$,
$$
\lim_{k \to +\infty} |u_k-v_k| =0, \quad
\lim_{k \to +\infty} \|\varphi'(v_k)\| =0.
$$

\medskip

{\em Dim.} Basta porre
$$
\ge_k = \cases{\varphi(u_k)-\inf_X \varphi &se $\varphi(u_k)-\inf_X
\varphi>0$ \cr
1/k &se $\varphi(u_k)-\inf_X \varphi=0$. \cr
}
$$

\pg;

\sec Passi di montagna: il caso finito-dimensionale

* L'idea di costruire punti ``quasi critici'' per un funzionale si
  rivela essere un approccio molto utile nello sviluppo di teoremi di
  ``vero'' punto critico. Cominciamo con un caso in dimensione finita.

\pg+

{\bf Teorema.} Supponiamo che $I \in C^1(\R^N)$ sia un funzionale
coercivo, e che $I$ possieda due punti distinti di minimo stretto,
$x_1$ e $x_2$. Allora $I$ possiede un terzo punto critico $x_3$ che
non \`e un minimo locale, e dunque distinto da $x_1$ e $x_2$.

\pg+

* Immaginiamo dunque (almeno nel caso $N=2$) il grafico di $I$, con
  due punti di minimo locale stretto. Il teorema afferma, sotto
  l'ipotesi che $I$ diverga all'infinito per argomenti divergenti
  all'infinito, che da qualche parte esiste un punto di {\em sella}.

\pg;

{\em Dim.} Definiamo il valore
$$
\beta = \inf_{p \in P} \max_{x \in p} I(x),
$$
dove
$$
P = \left\{ p \subset \R^N \mid x_1 \in p,\ x_2 \in p, \ \hbox{$p$ \`e
compatto e connesso} \right\}.
$$
Prendiamo una successione $\{p_m\}_m \subset P$ minimizzante per $\beta$, nel
senso che
$$
\lim_{m \to +\infty} \max_{x\in p_m} I(x) = \beta.
$$
Poich\'e $I$ \`e coercivo, gli insiemi $p_m$ sono uniformemente
limitati.
\pg+
L'insieme dei punti di accumulazione di $\{p_m\}_m$,
$$
p = \bigcap_{m \in {\bbchar N}} \overline{\bigcup_{l \geq m} p_l},
$$
\`e l'intersezione di una successione decrescente di insiemi compatti
e connessi: dunque anch'esso \`e compatto e connesso. Inoltre
$\{x_1,x_2\} \subset p$, poich\'e $\{x_1,x_2\} \subset p_m$ per ogni
$m$.

\pg;

Deduciamo che
$$
\max_{x \in p} I(x) \geq \inf_{p'\in P} \max_{x \in p'} I(x) = \beta.
$$
D'altra parte, per continuit\`a,
$$
\max_{x \in p} I(x) \leq \limsup_{m \to+\infty} \max_{x \in p_m} I(x)
= \beta,
$$
sicch\'e $\max_{x \in p} I(x) = \beta$.

\pg+

Poich\'e $x_1$ e $x_2$ sono due punti di minimo locale stretto
collegati da $p$, risulta $\beta > \max \{ I(x_1),I(x_2)\}$.

\pg+

* Dimostriamo che esiste un punto critico $x_3 \in p$ tale che
  $I(x_3)=\beta$.

\pg+

Procediamo per assurdo. Innanzitutto l'insieme
$$
K = \left\{ x \in p \mid I(x)=\beta \right\}
$$
\`e chiuso (perch\'e?) e limitato, dunque compatto. Supponiamo che $I'
\neq 0$ in $K$.

\pg;

L'ipotesi assurda garantisce l'esistenza di un numero $\delta>0$ tale
che $|I'(x) | \geq 2\delta>0$ per ogni $x \in K$. Per continuit\`a,
esiste un intorno
$$
U_\ge = \left\{ x \in \R^N \mid |x-y| < \ge \ \hbox{per qualche $y \in
K$} \right\}
$$
di $K$ tale che $|I'| \geq \delta$ su $U_\ge$. In particolare, $x_i
\notin U_\ge$, $i=1,2$.

\pg+

Sia $\eta$ una funzione continua con supporto contenuto in $U_\ge$,
tale che $0 \leq \eta \leq 1$ e $\eta \equiv 1$ in un intorno di
$K$.

\pg+

Definiamo $\Phi \colon \R^N \times \R \to \R^N$ mediante
$$
\Phi(x,t) = x - t \eta(x) \nabla I(x).
$$
Un calcolo diretto mostra che
$$
\left. {d \over dt} \right|_{t=0} I(\Phi(x,t)) = -\eta(x) |\nabla
I(x)|^2.
$$

\pg;

Inoltre $|\nabla I(x)|^2 \geq \delta^2 >0$ su $\mathop{\rm supp} \eta
\subset U_\ge$. Per continuit\`a, esiste $T>0$ tale che
$$
{d \over dt} I(\Phi(x,t)) \leq -{\eta(x) \over 2} |\nabla I(x)|^2
$$
per ogni $0 \leq t \leq T$, uniformemente rispetto a $x$.

\pg+

Sia $p_T = \{\Phi(x,T) \mid x \in p\}$; per ogni $\Phi(x,T) \in p_T$
calcoliamo
$$
\eqalign{%
I(\Phi(x,T)) &= I(x) + \int_0^T {d \over dt} I(\Phi(x,t))\, dt \cr
&\leq I(x) - {T \over 2} \eta(x) |\nabla I(x)|^2, \cr
}
$$
e l'ultimo termine \`e $\leq I(x)=\beta$ se $x \notin K$, oppure $\leq
\beta - {T \over 2} \delta^2 < \beta$ se $x \in K$. In ogni caso,
$$
\max_{x \in p_T} I(x) < \beta.
$$
Ma:
\begitems
* $p_T$ \`e compatto e connesso;
* $x_i = \Phi(x_i,T) \in p_T$, $i=1,2$.
\enditems
Pertanto $p_T \in P$, e questo contraddice la definizione di $\beta$.

\pg;

Se tutti i punti critici $u$ di $I$ in $p$ con $I(u)=\beta$ fossero
minimi locali, l'insieme $\tilde{K}$ di tali punti sarebbe aperto in
$p$  e --- per la continuit\`a di $I$ e di $I'$ --- anche chiuso. Ma $
\tilde{K} \neq \emptyset$ per la discussione precedente, dunque
$p=\tilde{K}$. Ci\`o contraddice il fatto che $I(x_1)<\beta$,
$I(x_2)<\beta$.

\pg+

Pertanto almeno un punto critico di $I$ in $p$ al livello $\beta$ non
\`e un minimo locale. La dimostrazione \`e completa.

\pg+

* L'interpretazione di questo teorema \`e suggestiva: $I(x)$ misura
  l'altitudine di un punto $x$ in un panorama. I due minimi $x_1$ e
  $x_2$ corrispondono a due villaggi collocati al fondo di due valli
  separate da una cresta montagnosa. Se camminiamo lungo un sentiero
  $p$ che unisce i due villaggi, scelto in modo che la quota massima
  $I(x)$ raggiunta nei punti $x \in p$ sia minimale tra le quote di
  tutti i possibili sentieri, il teorema afferma che attraverseremo la
  cresta in un punto di sella (che i montanari chiamano anche {\it
  forcella}).

\pg;

* Il teorema precedente \`e la versione finito-dimensionale del {\em
  teorema del passo di montagna}, dimostrato nel 1972 da A. Ambrosetti
  e P. H. Rabinowitz.

\pg+

* La versione ambientata in uno spazio normato di dimensione
  (eventualmente) infinita \`e generalmente falsa senza ulteriori
  ipotesi sul funzionale $I$.

\pg+

* Nalla dimostrazione appena vista, la coercivit\`a di $I$ garantiva
  immediatamente la limitatezza --- e dunque la {\em relativa
  compattezza} --- di tutti i (sotto)livelli di $I$. La compattezza
  (relativa) della palla unitaria di uno spazio di Banach equivale
  per\`o all'avere dimensione finita.

\pg+

* Occorre, nel caso infinito-dimensionale, un surrogato della
  coercivit\`a, che garantisca la necessaria compattezza.
















\pg. %------------------------------FINE-------------------------------
