\input ctuslides2


\def\R{{\bbchar R}}

\def\ge{{\varepsilon}}

\def\frac#1#2{{#1 \over #2}}

\slideshow

\tit Calcolo delle Variazioni\nl
     a.a. 2019-2020\nl

\subtit\bf Simone Secchi\nl simone.secchi@unimib.it

\subtit\rm \url{http://elearning.unimib.it}

\pg;

\sec Prerequisiti e strumenti

* Calcolo differenziale in spazi euclidei di dimensione finita
* Teoria della misura e dell'integrazione secondo Lebesgue
* Principi di Analisi Funzionale Lineare
* Teoria elementare degli spazi di Sobolev (almeno il caso hilbertiano $p=2$)


\pg;

\sec Strumenti: il calcolo differenziale in dimensione infinita

{\bf Notazione.} Se $X$ \`e uno spazio di Banach (reale), il suo duale
topologico sar\`a denotato con il simbolo $X^\*$. Se $A \in X^\*$, il
simbolo $A[v]$ indicher\`a il valore di $A$ nel punto $v$; talvolta
semplificheremo la notazione e scriveremo $Av$ al posto di $A[v]$.

\bigskip

{\bf Definizione.} Siano $X$ uno spazio di Banach, e $U \subset X$ un
suo aperto. Un funzionale su $U$ \`e un'applicazione $I \colon U \to
\R$. Si noti che i ``nostri'' funzionali {\bf non} sono
necessariamente {\bf lineari}!

\pg;

{\bf Definizione.} Sia $I \colon U \to \R$ un funzionale. Diremo che
$I$ \`e derivabile secondo Fr\'echet nel punto $u \in U$ se esiste un
elemento $A \in X^\*$ tale che
$$
\lim_{\|v\| \to 0} {I(u+v)-I(u)-Av \over \|v\|} =0, \eqmark
$$
o, equivalentemente, se
$$
I(u+v)=I(u)+Av+o(\|v\|) \quad\hbox{per $v \to 0$}.
$$

\medskip

Si osservi che questa \`e la definizione di funzione differenziabile
quando $X=\R^n$.

\pg;

{\bf Lemma.} Se $I$ \`e derivabile nel punto $u \in U$, allora
l'elemento $A$ che soddisfa (1) \`e univocamente determinato.

\smallskip

{\em Dim.} Infatti, supponiamo che $A$ e $B$ siano due elementi di
$X^\*$ che soddisfano (1). Per sottrazione,
$$
\lim_{\|v\| \to 0} {(A-B)v \over \|v\|} =0.
$$
Fissiamo $u \in X$ con $\|u\|=1$, e scegliamo $v=tu$, $t \to
0^+$. Allora
$$
(A-B)u=\lim_{t \to 0+} {t(A-B)u \over t\|u\|} =0.
$$
Per l'arbitrariet\`a di $u$, concludiamo che $A=B$.

\pg;

{\bf Definizione.} Se $I$ \`e un funzionale derivabile secondo
Fr\'echet nel punto $u \in U$, la derivata (talvolta: il
differenziale) di Fr\'echet di $I$ in $u$ \`e l'unico elemento $I'(u)
\in X^\*$ (talvolta: $dI(u)$) tale che
$$
I(u+v)=I(u)+I'(u)[v]+o(\|v\|)
$$
per $v \to 0$.

\bigskip

{\bf Definizione.} Se $I \colon U \to \R$ \`e derivabile secondo
Fr\'echet in ogni punto $u \in U$, diremo che $I$ \`e
Fr\'echet-derivabile in $U$. La derivata di Fr\'echet di $I$ \`e
allora la mappa $I' \colon U \to X^\*$ che ad $u \in U$ associa $I'(u)
\in X^\*$. Si tratta --- in generale --- di una mappa {\em non
  lineare}.

Se $I'$ \`e una mappa continua da $U$ in $X^\*$, diremo che $I \in C^1(U)$.

\pg;

\sec Il caso hilbertiano

Se $H$ \`e uno spazio di Hilbert (reale), \`e noto che gli elementi
del duale $H^\*$ sono isometricamente identificati con vettori di $H$
attraverso l'isomorfismo di Riesz. In particolare, un funzionale $I$
definito su $U \subset H$ \`e derivabile in $u \in U$ se e solo esiste
un vettore, detto d'ora in poi {\em gradiente} di $I$ in $u$ e
denotato $\nabla I(u)$, tale che
$$
I(u+v) = I(u) + \langle \nabla I(u) \mid v \rangle + o(\|v\|)
$$
per $v\to 0$.

\pg;

{\bf Proposizione.} Siano $I$ e $J$ due funzionali derivabili nel
punto $u \in X$. Allora valgono le seguenti affermazioni.
\begitems
\style n
* Se $a$ e $b$ sono numeri reali, allora $aI+bJ$ \`e derivabile in
$u$, e vale $(aI+bJ)'(u)=aI'(u)+bJ'(u)$.
* Il prodotto $IJ$ \`e derivabile in $u$, e vale $(IJ)'(u) =
J(u)I'(u)+I(u)J'(u)$.
* Se $\gamma \colon \R \to U$ \`e una curva derivabile in $t_0$ e
$u=\gamma(t_0)$, allora la composizione $\eta \colon \R \to \R$
definita da $\eta(t)=I(\gamma(t))$ \`e derivabile in $t_0$, e vale
$\eta'(t_0)=I'(u)[\gamma'(t_0)]$.
* Se $A \subset \R$ \`e un aperto, $f \colon A \to \R$ \`e derivabile in
$I(u) \in A$, allora la composizione $K(u)=f(I(u))$ \`e definita in un
intorno $V$ di $u$, \`e derivabile in $u$ e vale $K'(u)=f'(I(u))I'(u)$.
\enditems

\pg;

{\em Dim.} La prima affermazione \`e banale (esercizio!). Per quanto
riguarda la seconda, quando $v \to 0$ in $X$, abbiamo
$$
\eqalign{%
I(u+v)J(u+v) &= \left( I(u)+I'(u)[v]+o(\|v\|) \right) \left(
  J(u)+J'(u)[v]+o(\|v\|) \right) \cr
&= I(u)J(u) + J(u)I'(u)[v] + I(u)J'(u)[v] + I'(u)[v] J'(u)[v] \cr
&\quad {} + o(\|v\|) \left( I(u)+I'(u)[v]+J(u)+J'(u)[v]+o(\|v\|)
\right). \cr
}%
$$
Concludiamo osservando che
$$
I'(u)[v] J'(u)[v]
+ o(\|v\|) \left( I(u)+I'(u)[v]+J(u)+J'(u)[v]+o(\|v\|) \right)
$$
\`e $o(\|v\|)$ per $v \to 0$. La terza affermazione \`e simile,
infatti per $h \to 0$ in $\R$
$$
\eqalign{%
  \eta(t_0+h) &= I(\gamma (t_0+h)) = I(\gamma(t_0)+\gamma'(t_0)h +
  o(|h|)) \cr
  &= I(u) + I'(u)[\gamma'(t_0)h+o(|h|)] + o(\|\gamma'(t_0)h+o(|h|)\|)
  \cr
  &= \eta(t_0) +I'(u)[\gamma'(t_0)h] +I'(u)[o(|h|)] +
  o(\|\gamma'(t_0)h+o(|h|)\|). \cr
}%
$$
Poich\'e gli ultimi due addendi sono $o(|h|)$, otteniamo che
$$
\eta(t_0+h) = \eta(t_0)+I'(u)[\gamma'(t_0)h]+o(|h|).
$$

\pg;

Infine, quando $v \to 0$ in $X$, si verifica come prima che
$$
\eqalign{%
  K(u+v)&= f(I(u+v)) = f(I(u)+I'(u)[v]+o(\|v\|)) \cr
  &= f(I(u))+f'(I(u))(I'(u)[v]+o(\|v\|))+o(I'(u)[v]+o(\|v\|)) \cr
  &= f(I(u)) + f'(I(u))I'(u)[v]+o(\|v\|).
}%
$$

\bigskip

{\bf Osservazione.} \`E possibile introdurre il concetto di derivata
per applicazioni tra due spazi di Banach $X$ e $Y$. Solo in questo
contesto pu\`o essere enunciata una formulazione completa della regola
di derivazione delle funzioni composte.

Poich\'e non ne faremo uso in queste lezioni, rimandiamo al testo di
Ambrosetti e Prodi per ulteriori approfondimenti.

\pg;

{\bf Definizione.} Sia $I$ un funzionale definito nell'aperto $U$ di
$X$, e sia $u \in U$. Diremo che $I$ \`e derivabile secondo G\^ateaux
in $u$ se esiste un elemento $A \in X^\*$ tale che
$$
\lim_{t \to 0} {I(u+tv)-I(u) \over t} = Av \eqmark
$$
per ogni $v \in X$. In tal caso, l'unico (esercizio!) elemento
siffatto prende il nome di derivata secondo G\^ateaux di $I$ in $u$, e
si denota con $I'_G(u)$ o con $d_GI(u)$.

\bigskip

Osserviamo che questa nuova derivata riprende la cosiddetta {\em
  derivata direzionale} gi\`a nota nell'ambito del calcolo
differenziale in dimensione finita.

In particolare, ricordando i ``soliti'' esempi in $\R^2$, deduciamo
che esistono funzionali (non lineari) derivabili secondo G\^ateaux ma
non derivabili secondo Fr\'echet.

\pg;

\sec Condizione sufficiente per la derivabilit\`a secondo Fr\'echet

{\bf Proposizione.} Supponiamo che $U \subset X$ sia un aperto, che
$I$ sia G\^ateaux-derivabile in $U$, e che $I'_G$ sia continua in un
punto $u \in U$. Allora $I$ \`e Fr\'echet-derivabile in $u$, e
(ovviamente) $I'(u)=I'_G(u)$.

\bigskip

Omettiamo la dimostrazione, che \`e probabilmente stata proposta nel
caso $X=\R^2$ nel corso di Analisi Matematica 2.

\pg;

\sec Punti critici

{\bf Definizione.} Siano $X$ uno spazio di Banach, $U$ un aperto di
$X$, e $I$ un funzionale definito su $U$. Diremo che $u \in U$ \`e un
punto critico di $I$ se $I$ \`e derivabile in $u$ e
$$
I'(u)=0.
$$
Pi\`u esplicitamente, questo significa che $I'(u)[v]=0$ per ogni $v
\in  X$.

Se $u$ \`e un punto critico di $I$ e $I(u)=c$, diremo che $u$ \`e un
punto critico (di $I$) al livello $c$. Se, per qualche $c \in \R$,
l'insieme $I^{-1}(\{c\})\subset X$ contiene almeno un elemento, diremo
che $c$ \`e un valore critico per $I$.

\bigskip

L'equazione $I'(u)=0$ \`e nota come equazione di Eulero (o di
Eulero-Lagrange) associata al funzionale $I$.

\pg;

\sec Esempi

{\bf Esempio 1.} Ogni $A \in X^\*$ \`e derivabile. Infatti, basta
scrivere
$$
A[u+v]=Au+Av
$$
per dedurre che $A'(u)=A$ per qualsiasi $u \in X$.

\medskip

{\bf Esempio 2.} Sia $X$ uno spazio di Banach, e sia $a \colon X
\times X \to \R$ una forma bilineare continua. Denotiamo con $J \colon
X \to \R$ il funzionale definito da $J(u)=a(u,u)$ per ogni $u \in
X$. Allora $J$ \`e derivabile in $X$. Infatti
$$
\eqalign{%
J(u+v) &=a(u+v,u+v) = a(u,u)+a(u,v)+a(v,u)+a(v,v) \cr
&= J(u) +a(u,v)+a(v,u)+a(v,v).
}%
$$
Poich\'e $|a(v,v)| \leq M \|v\|^2$ per l'ipotesi di continuit\`a di
$a$ come forma bilineare, deduciamo che $a(v,v)=o(\|v\|)$ per $v \to
0$, e dunque che
$$
J'(u)[v]=a(u,v)+a(v,u).
$$

\pg;

{\bf Esempio 3.} (esercizio) Sia $H$ uno spazio di Hilbert con norma $\| \cdot
\|$. Il funzionale $J(u)=\|u\|$ \`e derivabile in ogni punto $u \neq
0$, e risulta
$$
\nabla J(u) = {u \over \|u\|}.
$$

\medskip

{\bf Esempio 4.} Sia $X$ uno spazio di Banach, e siano $I$, $J$ due
funzionali derivabili in $X$. Definiamo
$$
Q(u) = {I(u) \over J(u)}
$$
sul sottoinsieme (aperto) $\{u \in  X \mid J(u) \neq 0\}$. Per la
Proposizione sulle regole di calcolo dimostrata sopra, possiamo
affermare che $Q$ \`e derivabile e che
$$
Q'(u) = {J(u) I'(u)[v] - I(u) J'(u)[v] \over J(u)^2}
$$
per ogni $u\ in X$ tale che $J(u) \neq 0$.

\pg;

\sec Esempi in spazi concreti

{\bf Esempio 5.} Sia $\Omega \subset \R^N$, $N \geq 1$, un insieme
aperto e limitato. Definiamo i funzionali
$$
I \colon L^2(\Omega) \to \R, \quad I(u) = \int_\Omega |u(x)|^2 \, dx,
$$
$$
J \colon H_0^1(\Omega) \to \R, \quad J(u) = \int_\Omega |\nabla
u(x)|^2 \, dx,
$$
$$
K \colon H^1(\Omega) \to \R, \quad K(u) = \int_\Omega |\nabla
u(x)|^2 \, dx,
$$
$$
L \colon H^1(\Omega) \to \R, \quad L(u) = \int_\Omega |\nabla
u(x)|^2 \, dx + \int_\Omega |u(x)|^2 \, dx.
$$
Trattandosi di forme quadratiche associate a forme bilineari continue,
sappiamo gi\`a che i quattro funzionali sono derivabili.

\pg;

Esplicitamente, valgono le relazioni
$$
\eqalign{%
\nabla I(u) &= 2u \cr
\nabla L(u) &= 2u \cr
\nabla J(u) &= 2u.
}
$$
Un
calcolo diretto mostra che
$$
K'(u)[v] = 2 \int_\Omega \nabla u(x) \cdot \nabla v(x) \, dx
$$
per ogni $u$, $v \in H^1(\Omega)$, ma non siamo autorizzati ad
affermare che $\nabla K(u)=2u$ (perch\'e?)

\pg;

\sec Inversione della Convergenza Dominata

{\bf Teorema di Lebesgue.} Sia $\Omega$ un aperto di $\R^N$, e sia
$\{u_k\}_k$ una successione in $L^1(\Omega)$ tale che
\begitems
\style n
* $u_k(x) \to u(x)$ per q.o $x \in \Omega$;
* esiste $v \in L^1(\Omega)$ tale che $|u_k(x)| \leq v(x)$ per q.o. $x
\in \Omega$ e ogni $k$.
\enditems
Allora $u \in L^1(\Omega)$ e $u_k \to u$ nella norma di $L^1(\Omega)$.

\bigskip

Questo risultato fondamentale di Teoria della Misura pu\`o essere {\em
parzialmente} invertito, come mostra il seguente teorema. Per la
dimostrazione, rimandiamo al libro di H. Brezis, Analisi funzionale.

\pg;

{\bf Teorema.} Sia $\Omega$ un aperto di $\R^N$, e sia $\{u_k\}_k$ una
successione di $L^p(\Omega)$, $p \in [1,+\infty]$, tale che $u_k \to
u$ in $L^p(\Omega)$. Allora esistono una sottosuccessione
$\{u_{k_j}\}_j$ ed una funzione $v \in L^p(\Omega)$ tali che
\begitems
\style n
* $u_{k_j}(x) \to u(x)$ per q.o. $x \in \Omega$;
* per ogni $j$, $|u_{k_j}(x)| \leq v(x)$ per q.o. $x \in \Omega$.
\enditems

\bigskip

Questo teorema mostra che la convergenza forte in $L^p$ implica --- a
meno di sottosuccessioni --- l'esistenza di una funzione dominante.

\pg;

\sec Operatori di Nemitskii

Siano $\Omega$ un aperto limitato di $\R^N$, $N \geq 3$, con frontiera
regolare, e sia $f \colon \R \to \R$ una funzione continua. Supponiamo
che esistano $a>0$ e $b>0$ tali che
$$
|f(t)| \leq a+b|t|^{2^\*-1},
$$
dove $2^\* = 2N/(N-2)$ \`e l'esponente critico di Sobolev. Definiamo
$$
F(t) = \int_0^t f(x) \, dx
$$
e consideriamo il funzionale $J \colon H^1(\Omega) \to \R$ dato da
$$
J(u) = \int_\Omega F(u(x))\, dx.
$$

\pg;

{\bf Proposizione.} Sotto le ipotesi precedenti, $J$ \`e un funzionale
derivabile in $H^1(\Omega)$, e vale
$$
J'(u)[v] = \int_\Omega f(u(x))v(x)\, dx
$$
per ogni $u$, $v \in H^1(\Omega)$.

\bigskip

La dimostrazione non \`e immediata: mostriamo prima che $J$ \`e
G\^a\-te\-aux-derivabile, e poi che la derivata di G\^ateaux \`e
continua. Come abbiamo visto sopra, ci\`o implica che $J$ \`e
Fr\'echet-derivabile.

\pg;

* Derivata di G\^ateaux

Per q.o. $x\in\Omega$, risulta
$$
\lim_{t \to 0} {F(u(x)+t(v(x)) - F(u(x)) \over t} = f(u(x))v(x).
$$
Per il teorema di Lagrange, esiste un numero reale $\theta$ tale che
$|\theta| \leq |t|$ e
$$
\eqalign{%
\left| {F(u(x)+t(v(x)) - F(u(x)) \over t} \right| &= \left|
f(u(x)+\theta v(x))v(x) \right| \cr
&\leq \left( a+b|u(x)+\theta v(x)|^{2^\*-1} \right) |v(x)| \cr
&\leq C \left( |v(x)|+|u(x)|^{2^\*-1}|v(x)|+|v(x)|^{2^\*} \right).
}
$$
Per Convergenza Dominata,
$$
\lim_{t \to 0} \int_\Omega {F(u(x)+t(v(x)) - F(u(x)) \over t}\, dx =
\int_\Omega f(u(x))v(x)\, dx.
$$

\pg;

Poich\'e $v \mapsto \int_\Omega f(u(x))v(x)\, dx$ \`e un operatore
lineare e continuo  in $H^1(\Omega)$ (disuguaglianza di H\"older e di
Sobolev), abbiamo individuato la derivata secondo G\^ateaux di $J$:
$$
J'_G(u)[v] = \int_\Omega f(u(x))v(x)\, dx.
$$

* Derivata di Fr\'echet

Mostriamo che $J'_G \colon H^1(\Omega) \to (H^1(\Omega))^\*$ \`e
un'applicazione continua. A tal fine, sia $\{u_k\}_k$ una successione
che converge a $u$ in $H^1(\Omega)$. Per il teorema di convergenza
dominata inversa, possiamo supporre che --- a meno di sottosuccessioni
---
\begitems
* $u_k \to u$ in $L^{2^\*}(\Omega)$;
* $u_k(x) \to u(x)$ per q.o. $x \in \Omega$;
* esiste $w \in L^{2^\*}(\Omega)$ tale che $|u_k(x)| \leq w(x)$ per
q.o. $x \in \Omega$ e ogni $k$.
\enditems

\pg;

Usiamo la disuguaglianza di H\"older:
$$
\eqalign{%
\left| (J'_G(u_k)-J'_G(u))[v] \right| &\leq \int_\Omega
|f(u_k(x))-f(u(x))| |v(x)| \, dx \cr
&\leq \left( \int_\Omega |f(u_k(x))-f(u(x))|^{2^\* \over 2^\*-1}\, dx
\right)^{2^\*-1 \over 2^\*} \times \cr
&\quad \times \left( \int_\Omega |v(x)|^{2^\*}\, dx
\right)^{1/2^\*}.
}
$$
La continuit\`a di $f$ implica $\lim_{k \to +\infty}
|f(u_k(x))-f(u(x))|=0$ per q.o. $x\in\Omega$, e inoltre
$$
\eqalign{%
|f(u_k(x))-f(u(x))|^{2^\* \over 2^\* -1} &\leq C \left(
1+|u_k(x)|^{2^\*-1} + |u(x)|^{2^\* -1} \right)^{2^\* \over 2^\* -1} \cr
&\leq C \left(
1+|w(x)|^{2^\*-1} + |w(x)|^{2^\* -1} \right)^{2^\* \over 2^\* -1} \cr
&\leq C \left(
1+|w(x)|^{2^\*} + |w(x)|^{2^\* } \right) \in L^1(\Omega).  \cr
}
$$
Per Convergenza Dominata,
$$
\lim_{k \to +\infty} \int_\Omega |f(u_k(x))-f(u(x))|^{2^\* \over 2^\*
-1} \, dx =0.
$$

\pg;

Perci\`o
$$
\eqalign{%
\|J'_G(u_k)-J'_G(u)\| &= \sup \{ (J'_G(u_k)-J'_G(u))[v] \mid v \in
H^1(\Omega),\ \|v\|=1 \} \cr
&\leq C \left( \int_\Omega |f(u_k(x))-f(u(x))|^{2^\* \over 2^\*
-1} \, dx \right)^{2^\* -1 \over 2^\*} \to 0.
}
$$
Riassumendo: abbiamo dimostrato che da ogni successione $\{u_k\}_k$
convergente a $u$ \`e possibile estrarre una sottosuccessione tale che
$J'_G(u_k) \to J'_G(u)$ in $(H^1(\Omega))^\*$. \`E ora un esercizio di
Topologia Generale dedurre che l'intera successione $\{u_k\}_k$ gode
di questa propriet\`a (perch\'e il limite \`e indipendente dalla
sottosuccessione scelta).

\pg;

\`E possibile estendere quanto dimostrato al caso in cui $\Omega$ sia
un aperto qualunque, anche illimitato. Il prezzo da pagare \`e un
rafforzamento delle ipotesi sulla funzione $f$

\pg+

Sia dunque $\Omega$ un aperto di $\R^N$ con frontiera regolare, e sia
$f \colon \R \to \R$ una funzione continua e tale che
$$
|f(t)| \leq a|t| + b |t|^{2^\*-1}.
$$
Dimostriamo che il funzionale $J(u)=\int_\Omega F(u(x))\, dx$ \`e
derivabile in $H^1(\Omega)$.

\pg+

* Derivata di G\^ateaux

Per q.o. $x \in \Omega$ e per ogni $v \in H^1(\Omega)$,
$$
\lim_{t \to 0} { F(u(x)+tv(x)) - F(u(x)) \over t} = f(u(x))v(x).
$$


\pg;

Per il teorema di Lagrange, esiste $\theta=\theta(x)$ tale che
$|\theta| \leq |t|$ e
$$
\eqalign{%
\left| { F(u(x)+tv(x)) - F(u(x)) \over t}\right| &= \left|
f(u(x)+\theta v(x))v(x) \right| \cr
&\leq C \left( |u(x)+\theta v(x)| + |u(x)+\theta v(x)|^{2^\* -1} \right)
\cr
&\leq C \left( |u(x)| |v(x)| + |v(x)|^{2^\*} + |u(x)|^{2^\* -1} |v(x)|
+ |v(x)|^{2^\*} \right) \cr
&\in L^{1}(\Omega). \cr
}
$$
Concludiamo ancora per Convergenza Dominata.

\pg+

Se poi $\{u_k\}_k$ \`e una successione che tende a $u$ in
$H^1(\Omega)$, a meno di sottosuccessioni possiamo anche supporre che
\begitems
* $u_k(x) \to u(x)$ per q.o. $x\in \Omega$
* $u_k \to u$ in $L^2(\Omega)$ e in $L^{2^\*}(\Omega)$
* esistono $w_1 \in L^{2^\*}(\Omega)$ e $w_2 \in L^2(\Omega)$ tali che
$|u_k(x)| \leq w_i(x)$, $i=1$, $2$, per q.o. $x \in \Omega$.
\enditems

\pg;

Sia $\varepsilon>0$, e scegliamo $R_\varepsilon>0$ tale che
$$
\|u\|_{L^2(\Omega_\ge)} + \|u\|_{L^{2^\*}(\Omega_\ge)}^{2^\* -1} +
\|w_1\|_{L^{2^\*}(\Omega_\ge)}^{2^\* -1} + \|w_2\|_{L^2(\Omega_\ge)}
\leq \ge,
$$
dove $\Omega_\ge = \{x \in \Omega \mid |x| > R_\ge \}$. Ora,
$$
\eqalign{%
\left| (J'_G(u_k)-J'_G(u))[v] \right| &\leq \int_\Omega |f(u_k)-f(u)|
|v| \, dx \cr
&= \int_{\Omega \cap B(0,R_\ge)} |f(u_k)-f(u)|
|v| \, dx + \int_{\Omega_\ge} |f(u_k)-f(u)|
|v| \, dx. \cr
}
$$
Trattiamo separatamente gli ultimi due integrali.

\pg;

Innanzitutto
$$
\eqalign{%
&\int_{\Omega_\ge} |f(u_k)-f(u)|
|v| \, dx \cr
&\leq C \int_{\Omega_\ge} \left( |u_k|+|u|+|u_k|^{2^\* -1} + |u|^{2^\*
-1} \right)|v| \, dx \cr
&\leq C \left( \int_{\Omega_\ge} |w_2| |v| \, dx + \int_{\Omega_\ge}
|u| |v| \, dx  + \int_{\Omega_\ge} |w_1|^{2^\* -1} |v| \, dx +
\int_{\Omega_\ge} |u|^{2^\* -1} |v| \, dx \right) \cr
&\leq C \|v\| \left( \|u\|_{L^2(\Omega_\ge)} + \|u\|_{L^{2^\*}(\Omega_\ge)}^{2^\* -1} +
\|w_1\|_{L^{2^\*}(\Omega_\ge)}^{2^\* -1} + \|w_2\|_{L^2(\Omega_\ge)}
\right) \cr
&\leq C \|v\| \ge.
}
$$

\pg;

D'altra parte,
$$
\int_{\Omega \cap B(0,R_\ge)} |f(u_k)-f(u)| |v| \, dx
\leq C \left( \int_{\Omega \cap B(0,R_\ge)} |f(u_k)-f(u)|^{2^\* \over
2^\* -1} \, dx \right)^{2^\* -1 \over 2^\*} \|v\|.
$$
Sui sottoinsiemi limitati di $\R$, la funzione $f$ soddisfa una
maggiorazione del tipo $|f(t)| \leq C (1+|t|^{2^\* -1} )$, e come
sopra concludiamo che
$$
\lim_{k \to +\infty} \int_{\Omega \cap B(0,R_\ge)} |f(u_k)-f(u)|^{2^\* \over
2^\* -1} \, dx =0.
$$
Ricapitolando,
$$
\eqalign{%
\| (J'_G(u_k)-J'_G(u))\| &= \sup \left\{ (J'_G(u_k)-J'_G(u))[v] \mid v \in H^1(\Omega),\ \|v\|=1
\right\} \cr
&\leq C \left( \int_{\Omega \cap B(0,R_\ge)} |f(u_k)-f(u)|^{2^\* \over
2^\* -1} \, dx \right)^{2^\* -1 \over 2^\*} + C \ge \cr
&= o(1) + C \ge.
}
$$
Per l'arbitrariet\`a di $\ge>0$, concludiamo che $J'_G(u_K) \to
J'_G(u)$.

\pg;

{\bf Osservazione.} La regolarit\`a della frontiera di $\Omega$ \`e
stata utilizzata solo {\em implicitamente} per garantire la validit\`a
di tutte le immersioni di Sobolev. Ne consegue che gli stessi
risultati sussistono, senza alcuna ipotesi su $\partial \Omega$, se
restringiamo il funzionale $J$ al sottospazio $H_0^1(\Omega)$.

\pg;

\sec Un problema lineare ellittico

Prenderemo a modello di applicazione
un'equazione alle derivate parziali del secondo ordine, avente la
forma
$$
\left\{
\matrix{ -\Delta u + q(x) u = h(x), \hfill &x \in \Omega \cr
u(x)=0, \hfill &x \in \partial \Omega \cr}
\right. \eqno(P)
$$
dove
\begitems
* $\Omega$ \`e un aperto limitato di $\R^N$
* $q\in C(\Omega)$, $h \in C(\Omega)$.
\enditems

\pg+

Il problema (P) prende il nome di {\em problema di Dirichlet
omogeneo}. L'aggettivo {\em omogeneo} si riferisce qui alla condizione
{\em al bordo} $u=0$ su $\partial\Omega$. Osserviamo che il problema
\`e {\em lineare}.

\pg;

* Una {\em soluzione classica} di (P) \`e una funzione $u \in
  C^2(\overline{\Omega})$ tale che (P) sia soddisfatto puntualmente in
  $\overline{\Omega}$.

\pg+

Fissiamo $v \in C_0^1(\Omega)$ e moltiplichiamo l'equazione in (P) per
$v$. Integrando su $\Omega$ con l'ausilio del Teorema di Stokes
(versione nota anche come {\em formula di Gauss-Green}), otteniamo che
$$
\int_\Omega \nabla u \cdot \nabla v \, dx + \int_\Omega q(x) u v \, dx
= \int_\Omega h(x)v \, dx.
$$
Questa uguaglianza ha senso sotto ipotesi ben pi\`u deboli di quelle
finora assunte. Ad esempio gli integrali sono finiti quando $u$, $v$
sono funzioni di $L^2(\Omega)$ tali che $\partial u / \partial x_i$ e
$\partial v / \partial x_i$ appartengano ad $L^2(\Omega)$ per ogni
indice $i$. La continuit\`a di $q$ e $h$ \`e allora eccessiva, e
possiamo sostituirla con $q \in L^\infty(\Omega)$, $h \in
L^2(\Omega)$.

\pg;

* Siano dunque $q \in L^\infty(\Omega)$, $h \in
L^2(\Omega)$. Una {\em soluzione debole} di (P) \`e una funzione $u
\in H_0^1(\Omega)$ tale che
$$
\int_\Omega \nabla u \cdot \nabla v \, dx + \int_\Omega q(x) u v \, dx
= \int_\Omega h(x) v \, dx
$$
per ogni $v \in H_0^1(\Omega)$.

\pg+

{\bf Osservazione.} Ogni soluzione classica \`e anche soluzione
debole.

Infatti, $u \in C^2(\overline{\Omega})$ implica $u \in
H^1(\Omega)$. Per una nota propriet\`a degli spazi di Sobolev,
poich\'e $u$ \`e continua in $\overline{\Omega}$ e $u=0$ su
$\partial\Omega$, abbiamo $u \in H_0^1(\Omega)$.

Sappiamo che, per ogni $v \in C_0^1(\Omega)$,
$$
\int_\Omega \nabla u \cdot \nabla v \, dx + \int_\Omega q(x) u v \, dx
= \int_\Omega h(x)v \, dx.
$$
Poich\'e $C_0^1(\Omega)$ \`e un sottospazio denso di $H_0^1(\Omega)$,
ad ogni $v \in H_0^1(\Omega)$ facciamo corrispondere una successione
$\{v_n\}_n \subset C_0^1(\Omega)$ tale che $v_n \to v$ in
$H_0^1(\Omega)$.

\pg;

Facendo tendere $n \to +\infty$ nella relazione
$$
\int_\Omega \nabla u \cdot \nabla v_n \, dx + \int_\Omega q(x) u v_n \, dx
= \int_\Omega h(x)v_n \, dx,
$$
deduciamo che $u$ \`e una soluzione debole di (P).

\pg+

\`E ragionevole chiedersi se ogni soluzione debole sia anche una
soluzione classica. Vediamo che cosa possiamo dire.

\pg+

Sia $u \in H_0^1(\Omega)$ una soluzione debole di (P), e supponiamo
che $h \in C(\overline{\Omega})$. {\em Se} \`e noto, per qualche
motivo, che $u \in C^2(\Omega)$, allora possiamo dedurre che $u=0$ su
$\partial\Omega$.

\pg;

Scegliendo in particolare $v \in C_0^1(\Omega)$ nella definizione di
soluzione debole, otteniamo che, per ogni $v \in C_0^1(\Omega)$,
$$
\int_\Omega \nabla u \cdot \nabla v \, dx + \int_\Omega q(x) u v \, dx
= \int_\Omega h(x)v \, dx.
$$

\pg+

Usando nel senso contrario la formula di Stokes, arriviamo alla
relazione
$$
\int_\Omega \left( -\Delta u + q(x) u -h(x) \right) v\, dx=0
$$
per ogni $v \in C_0^1(\Omega)$.

\pg+

Per densit\`a di $C_0^1(\Omega)$ in $L^2(\Omega)$, concludiamo che
$-\Delta u + q(x) u -h(x)=0$ quasi ovunque, e che $u=0$ quasi ovunque
in $\partial\Omega$.

\pg+

* Morale della favola: abbiamo bisogno di una {\em teoria della
regolarit\`a} per le soluzioni deboli di (P).

\pg;

\sec Soluzioni deboli e punti critici

Definiamo il funzionale $J \colon H_0^1(\Omega) \to \R$ mediante la formula
$$
J(u) = \frac{1}{2} \int_\Omega |\nabla u|^2 \, dx + \frac{1}{2}
\int_\Omega q(x) |u|^2 \, dx - \int_\Omega h(x) u \, dx.
$$

\pg+

Segue dagli esempi sulla derivabilit\`a che $J$ \`e derivabile secondo
Fr\'echet e che
$$
J'(u)[v] = \int_\Omega \nabla u \cdot \nabla v\, dx + \int_\Omega q(x)
u v \, dx - \int_\Omega h(x) v \, dx
$$
per ogni $u$, $v \in H_0^1(\Omega)$.

\pg+

Quindi le soluzioni deboli di (P) sono esattamente i punti critici del
funzionale $J$.

\pg+

Il funzionale $J$ \`e chiamato {\em funzionale dell'energia} associato
a (P), anche se dovremmo chiamarlo pi\`u propriamente funzionale di
azione o di Eulero-Lagrange.

\pg;

\sec Un problema non lineare

Molti modelli della Fisica Moderna conducono ad equazioni {\em non
lineari}. Vediamo come la discussione precedente possa essere estesa
ad un prototipo di equazione alle derivate parziali {\em semilineare}.

\pg+

Sia $\Omega$ un aperto limitato di $\R^N$. Supponiamo che $q \in
L^\infty(\Omega)$ e che $f \colon \R \to \R$ sia una funzione continua
e tale che
$$
|f(t)| \leq a + b |t|^{2^\* -1}.
$$
\pg+
Consideriamo il problema
$$
\left\{
\matrix{-\Delta u + q(x) u = f(u), \hfill &x \in \Omega \cr
u=0, \hfill &x \in \partial \Omega \cr}
\right. \eqno(SP)
$$

\pg;

{\bf Definizione.} Una soluzione debole di (SP) \`e una funzione $u
\in H_0^1(\Omega)$ tale che
$$
\int_\Omega \nabla u \cdot \nabla v \, dx + \int_\Omega q(x) uv \, dx
= \int_\Omega f(u)v \, dx
$$
per ogni $v \in H_0^1(\Omega)$.

\pg+

Sia $F(t) = \int_0^t f(x)\, dx$, e definiamo un funzionale $J \colon
H_0^1(\Omega) \to \R$ mediante la formula
$$
J(u) = \frac{1}{2} \int_\Omega |\nabla u|^2 \, dx + \frac12
\int_\Omega q(x)|u|^2 \, dx - \int_\Omega F(u)\, dx.
$$

\pg+

Sappiamo che $J$ \`e derivabile e che
$$
J'(u)[v] = \int_\Omega \nabla u \cdot \nabla v \, dx - \int_\Omega
q(x) uv\, dx - \int_\Omega f(u)v\, dx
$$
per ogni $v \in H_0^1(\Omega)$.

\pg+

* Ancora una volta, le soluzioni deboli di (SP) corrispondono ai punti
  critici del funzionale dell'energia $J$.

\pg;

\sec Riassunto

* Abbiamo visto che \`e possibile estendere il calcolo differenziale
elementare (cio\`e quello delle funzioni di pi\`u variabili) alle
funzioni di {\em infinite} variabili.

* Con questo linguaggio, abbiamo messo in corrispondenza biunivoca
  opportune soluzioni di equazioni differenziali con gli zeri della
  derivata di opportuni funzionali (non lineari).

\pg;

\sec Prospettive

* Ci prefiggiamo ora di... andare a caccia dei punti critici, al fine
  di {\em risolvere} equazioni differenziali.

* Per far ci\`o, vedremo che occorrono strumenti nuovi, e che la {\em
  topologia} dello spazio di riferimento avr\`a un ruolo fondamentale.

\pg;

\sec Problemi (in tutti i sensi) di minimizzazione

Uno dei pi\`u importanti teoremi dell'Analisi Matematica recita:

\medskip

{\bf Teorema.} Ogni funzione reale continua su un insieme compatto di $\R^N$
possiede massimi e minimi assoluti.

\pg+

Questo enunciato continua a sussistere per funzioni continue definite
su spazi metrici compatti, con dimostrazione sostanzialmente identica.

\pg+ Il ruolo della compattezza nel Teorema di Weierstrass \`e
fondamentale, come mostra il seguente controesempio, dovuto anch'esso
a Weierstrass.

\pg;

{\bf Esempio.} Sia
$$
I(u) = \int_{-1}^1 \left| x u'(x) \right|^2 \, dx
$$
definito per ogni funzione $u \in C^1([-1,1])$ a valori reali. Il
problema
$$
\min_{u \in X} I(u),
$$
dove $X = \left\{ u \in C^1([-1,1]) \mid u(\pm 1) = \pm 1 \right\}$
non ha soluzioni.

\pg+

Infatti, la famiglia di funzioni
$$
u_\ge (x) = \frac{\arctan(x/\ge)}{\arctan(1/\ge)}
$$
mostra con un calcolo diretto che $\inf_X I=0$. \`E poi evidente che
$I(u)=0$ implica $u'=0$ in $[-1,1]$, cio\`e $u$ \`e costante. Pertanto
$u \notin X$.

\pg;

\sec Weierstrass in astratto

{\bf Teorema.} Sia $M$ uno spazio topologico di Hausdorff, e
supponiamo che $I \colon M \to \R \cup \{+\infty\}$ soddisfi la
seguente condizione:

\smallskip

Per ogni $\alpha \in \R$, l'insieme $K_\alpha = \{ u \in M \mid I(u)
\leq \alpha \}$ \`e compatto.

\smallskip

Allora $I$ raggiunge il suo estremo inferiore $\inf_M I$.

\medskip

{\em Dim.} Possiamo evidentemente supporre che $I$ non sia
identicamente uguale a $+\infty$. Poniamo
$$
\alpha_0 = \inf_M I \geq -\infty,
$$
e consideriamo una successione $\{\alpha_m\}_m$ strettamente
decrescente verso $\alpha_0$. Poniamo per brevit\`a
$K_m=K_{\alpha_m}$.
\pg+
Per ipotesi, ogni $K_m$ \`e compatto e non-vuoto. Inoltre $K_m \supset
K_{m+1}$. Per la propriet\`a dell'intersezione finita, esiste
$$
u \in \bigcap_{m \in {\bbchar N}} K_m,
$$
cio\`e $I(u) \leq \alpha_m$ per ogni $m$. Facendo tendere $m \to
+\infty$, concludiamo che $I(u) \leq \alpha_0$, cio\`e $u$ \`e un
minimo assoluto di $I$ su $M$.

\pg;

* Nell'ipotesi del Teorema precedente, per ogni $\alpha \in \R$
  l'insieme
  $$
  \left\{ u \in M \mid I(u) > \alpha \right\} = M \setminus K_\alpha
  $$
  \`e aperto in $M$. Questo significa, per definizione, che $I$ \`e
  una funzione {\em semicontinua inferiormente} su $M$.

* Nei casi concreti, la struttura di $M$ pu\`o essere pi\`u ricca di
  quella di un mero spazio topologico. Di seguito un caso piuttosto
  frequente nell'Analisi Variazionale.

\pg;

{\bf Teorema.} Sia $V$ uno spazio di Banach riflessivo con norma $\|
\cdot \|$, e sia $M \subset V$ un sottospazio debolmente
chiuso. Supponiamo che $I \colon M \to \R \cup \{+\infty\}$ sia un
funzionale tale che
\begitems
* $I(u) \to +\infty$ se $\|u\| \to +\infty$;
* per ogni $u \in M$ ed ogni successione $\{u_m\}_k$ in $M$ tale che
$u_m \rightharpoonup u$, risulta: $I(u) \leq \liminf_{m \to +\infty}
I(u_m)$.
\enditems
Allora $I$ \`e limitato dal basso, e raggiunge il suo minimo assoluto.

\medskip

{\em Dim.} Sia $\alpha_0 = \inf_M I$ e sia $\{u_m\}_m$ una successione
in $M$ tale che $I(u_m) \to \alpha_0$ per $m \to +\infty$. Per la
prima ipotesi, $\{u_m\}_m$ \`e una successione limitata in $V$
(altrimenti...). Il teorema di Eberlein-Smulian garantisce la
convergenza debole di tale successione a qualche $u \in V$. Per
ipotesi $M$ \`e debolmente chiuso, sicch\'e $u \in M$. Infine,
$$
I(u) \leq \liminf_{m \to +\infty} I(u_m) = \alpha_0.
$$

\pg;

* La semicontinuit\`a inferiore debole del precedente teorema \`e
  sovente garantita dalla {\em convessit\`a} del funzionale.

\pg+

{\bf Lemma.} Siano $X$ uno spazio di Banach, $K$ un sottoinsieme
convesso e chiuso di $X$, e $I$ un funzionale convesso s.c.i. su
$K$. Allora $I$ \`e debolmente s.c.i.

\medskip

{\em Dim.} Per ogni $\alpha \in \R$, l'insieme $K_\alpha$ \`e convesso
e chiuso. Per un noto risultato di Analisi Funzionale Lineare, tale
insieme \`e anche debolmente chiuso, quindi $I$ \`e debolmente s.c.i.

\pg+

{\bf Notazione.} Per rendere pi\`u espressiva la simbologia,
utilizzeremo anche la scrittura
$$
[I \leq \alpha] = \left\{ u \in X \mid I(u) \leq \alpha \right\}.
$$

\pg;

\sec Punti critici e topologia

Quando $X$ \`e uno spazio di Banach riflessivo e $I$ \`e un funzionale
convesso, la strategia per dimostrare che $I$ raggiunge il suo minimo
assoluto $\alpha_0$ consiste in due passi:
\begitems
* $\alpha_0 \in \R$ \`e tale che $[I < \alpha_0]=\emptyset$;
* per $\ge>0$ abbastanza piccolo, l'insieme $[I \leq \alpha_0+\ge]$
\`e non-vuoto e debolmente compatto.
\enditems

\pg+

In realt\`a ci\`o indica la presenza di punti critici di $I$ \`e la
differenza topologica dei sottolivelli $[I \leq c]$ e $[I \leq
c+\ge]$.

\pg;

\sec Esempi

* Sia $I(x)=x^3-3x$ per ogni $x \in \R$. La derivata di $I$ si annulla
  in $\pm 1$, e se poniamo $c_1=I(1)=-2$, $c_2=I(-1)=2$, vediamo che
  \begitems
  \style n
  * se $a_1<c_1$, l'insieme $[I \leq a_1]$ \`e un intervallo del tipo
  $(-\infty,\alpha_1]$;
  * se $c_1<a_2<c_2$, risulta $[I \leq a_2] = (-\infty,\alpha_2] \cup
  [\beta_2,\gamma_2]$ con $\alpha_2<\beta_2<\gamma_2$;
  * se $a_3>c_2$, risulta $[I \leq a_3] = (-\infty,\alpha_3]$.
  \enditems
  Nell'attraversare i valori $c_1$ e $c_2$, i sottolivelli del
  funzionale sono topologicamente distinti.

* Sia $I(x,y) = x^2-y^2$ per ogni $(x,y) \in \R^2$. Sappiamo che $0$
  \`e l'unico valore critico di $I$. Per ogni $\ge>0$, l'insieme $[I
  \leq \ge]$ \`e connesso, mentre $[I \leq -\ge]$ ha due componenti
  connesse.

\pg;

* Sia $I(x,y) = (x^2+y^2)^2-2(x^2+y^2)$ per ogni $(x,y) \in
  \R^2$. Esistono due valori critici $c_1=-1$ e $c_2=0$. Si vede
  facilmente (tutta la geometria del funzionale \`e radiale!) che se
  $a_1<c_1$, l'insieme $[I \leq a_1]$ \`e vuoto, che se $c_1<a_2<c_2$
  l'insieme $[I \leq a_2]$ \`e un anello del tipo $r^2 \leq x^2+y^2
  \leq R^2$, e infine che se $a_3>c_2$ l'insieme $[I \leq a_3]$ \`e
  una palla $B(0,R)$. Quindi il numero di componenti connesse non
  cambia nell'attraversare il livello $c_2=0$, e tuttavia anello e
  palla hanno invarianti topologici diversi.

\pg+

Questa idea di collegare la topologia dei sottolivelli all'esistenza
di punti critici si rivela vincente, e da essa si sviluppa la
cosiddetta {\em Teoria di Morse}.

\pg+ A causa del forte legame con la Topologia Algebrica, questa
teoria non rientra nei limiti del nostro corso.

\pg;

\sec Principi variazionali

* Come osservato, non \`e chiaro che una funzione limitata e
  semicontinua inferiormente debba raggiungere il suo minimo assoluto:
  si pensi alla funzione {\em analitica} $f(x)=\arctan x$ per ogni $x
  \in \R$.

\pg+

* Con il termine di {\it principi variazionali} ci si riferisce a
  teoremi che permettano di costruire {\it quasi minimi}, cio\`e
  tipicamente successioni minimizzanti per funzionali limitati e
  s.c.i., aventi per\`o ulteriori propriet\`a.

\pg;

\sec Il principio di Ekeland

{\bf Teorema.} Sia $M$ uno spazio metrico completo con metrica $d$, e
sia $I \colon M \to \R \cup \{+\infty\}$ un funzionale limitato dal
basso, s.c.i. e non identicamente infinito. Allora per ogni $\ge>0$ e
ogni $u \in M$ con
$$
I(u) \leq \inf_M I + \ge,
$$
esiste un elemento $v \in M$ tale che
$$
d(u,v) \leq 1, \quad I(v) < I(u),
$$
e, per ogni $w \neq v$ in $M$,
$$
I(w) > I(v)-\ge d(v,w).
$$

\pg;

{\em Dim.} Definiamo un ordinamento su $M$ ponendo
$w \leq v$ se e solo se $I(w)+\ge d(v,w) \leq I(v)$. Poniamo $u_0=u$,
e supponiamo di aver definito $u_n$. Sia
$$
S_n = \left\{ w \in M \mid w \leq u_n \right\},
$$
e scegliamo $u_{n+1} \in S_n$ tale che
$$
I(u_{n+1}) \leq \inf_{S_n} I + \frac{1}{n+1}.
$$
\pg+
\`E chiaro che $S_{n+1} \subset S_n$ poich\'e $u_{n+1} \leq u_n$;
inoltre $S_n$ \`e chiuso perch\'e $I$ \`e s.c.i.

Se $w\in S_{n+1}$, allora $w \leq u_{n+1} \leq u_n$ e dunque
$$
\ge d(w,u_{n+1}) \leq I(u_{n+1})-I(u_n) \leq \inf_{S_{n}} I +
\frac{1}{n+1} - \inf_{S_n} I = {1 \over n+1}.
$$
Deduciamo che
$$
\mathop{\rm diam} S_{n+1} \leq \frac{2}{\ge (n+1)}.
$$

\pg;

Poich\'e $M$ \`e uno spazio completo, \`e noto che
$$
\bigcap_n S_n = \{v\}
$$
per qualche $v \in M$. In particolare $v \in S_0$, cio\`e $v \leq
u_0=u$. Quindi
$$
I(v) \leq I(u)-\ge d(u,v) \leq I(u)
$$
e
$$
d(u,v) \leq \ge^{-1} \left( I(u)-I(v) \right) \leq \ge^{-1} \left(
\inf_M I + \ge - \inf_M I \right) =1.
$$
Per concludere, dimostriamo che $w \leq v$ implica $w=v$. Infatti, $w
\leq v$ implica $w \in u_n$ per ogni $n$, e dunque $w \in S_n$ per
ogni $n$. Quindi $w=v$.

\pg;

* Il senso del principio di Ekeland \`e che ad ogni punto in cui il
funzionale raggiunge ``quasi'' il minimo, \`e possibile associare un
punto ancora ``migliore'', che realizza anche il minimo assoluto {\em
proprio} di
$$
w \mapsto I(w)+\ge d(v,w).
$$

\pg+

* Questo principio diventa ancora pi\`u suggestivo se si arricchisce
  la struttura dello spazio $M$ e del funzionale $I$.

\pg+

{\bf Teorema.} Siano $X$ uno spazio di Banach, $\varphi \colon X \to
\R$ una funzione derivabile e limitata dal basso su $X$. Allora, per
ogni $\ge>0$ e per ogni $u \in X$ tale che $\varphi (u) \leq \inf_X
\varphi + \ge$, esiste $v \in X$ tale che $\varphi(v) \leq
\varphi(u)$,
$$
|v-u| \leq \sqrt{\ge}, \quad |\varphi'(v)| \leq \sqrt{\ge}.
$$

\pg;

{\em Dim.} Scegliamo $M=X$, $I=\varphi$ e, per $\ge>0$ dato, scegliamo
$d(x,y)=\ge^{-1/2} |x-y|$ nel Teorema di Ekeland. Otteniamo un
elemento $v \in X$ tale che $\varphi(w)>\varphi(v)-\sqrt{\ge}|w-v|$
per ogni $w \neq v$.

Scriviamo $w=v+th$ con $t>0$, $h \in X$, $|h|=1$, per ottenere
$$
\varphi(v+th)-\varphi(v) > -\sqrt{\ge} t.
$$
\pg+
Dividendo per $t$ e prendendo il limite per $t \to 0$, deduciamo
$$
-\sqrt{\ge} \leq \varphi'(v)[h].
$$
Per l'arbitrariet\`a di $h$ sulla sfera unitaria di $X$, deve essere
$|\varphi'(v)| \leq \sqrt{\ge}$.

Le altre propriet\`a di $v$ sono ovvia conseguenza del Teorema di
Ekeland.

\pg+

* \`E ormai spontaneo ``discretizzare'' il parametro $\ge>0$, per
  costruire successioni minimizzanti con derivata ``quasi'' nulla.

\pg;

{\bf Corollario.} Siano $X$ uno spazio di Banach, $\varphi \colon X
\to \R$ un funzionale limitato dal basso e derivabile in $X$. Allora,
per ogni successione minimizzante $\{u_k\}_k$ di $\varphi$, esiste una
successione minimizzante $\{v_k\}_k$ di $\varphi$ tale che
$\varphi(v_k) \leq \varphi(u_k)$,
$$
\lim_{k \to +\infty} |u_k-v_k| =0, \quad
\lim_{k \to +\infty} \|\varphi'(v_k)\| =0.
$$

\medskip

{\em Dim.} Basta porre
$$
\ge_k = \cases{\varphi(u_k)-\inf_X \varphi &se $\varphi(u_k)-\inf_X
\varphi>0$ \cr
1/k &se $\varphi(u_k)-\inf_X \varphi=0$. \cr
}
$$

\pg;

\sec Passi di montagna: il caso finito-dimensionale

* L'idea di costruire punti ``quasi critici'' per un funzionale si
  rivela essere un approccio molto utile nello sviluppo di teoremi di
  ``vero'' punto critico. Cominciamo con un caso in dimensione finita.

\pg+

{\bf Teorema.} Supponiamo che $I \in C^1(\R^N)$ sia un funzionale
coercivo, e che $I$ possieda due punti distinti di minimo stretto,
$x_1$ e $x_2$. Allora $I$ possiede un terzo punto critico $x_3$ che
non \`e un minimo locale, e dunque distinto da $x_1$ e $x_2$.

\pg+

* Immaginiamo dunque (almeno nel caso $N=2$) il grafico di $I$, con
  due punti di minimo locale stretto. Il teorema afferma, sotto
  l'ipotesi che $I$ diverga all'infinito per argomenti divergenti
  all'infinito, che da qualche parte esiste un punto di {\em sella}.

\pg;

{\em Dim.} Definiamo il valore
$$
\beta = \inf_{p \in P} \max_{x \in p} I(x),
$$
dove
$$
P = \left\{ p \subset \R^N \mid x_1 \in p,\ x_2 \in p, \ \hbox{$p$ \`e
compatto e connesso} \right\}.
$$
Prendiamo una successione $\{p_m\}_m \subset P$ minimizzante per $\beta$, nel
senso che
$$
\lim_{m \to +\infty} \max_{x\in p_m} I(x) = \beta.
$$
Poich\'e $I$ \`e coercivo, gli insiemi $p_m$ sono uniformemente
limitati.
\pg+
L'insieme dei punti di accumulazione di $\{p_m\}_m$,
$$
p = \bigcap_{m \in {\bbchar N}} \overline{\bigcup_{l \geq m} p_l},
$$
\`e l'intersezione di una successione decrescente di insiemi compatti
e connessi: dunque anch'esso \`e compatto e connesso. Inoltre
$\{x_1,x_2\} \subset p$, poich\'e $\{x_1,x_2\} \subset p_m$ per ogni
$m$.

\pg;

Deduciamo che
$$
\max_{x \in p} I(x) \geq \inf_{p'\in P} \max_{x \in p'} I(x) = \beta.
$$
D'altra parte, per continuit\`a,
$$
\max_{x \in p} I(x) \leq \limsup_{m \to+\infty} \max_{x \in p_m} I(x)
= \beta,
$$
sicch\'e $\max_{x \in p} I(x) = \beta$.

\pg+

Poich\'e $x_1$ e $x_2$ sono due punti di minimo locale stretto
collegati da $p$, risulta $\beta > \max \{ I(x_1),I(x_2)\}$.

\pg+

* Dimostriamo che esiste un punto critico $x_3 \in p$ tale che
  $I(x_3)=\beta$.

\pg+

Procediamo per assurdo. Innanzitutto l'insieme
$$
K = \left\{ x \in p \mid I(x)=\beta \right\}
$$
\`e chiuso (perch\'e?) e limitato, dunque compatto. Supponiamo che $I'
\neq 0$ in $K$.

\pg;

L'ipotesi assurda garantisce l'esistenza di un numero $\delta>0$ tale
che $|I'(x) | \geq 2\delta>0$ per ogni $x \in K$. Per continuit\`a,
esiste un intorno
$$
U_\ge = \left\{ x \in \R^N \mid |x-y| < \ge \ \hbox{per qualche $y \in
K$} \right\}
$$
di $K$ tale che $|I'| \geq \delta$ su $U_\ge$. In particolare, $x_i
\notin U_\ge$, $i=1,2$.

\pg+

Sia $\eta$ una funzione continua con supporto contenuto in $U_\ge$,
tale che $0 \leq \eta \leq 1$ e $\eta \equiv 1$ in un intorno di
$K$.

\pg+

Definiamo $\Phi \colon \R^N \times \R \to \R^N$ mediante
$$
\Phi(x,t) = x - t \eta(x) \nabla I(x).
$$
Un calcolo diretto mostra che
$$
\left. {d \over dt} \right|_{t=0} I(\Phi(x,t)) = -\eta(x) |\nabla
I(x)|^2.
$$

\pg;

Inoltre $|\nabla I(x)|^2 \geq \delta^2 >0$ su $\mathop{\rm supp} \eta
\subset U_\ge$. Per continuit\`a, esiste $T>0$ tale che
$$
{d \over dt} I(\Phi(x,t)) \leq -{\eta(x) \over 2} |\nabla I(x)|^2
$$
per ogni $0 \leq t \leq T$, uniformemente rispetto a $x$.

\pg+

Sia $p_T = \{\Phi(x,T) \mid x \in p\}$; per ogni $\Phi(x,T) \in p_T$
calcoliamo
$$
\eqalign{%
I(\Phi(x,T)) &= I(x) + \int_0^T {d \over dt} I(\Phi(x,t))\, dt \cr
&\leq I(x) - {T \over 2} \eta(x) |\nabla I(x)|^2, \cr
}
$$
e l'ultimo termine \`e $\leq I(x)=\beta$ se $x \notin K$, oppure $\leq
\beta - {T \over 2} \delta^2 < \beta$ se $x \in K$. In ogni caso,
$$
\max_{x \in p_T} I(x) < \beta.
$$
Ma:
\begitems
* $p_T$ \`e compatto e connesso;
* $x_i = \Phi(x_i,T) \in p_T$, $i=1,2$.
\enditems
Pertanto $p_T \in P$, e questo contraddice la definizione di $\beta$.

\pg;

Se tutti i punti critici $u$ di $I$ in $p$ con $I(u)=\beta$ fossero
minimi locali, l'insieme $\tilde{K}$ di tali punti sarebbe aperto in
$p$  e --- per la continuit\`a di $I$ e di $I'$ --- anche chiuso. Ma $
\tilde{K} \neq \emptyset$ per la discussione precedente, dunque
$p=\tilde{K}$. Ci\`o contraddice il fatto che $I(x_1)<\beta$,
$I(x_2)<\beta$.

\pg+

Pertanto almeno un punto critico di $I$ in $p$ al livello $\beta$ non
\`e un minimo locale. La dimostrazione \`e completa.

\pg+

* L'interpretazione di questo teorema \`e suggestiva: $I(x)$ misura
  l'altitudine di un punto $x$ in un panorama. I due minimi $x_1$ e
  $x_2$ corrispondono a due villaggi collocati al fondo di due valli
  separate da una cresta montagnosa. Se camminiamo lungo un sentiero
  $p$ che unisce i due villaggi, scelto in modo che la quota massima
  $I(x)$ raggiunta nei punti $x \in p$ sia minimale tra le quote di
  tutti i possibili sentieri, il teorema afferma che attraverseremo la
  cresta in un punto di sella (che i montanari chiamano anche {\it
  forcella}).

\pg;

* Il teorema precedente \`e la versione finito-dimensionale del {\em
  teorema del passo di montagna}, dimostrato nel 1972 da A. Ambrosetti
  e P. H. Rabinowitz.

\pg+

* La versione ambientata in uno spazio normato di dimensione
  (eventualmente) infinita \`e generalmente falsa senza ulteriori
  ipotesi sul funzionale $I$.

\pg+

* Nalla dimostrazione appena vista, la coercivit\`a di $I$ garantiva
  immediatamente la limitatezza --- e dunque la {\em relativa
  compattezza} --- di tutti i (sotto)livelli di $I$. La compattezza
  (relativa) della palla unitaria di uno spazio di Banach equivale
  per\`o all'avere dimensione finita.

\pg+

* Occorre, nel caso infinito-dimensionale, un surrogato della
  coercivit\`a, che garantisca la necessaria compattezza.

\pg;

\sec Un lemma di deformazione

Approfondiamo il legame tra l'esistenza di punti critici e la topologia dei sottolivelli.

\smallskip

{\bf Lemma di deformazione.} Siano $J \in C^{1,1}_{\rm loc}(\R^N,\R)$, $c \in \R$ e $\ge_0>0$ tali che
\begitems
\style n
* l'insieme $[c-\ge_0 \leq J \leq c+\ge_0]$ \`e compatto;
* esiste $\lambda>0$ tale che, per ogni $x \in [c-\ge_0 \leq J \leq c+\ge_0]$, si abbia $\|J'(x)\| \geq \lambda$.
\enditems
Allora, per ogni $\ge<\ge_0/2$, esiste un omeomorfismo $\Phi$ di $\R^N$ in $\R^N$, tale che $\Phi([J \leq c+\ge]) \subset [J \leq c-\ge]$.

\medskip

* L'ipotesi di assenza di punti critici {\em nella striscia}
 $[c-\ge_0 \leq J \leq c+\ge_0]$ assicura che il sottolivello $[J \leq c+\ge]$ abbia la stessa topologia del sottolivello $[J \leq c - \ge]$.

 \pg;

 {\em Dim.} Sia $0< \ge < \ge_0/2$. Poniamo $\alpha = 2\ge / \lambda^2$ e
 $$
 A = [J \leq c-\ge_0] \cup [J \geq c+\ge_0], \quad B = [c-\ge \leq J \leq c+\ge].
 $$
 Sia $f \colon \R^N \to [0,1]$ una funzione localmente lipschitziana tale che $f(A)=\{0\}$ e $f(B)=\{1\}$. Ad esempio possiamo scegliere
 $$
 f(x) = {d(x,A) \over d(x,A)+d(x,B)}.
 $$
 \pg+
 Allora l'equazione differenziale
 $$
 \cases{
 {dx \over dt} = -\alpha f(x(t))J'(x(t)) \cr
 x(0)=x_0
 }
 $$
 possiede un'unica soluzione $x$ che esiste per ogni tempo, dal momento che il secondo membro \`e un campo vettoriale localmente lipschitziano e uniformemente limitato. Inoltre la soluzione dipende con continuit\`a da $x_0$. Denotiamo questa soluzione con $\eta(t,x_0)$.

 \pg;

 Il funzionale $J$ decresce lungo la soluzione:
 $$
 \eqalign{%
  {d \over dt} J(\eta(t,x_0)) &= J'(\eta(t,x_0)) {d\eta \over dt} \cr
  &= -\alpha f(\eta(t,x_0)) \| J'(\eta(t,x_0)) \|^2 \leq 0.
  }
$$
* Definiamo $\Phi(x_0) = \eta(1,x_0)$ per ogni $x_0 \in \R^N$.

Sia $x_0 \in [J \leq c+\ge]$: Se $J(x_0) \leq c-\ge$, per la monotonia di $t \mapsto J(\eta(t,x_0))$ si ha $\Phi(x_0) \in [J \leq c-\ge]$. Se invece per ogni tempo $t$ risulta che $\eta(t,x_0)$ appartiene a $[c-\ge < J \leq c+\ge]$, allora
$$
\eqalign{
J(\eta(1,x_0)) &= J(x_0) + \int_0^1 {d \over dt} J(\eta(t,x_0))\, dt \cr
&= J(x_0) - \alpha \int_0^1 f(\eta(t,x_0)) \| J'(\eta(t,x_0)) \|^2 \, dt \cr
&\leq J(x_0) - \alpha \lambda^2 \leq c-\ge,
}
$$
il che significa che la traiettoria di $\eta$ passante per $x_0$ esce, al pi\`u tardi al tempo $t=1$, dall'insieme $[c-\ge<J \leq c+\ge]$ per entrare nell'insieme $[J \leq c-\ge]$.

\pg;

{\bf Osservazione.} La funzione $\eta(\cdot,x_0)$ \`e chiamata {\em flusso} associato a $J$ passante per $x_0$. Nei fatti abbiamo dimostrato ben pi\`u dell'enunciato:
\begitems
\style n
* per ogni $t \in [0,1]$, $x \mapsto \eta(t,x)$ \`e un omoemorfismo;
* per ogni $x \in \R^N$, $\eta(0,x)=x$
* se $x \notin [c-\ge_0 \leq J \leq c+\ge_0]$, abbiamo che $\eta(t,x)=x$ per ogni $t \in [0,1]$;
* se $x \in [c-\ge \leq J \leq c+\ge]$, abbiamo che $\eta(1,x) \in [J \leq c-\ge]$.
\enditems
\pg+
* Se $[c-\ge_0 \leq J \leq c+\ge_0]$ \`e limitato e non contiene punti critici di $J$, allora $\|J'\|$ \`e minorato su tale insieme da una costante positiva, e il lemma si applica. \`E questa una conseguenza del teorema di Weierstrass e della compattezza locale di $\R^N$. Ma che succede in dimensione infinita, dove la compattezza locale \`e (sempre) falsa?

\pg;

\sec La condizione di Palais-Smale

{\bf Definizione.} Sia $X$ uno spazio di Banach, e sia $J \in C^1(X)$ un funzionale su $X$. Se $c \in \R$, diciamo che $J$ soddisfa la condizione di Palais-Smale al livello $c$, se ogni successione $\{u_n\}_n$ in $X$ tale che
$$
J(u_n) \to c, \quad J'(u_n) \to 0
$$
contiene una sottosuccessione convergente in $X$.

\pg+

* Intuitivamente, stiamo pretendendo che le successioni che ``puntano'' verso valori critici convergano (almeno a meno di sottosuccessioni) effettivamente ad un punto critico.

\pg+

* La condizione non \`e banale: si pensi alla funzione reale di una variabile reale $f(x)=e^{-x}(2+\sin (e^{-2x}))$ con $c=0$.

\pg;

{\bf Corollario.} Se $J$ verifica la coondizione (PS) a livello $c \in \R$, allora l'insieme
$$
K(c) = \left\{ u \in X \mid J(u)=c, \quad J'(u)=0 \right\}
$$
\`e compatto.

\pg+

{\em Dim.} Supponiamo che $u_n \in K(c)$ per ogni $n$. Allora, in particolare, $\{u_n\}_n$ \`e una successione di Palais-Smale a livello $c$. Esiste pertanto una sottosuccessione convergente in $X$ ad un limite $u$, che risulta appartenere a $K(c)$ per continuit\`a di $J$ e di $J'$.

\pg;

\sec Un esempio

Sia $(A,D(A))$ l'operatore autoaggiunto a risolvente compatto definito in $L^2(\Omega)$, dove $\Omega$ \`e un aperto limitato, dalla formula $Au = -\Delta u$ per ogni $u \in D(A)$, essendo
$$
D(A) = \left\{ u \in H_0^1(\Omega) \mid \Delta u \in L^2 (\Omega) \right\}.
$$
Sia $\sigma_p(A)=\{\lambda_k\}_k$ l'insieme degli autovalori di $A$.

\pg+

Identificando $L^2(\Omega)$ con il suo duale, risulta $H_0^1(\Omega) \subset L^2(\Omega) \subset H^{-1}(\Omega)$, con immersioni continue e dense.

\pg+

Per $\lambda \in \R$ e $f \in H^{-1}(\Omega)$ fissati, sia $J$ il funzionale definito su $H_0^1(\Omega)$ come
$$
J(u) = \frac{1}{2} \int_\Omega \left[ |\nabla u|^2 + \lambda |u|^2 \right] \, dx - \langle f,u \rangle,
$$
dove $\langle \phantom{\cdot},\phantom{\cdot} \rangle$ denota la dualit\`a tra $H_0^1(\Omega)$ e $H^{-1}(\Omega)$.

\pg;

Dimostriamo che, se $\lambda \notin \sigma_p(A)$, allora $J$ soddisfa la condizione (PS) a qualunque livello.

\pg+

Infatti, denotando con $\tilde{A}$ l'estensione di $A$ a
$H_0^1(\Omega)$, risulta
$$
J'(u) = \tilde{A}u - \lambda u - f
$$
e $\tilde{A}-\lambda I$ \`e un omeomorfismo di $H_0^1(\Omega)$ su
$H^{-1}(\Omega)$.

\pg+

Se $\{u_n\}_n$ \`e una successione di $H_0^1(\Omega)$ tale che
$J(u_n)\to c$ e
$$
J'(u_n) = \tilde{A}u_n-\lambda u_n -f =\ge_n \to 0 \quad \hbox{in $H^{-1}(\Omega)$},
$$
allora $u_n = (\tilde{A}-\lambda I)^{-1} [f+\ge_n] \to u = (\tilde{A}-\lambda I)^{-1}f$  in $H_0^1(\Omega)$ per la compattezza del risolvente di $A$.

\pg+

Se invece $\lambda = \lambda_k \in \sigma_p(A)$, la condizione (PS)
non pu\`o essere soddisfatta. Basta considerare la successione $\{n
\varphi_k\}_n$, dove $\varphi_k$ \`e un'autofunzione associata
all'autovalore $\lambda_k$. Risulta $J(n \varphi_k)=0$ e $J'(n
\varphi_k)=0$ per ogni $n$, sebbene non esistano sottosuccessioni
convergenti.

\pg;

\sec Un esempio semilineare

Siano $\Omega$ un aperto limitato di $\R^N$, $p>1$ tale che
$(N-2)p<N+2$. Per $\lambda$ reale fissato, consideriamo il funzionale
$J$ definito su $H_0^1(\Omega)$ da
$$
J(u)={1 \over 2} \int_\Omega |\nabla u|^2 \, dx + {\lambda \over p+1}
\int_\Omega |u|^{p+1}\, dx.
$$
\pg+ Sia $\{u_n\}_n$ una successione di Palais-Smale al livello
$c$. Quindi
$$
J'(u_n) = -\Delta u_n + \lambda |u_n|^{p-1}u_n \to 0 \quad \hbox{in
$H^{-1}(\Omega)$}.
$$
\pg+
Calcoliamo
$$
\eqalign{ J'(u_n)[u_n] &= \int_\Omega |\nabla u_n|^2 + \lambda
\int_\Omega |u_n|^{p+1} \cr &= (p+1) J(u_n) - {p-1 \over 2}
\int_\Omega |\nabla u_n|^2 .  }
$$
Ora, $|J'(u_n)[u_n]| \leq \|J'(u_n)\| \| \nabla u_n\|$, e possiamo dedurre che
$$
{p-1 \over 2} \|\nabla u_n\|^2 \leq (p+1)J(u_n) + \|J'(u_n)\| \| \nabla u_n\|.
$$

\pg;

Poich\'e $p>1$, questa disuguaglianza dimostra che la successione
$\{u_n\}_n$ \`e limitata. Possiamo applicare il teorema di compattezza
di Rellich, e dedurre che esiste una sottosuccessione $v_i = u_{n_i}$
tale che $v_i \to v$ in $H_0^1(\Omega)$ debole e in $L^{p+1}(\Omega)$
forte.

\pg+

Ma allora $|v_i|^{p-1}v_i \to |v|^{p-1}v$ in $L^{(p+1)/p}(\Omega)$,
dunque anche in $H^{-1}(\Omega)$, e infine
$$
-\Delta v_i = J'(v_i)-\lambda |v_i|^{p-1}v_i \to -\lambda |v|^{p-1}v
$$
in $H^{-1}(\Omega)$.

\pg+

Chiamando $B$ l'operatore che a $f \in H^{-1}(\Omega)$ fa
corrispondere la soluzione $z$ di:
$$
z \in H_0^1(\Omega), \quad -\Delta z = h,
$$
sappiamo che $B$ \`e un operatore continuo da $H^{-1}(\Omega)$ in
$H_0^1(\Omega)$. Otteniamo
$$
v_i = B(J'(v_i)-\lambda |v_i|^{p-1}v_i) \to B(-\lambda |v|^{p-1}v)
$$
in $H_0^1(\Omega)$.

\pg;

Questo mostra non solo che $\{u_n\}_n$ possiede una sottosuccessione
covergente, ma anche che il limite $v$ di tale sottosuccessione \`e
una soluzione di
$$
-\Delta v + \lambda |v|^{p-1}v =0 \quad \hbox{in $H^{-1}(\Omega)$}.
$$

\pg;

\sec La condizione (PS) e il principio di Ekeland

Vediamo un interessante risultato che unisce la condizione di Palais-Smale al principio variazionale di Ekeland.

\medskip

{\bf Proposizione.} Siano $X$ uno spazio di Banach e $J \in C^1(X)$ un
funzionale su $X$. Supponiamo che $J$ sia limitato dal basso e
verifichi la condizione (PS) al livello $c = \inf_X J$. Allora $J$
raggiunge il suo minimo $c$.

\pg+

{\em Dim.} Per il principio variazionale, esiste una successione $u_n
\in X$ tale che
$$
\eqalign{%
&c \leq J(u_n) \leq c+{1 \over n} \cr
&\forall v \in X: \quad J(v)+{1 \over n} \|v-u_n\| \geq J(u_n).
}
$$
Scrivendo
$$
J(v)=J(u_n)+J'(u_n)[v-u_n]+o(\|v-u_n\|),
$$
deduciamo che
$$
\|J'(u_n)\| \leq {1 \over n}.
$$
Poich\'e $J$ soddisfa la condizione (PS) al livello $c$, esiste una
sottosuccessione convergente a qualche $u \in X$. Per continuit\`a,
$J(u)=c$ e $J'(u)=0$.

\pg;

\sec Campi vettoriali pseudo-gradienti

{\bf Definizione.} Siano $X$ uno spazio di Banach e $J \in
C^1(X,\R)$. Se $u \in X$, diciamo che $v \in X$ \`e un vettore
pseudogradiente di $J$ in $u$ se:
$$
\|v\| \leq 2 \|J'(u)\|, \quad J'(u)[v] \geq \|J'(u)\|^2.
$$
Denotato con $X_r = \{u \in X \mid J'(u) \neq 0\}$ l'insieme dei punti
regolari di $J$, un'applicazione $V \colon X_r \to X$ \`e chiamata
campo vettoriale pseudogradiente per $J$ se $V$ \`e localmente
lipschitziana su $X_r$ e per ogni $u \in X_r$, $V(u)$ \`e un vettore
pseudogradiente per $J$ in $u$.

\pg+

* Se $X$ \`e uno spazio di Hilbert e se $\nabla J$ \`e localmente
  lipschitziano, allora $\nabla J$ \`e un c.v.p.g.

* Se $J$ non ha derivata localmente lipschitziana, l'esistenza di un
  c.v.p.g. non \`e affatto evidente.

\pg;

{\bf Lemma.} Siano $X$ uno spazio di Banach, e $J \in C^1(X)$ un
funzionale non costante. Esiste allora un c.v.p.g. per $J$.

\smallskip

{\em Dim.} Sia $u \in X_r$ un punto regolare. Per definizione,
$$
\|J'(u) \| = \sup_{\|x\|=1} J'(u)[x],
$$
e pertanto esiste $x_u \in X$ tale che $\|x_u\|=1$ e
$$
J'(u)[x_u] > {2 \over 3} \|J'(u)\|.
$$
Ponendo $v=v_u={3 \over 2} \|J'(u)\|x_u$, abbiamo:
$$
\eqalign{%
\|v_u\| &= {3 \over 2} \|J'(u)\| < 2 \|J'(u)\| \cr
J'(u)[v_u] &= {3 \over 2} J'(u)[x_u] > \|J'(u)\|^2. \cr
}
$$

\pg;

Abbiamo costruito un vettore pseudogradiente per $J$ in $u$. Adesso
facciamo una tipica operazione di Topologia Differenziale: incolliamo,
mediante una partizione dell'unit\`a, questi vettori.

\pg+

* Per continuit\`a di $J'$, esiste un intorno $\Omega_u$ di $u$,
  contenuto in $X_r$, tale per ogni $x \in \Omega_u$, $v_u$ sia un
  vettore pseudogradiente per $J$ in $x$.

* La famiglia $\{\Omega_u \mid u \in X_r\}$ \`e un ricoprimento aperto
  di $X_r$. Poich\'e tutti gli spazi metrici sono {\em paracompatti},
  esistono un sottoricoprimento localmente finito $\{\omega_j\}_{j \in
  I}$ e una partizione dell'unit\`a localmente lipschitziana
  $\{\theta_j\}_{j \in I}$, subordinata a $\{\omega_j\}_{j \in I}$.

\pg;

Ponendo
$$
V(z) = \sum_{j \in J} \theta_j(z) v_{u_j},
$$
basta osservare che la somma \`e finita: dunque $V$ \`e un campo
vettoriale localmente lipschitziano su $X_r$ a valori in $X$, che
risulta pseudo-gradiente per la scelta di $v_{u_j}$.

\pg+

* I risultati topologici utilizzati nel corso della dimostrazione
  possono essere recuperati in molti manuali di topologia generale e/o
  differenziale. Ad esempio quelli di Bourbaki o Dieudonn\'e.

\pg+

* Siamo pronti ad enunciare e dimostrare un lemma di deformazione nel
  contesto degli spazi di Banach di dimensione qualunque.

\pg;

\sec Un lemma di deformazione

{\bf Lemma.} Siano $X$ uno spazio di Banach, $J \in C^1(X,\R)$ un
funzionale non costante che soddisfa la condizione (PS) (ad ogni
livello), e sia $c \in \R$ un valore regolare di $J$. Esiste allora
$\ge_0>0$ tale che per $0<\ge<\ge_0$ esiste un'applicazione $\eta \in
C(\R \times X,X)$, chiamata flusso associato a $J$, che verifica le
seguenti condizioni:
\begitems
\style n
* per ogni $u \in X$, si ha $\eta(0,u)=u$;
* per ogni $t \in R$ e $u \notin [c-\ge_0\leq J \leq c+\ge_0]$, si ha
$\eta(t,u) = u$;
* per ogni $t \in \R$, $\eta(t,\cdot)$ \`e un omeomorfismo di $X$ in
$X$;
* per ogni $u \in X$, la funzione $t \mapsto J(\eta(t,x))$ \`e
monotona decrescente;
* Se $u \in [J \leq c+\ge]$, allora $\eta(1,u) \in [J \leq c-\ge]$;
\enditems

\pg;

{\em Dim.} Poich\'e $J$ soddisfa la condizione (PS) al valore $c$, e
$c$ non \`e un valore critico, si verifica immediatamente per
contraddizione che esistono $\ge_1>0$ e $\delta \in (0,1]$ tali che
$$
\forall u \in [c-\ge_1 \leq J \leq c+\ge_1]: \quad \|J'(u)\| \geq
\delta.
$$
\pg+
Poniamo $\ge_0 = \min\{\ge_1,\delta^2/8\}$, e per $0<\ge<\ge_0$,
$$
A=[J \leq c-\ge_0] \cup [J \geq c+\ge_0], \quad B = [c-\ge \leq J \leq
c+\ge].
$$
Poich\'e $A \cap B = \emptyset$, la funzione
$$
\alpha(x) = {d(x,A) \over d(x,A)+d(x,B)}
$$
\`e localmente lipschitziana e verifica $\alpha=0$ su $A$, $\alpha=1$
su $B$.
\pg+
Infine, sia $V$ un campo vettoriale pseudogradiente per $J$ su
$X_r$. Sia, per ogni $x \in X$,
$$
W(x) = \alpha(x) \min \left\{ 1,{1 \over \|V(x)\|} \right\} V(x).
$$

\pg;

Si verifica direttamente che $\|W(\cdot)\| \leq 1$. Per la teoria
generale delle equazioni differenziali ordinarie, il problema
$$
\cases{%
{d \over dt} \eta(t,x) = -W(\eta(t,x)) \cr
\null \cr
\eta(0,x)=x \cr
}
$$
possiede una ed una sola soluzione $\eta(\cdot,x) \in C^1(\R,X)$, e
$\eta$ \`e localmente lipschitziana su $\R \times X$.

\pg+

Per la propriet\`a di flusso $\eta(t,\eta(s,x))=\eta(t+s,x)$, per ogni
$t \in \R$ la funzione $\eta(t,\cdot)$ \`e un omeomorfismo di $X$ in
$X$, la cui inversa \`e $\eta(-t,\cdot)$.

\pg+

Le condizioni 1. e 3. sono evidenti per costruzione. Se
$$
u \notin [c-\ge_0\leq J \leq c+\ge_0],
$$
allora $W(u)=0$, e l'unicit\`a della soluzione implica che
$\eta(t,u)=u$ per ogni tempo $t$.

\pg;

Anche la propriet\`a 4. si verifica direttamente, calcolando
$$
\eqalign{%
{d \over dt} J(\eta(t,u)) &= J'(\eta(t,u))\left[ {d \eta \over dt}
\right] \cr
&= -\alpha(\eta(t,u)) \min \left\{ 1, {1 \over \|V(\eta(t,u))\|}
\right\} J'(\eta(t,u)) [V(\eta(t,u))] \cr
&\leq -\alpha(\eta(t,u)) \min \left\{ 1, {1 \over \|V(\eta(t,u))\|}
\right\} \|J'(\eta(t,u))\|^2, \cr
}
$$
ci\`o che dimostra la monotonia desiderata.

\pg+

Per verificare 5. consideriamo $u \in [J \leq c+\ge]$ e notiamo che se
per qualche $t_0 \in [0,1)$ si ha
$$
\eta(t_0,u) \in [J \leq c-\ge],
$$
allora $\eta(1,u)$ resta in $[J \leq c-\ge]$ per la monotonia appena
dimostrata. Supponiamo dunque che, per ogni $t \in [0,1)$, si abbia
$\eta(t,u) \in [c-\ge < J \leq c+\ge]$.

\pg;

Ricordando che $\|J'(x)\| \leq \|V(x) \leq 2 \|J'(x)\|$, risulta
$$
\eqalign{%
{d \over dt} J(\eta(t,u)) &\leq -{1 \over 4} \min \left\{ 1,{1 \over
\|V(\eta(t,u))\|} \right\} \|V(\eta(t,u))\|^2 \cr
&\leq \cases{-{1 \over 4} &se $\|V(\eta(t,u))\| \geq 1$, \cr
-{\delta^2 \over 4} &se $\|V(\eta(t,u))\| < 1$. \cr
}
}
$$
\pg+

Poich\'e $\delta \leq 1$, concludiamo facilmente che
$$
J(\eta(1,u)) \leq -{\delta^2 \over 4} + J(u) \leq -{\delta^2 \over 4}
+ c+\ge.
$$
Ricordando la scelta di $\ge_0$, otteniamo $J(\eta(1,u))\leq c-\ge$.

\pg;

\sec Un teorema di M. Morse

{\bf Teorema.} Siano $X$ uno spazio di Banach e $J \in C^1(X)$ un funzionale non costante che soddisfa la condizione (PS). Se un valore $c\in\R$ non \`e critico, allora esiste $\ge_0>0$ tale che per ogni $0<\ge<\ge_0$, il livello $[J \leq c-\ge]$ \`e un retratto di deformazione di $[J \leq c+\ge]$.

\medskip

{\em Dim.} Se $\ge_0>0$ \`e il numero del Lemma di Deformazione, con le stesse notazioni \`e sufficiente definire
$$
\varphi(t,u) = \eta \left( {4t \over \delta^2} (J(u)-c+\ge)^{+},u \right)
,
$$
avendo indicato con $z^{+}$ la parte positiva del numero reale $z$.
\pg+

Risulta evidentemente $\varphi(0,\cdot)=I$ e $\varphi(t,u)=u$ se $u \in [J \leq c-\ge]$, sicch\'e $[J \leq c-\ge] \subset \varphi(1,[J \leq c+\ge])$. Per mostrare l'inclusione inversa, siano $u \in [c-\ge<J\leq c+\ge]$ e $\tau>0$ tali che per $0 \leq t \leq \tau$ si abbia $\eta(t,u) \in [c-\ge<J \leq c+\ge]$.

\pg;

Come nella dimostrazione del Lemma di Deformazione, risulta
$$
c-\ge < J(\eta(\tau,u)) \leq J(u) - {\delta^2 \over 4} \tau,
$$
cio\`e $\tau<4(J(u)-c+\ge))/\delta^2$. Di conseguenza per
$$
t_0 = \frac{4(J(u)-c+\ge)}{\delta^2},
$$
risulta $\varphi(1,u) = \eta(t_0,u) \in [J \leq c-\ge]$, e la dimostrazione \`e completa.

\pg;

\sec Il principio di minimax

Il lemma di deformazione ci permette di enunciare un principio
variazionale molto potente, del quale vedremo un caso speciale di
utilizzo diffuso nello studio delle equazioni differenziali.

\pg+

* Sebbene faremo discendere questo principio variazionale dal Lemma di
  Deformazione, \`e possibile utilizzare esclusivamente il principio
  variazionale di Ekeland.

\pg+

{\bf Teorema di minimax.} Siano $X$ uno spazio di Banach, $J \in
C^1(X)$ un funzionale che soddisfa la condizione (PS), e ${\cal B}$
una famiglia non vuota di sottoinsiemi non vuoti di $X$. Supponiamo
che per ogni $c \in R$ e $\ge>0$ sufficientemente piccolo, il flusso
$\eta(1,\cdot)$ costruito nel Lemma di Deformazione preservi ${\cal
B}$, nel senso che $\eta(1,B) \in {\cal B}$ se $B \in {\cal
B}$. Poniamo
$$
\tilde{c} = \inf_{A \in {\cal B}} \sup_{v \in A} J(v).
$$
Se $\tilde{c} \in \R$, allora $\tilde{c}$ \`e un valore critico di $J$.

\pg;

{\em Dim.} La dimostrazione \`e molto semplice, a questo punto. Sia
infatti $\tilde{c} \in \R$. Se non fosse un valore critico, scegliendo
$\ge>0$ sufficientemente piccolo potremmo selezionare un elemento $A
\in {\cal B}$ tale che $\tilde{c} \leq \sup_{v \in A} J(v) \leq
\tilde{c}+\ge$. Ma per ipotesi l'insieme $B = \eta(1,A)$ da una parte
appartiene alla famiglia ${\cal B}$; dall'altra soddisfa $B \subset [J
\leq \tilde{c}-\ge]$, contraddicendo la definizione di $\tilde{c}$.

\pg+

* Ovviamente la parola ``minimax'' non deve essere presa alla lettera,
  ma ``infisup'' non sembra un'alternativa foneticamente gradevole.

\pg+

* L'enunciato pu\`o sembrare poco elegante, poich\'e le ipotesi fanno
  uso esplicito del flusso costruito nel Lemma di
  Deformazione. Vedremo una variante completamente ``intrinseca'', le
  cui ipotesi non fanno riferimento ad enti costruiti all'interno di
  dimostrazioni precedenti.

\pg;

{\bf Esempio.} Abbiamo gi\`a detto quanto questi risultati siano
connessi al principio variazionale di Ekeland. Per convincercene,
possiamo scegliere ${\cal B} = \{\{x\}\mid x \in X\}$. Allora
$$\inf_{A\in{\cal B}} \sup_{v \in A} J(v) = \inf_{v \in X} J(v),$$
e ritroviamo il corollario del principio di Ekeland gi\`a dimostrato.

\pg+

Similmente, scegliendo ${\cal B} = \{X\}$, risulta
$$
\inf_{A \in {\cal B}} \sup_{v \in A} J(v) = \sup_{v \in X} J(v),
$$
che \`e lo stesso corollario applicato a $-J$.

\pg;

\sec Il teorema del Passo di Montagna

* Il Teorema del Passo di Montagna fu enunciato e dimostrato da
  A. Ambrosetti e P.H. Rabinowitz in {\em Dual variational methods in
  critical point theory and applications}, J. Funct. Anal. {\bf 14}
  (1973).

* L'articolo originale si basa su una versione del Lemma di
  Deformazione gi\`a nota in letteratura, e dovuta a Clarke.

* Successivamente sono state proposte molte generalizzazioni del Passo
  di Montagna, soprattutto mediante strumenti topologici. Sebbene il
  Passo di Montagna sia un caso particolare di risultati pi\`u
  generali, rimane ancora oggi lo strumento pi\`u utilizzato nella
  risoluzione variazionale delle equazioni semilineari ellittiche.

\pg;

{\bf Teorema.} Siano $X$ uno spazio di Banach, e $J \in C^1(X)$ un
funzionale che soddisfa la condizione (PS). Supponiamo che $J(0)=0$ e
che:
\begitems
\style n
* esistano $R>0$ e $a>0$ tali che se
$\|u\|=R$, allora $J(u) \geq a$;
* esista $u_0 \in X$ tale che
$\|u_0\|>R$ e e$J(u_0)<a$.
\enditems
Allora $J$ possiede un valore critico $c\geq a$. Pi\`u precisamente,
$$
c = \inf_{A \in {\cal B}} \max_{v \in A} J(v),
$$
dove
$$
{\cal B} = \left\{ \varphi([0,1]) \mid \varphi \in C([0,1],X),\;
\varphi(0)=0,\; \varphi(1)=u_0 \right\}.
$$
\pg+

{\em Dim.} Evidentemente ${\cal B} \neq \emptyset$.  Per ogni $A \in
{\cal B}$, l'intersezione $A \cap \{u \in X \mid \|u\|=R\} \neq
\emptyset$ per questioni di connessione. Dunque $\max_{v \in A} J(v)
\geq a$ e infine $c \geq a$.  \pg+

Se $c$ non fosse un valore critico, con le stesse notazioni del Lemma
di Deformazione e con $0<\ge < \ge_0$ sarebbe possibile trovare $A \in
{\cal B}$ tale che
$$
A = \varphi([0,1]), \quad c \leq \max_{v \in A} J(v) \leq c+\ge.
$$

\pg;

Ponendo $\psi(\tau) = \eta(1,\varphi(\tau))$ e $B = \psi([0,1])$,
risulta $B \in {\cal B}$. Ma allora $B \subset [J \leq c-\ge]$, in
contraddizione con il fatto che $\max_{v \in B} J(v) \geq c$.

\pg+

* Si osservi che ${\cal B}$ \`e costituito dalle immagini dei cammini
  che uniscono in $X$ l'origine con il punto $u_0$. In altri termini,
  stiamo facendo un minimax del funzionale $J$ sui cammini che partono
  dal punto di origine, e terminano nel punto $u_0$ esterno alla
  ``montagna'' $\{u \in X \mid \|u\|=R\}$.

\pg;

\sec Strumenti per applicazioni (non banali)

* Per poter apprezzare la portata della Teoria dei Punti Critici nella
  risoluzione di equazioni differenziali alle derivate parziali,
  dobbiamo raccogliere alcuni risultati fondamentali sugli operatori
  differenziali.

* Per la maggior parte di essi non daremo la dimostrazione, perch\'e
  non fa uso di tecniche inerenti al nostro corso.

* Un riferimento insuperabile, per quanto in lingua francese, \`e il
  libro di O. Kavian, Introduction \`a la th\'eorie des points
  critiques, edito da Springer.

\pg;

\sec Operatori ellittici del secondo ordine

* Siano $E$, $F$ due spazi di Banach. Un operatore lineare su $E$ a
valori in $F$ \`e una coppia $(A,D(A))$ in cui $D(A)$ \`e un
sottospazio vettoriale di $E$ e $A$ \`e un'applicazione lineare di
$D(A)$ in $F$.

* $D(A)$ \`e il dominio dell'operatore $A$. Il grafico di $A$ \`e
  l'insieme
  $$
  G(A) = \left\{ (u,Au) \mid u \in D(A) \right\} \subset E \times F.
  $$

* Se $(A,D(A))$ e $(B,D(B))$ sono due operatori di $E$ in $F$,
  scriveremo $A \subset B$ per indicare che $D(A) \subset D(B)$ e che
  $Au=Bu$ per ogni $u \in D(A)$. Quindi $A=B$ significa che $A \subset
  B$ e $B \subset A$.

* L'operatore $(A,D(A))$ \`e chiuso se $G(A)$ \`e chiuso in $E \times
  F$. Quando ci\`o accade, \`e possibile introdurre in $D(A)$ la norma
  $$
  \|u\|_{D(A)} = \|u\|_A = \|u\|_E + \|Au\|_F,
  $$
  detta appunto {\em norma del grafico}.

\pg;

{\bf Esempio.} Siano $\Omega$ un aperto di $\R^N$ e $[a_{ij}]$ una
matrice quadrata $N \times N$ definita quasi ovunque in $\Omega$ e
tale che:

esiste $\alpha>0$, tale che per ogni $\xi \in \R^N$, q.o. in $\Omega$,
si abbia $a(x) \xi \cdot \xi = \sum_{i,j=1}^N a_{ij}(x) \xi_i \xi_j
\geq \alpha |\xi|^2$.

Questa condizione \`e detta {\em coercivit\`a} della matrice
$[a_{ij}]$.

\pg+

Supporremo inoltre che $a_{ij} \in L^\infty(\Omega)$ per ogni $i$, $j$
tra $1$ e $N$.

\pg+

Definiamo allora
$$
D(A) = \left\{ u \in H_0^1(\Omega) \mid \sum_{i,j=1}^N \partial_i
\left( a_{ij} \partial_j u \right) \in L^2(\Omega) \right\}
$$
e
$$
Au = -\sum_{i,j=1}^N \partial_i
\left( a_{ij} \partial_j u \right) = - \mathop{\rm div}(a(\cdot)
\nabla u).
$$

\pg;

* Operatori di questo tipo sono chiamati {\em operatori ellittici del
secondo ordine in forma di divergenza}.

* Si verifica che tali operatori sono, nelle ipotesi suddette, chiusi.

* Nella definizione del dominio, \`e richiesto che la somma degli
  addendi $\partial_i \left( a_{ij} \partial_j u \right)$ appartenga a
  $L^2(\Omega)$, e non che ciascun addendo abbia questa propriet\`a.

* Il prossimo risultato raccoglie alcune propriet\`a utili di questi
  operatori, la cui dimostrazione deriva dal teorema di Lax-Milgram e
  da principi generali di analisi funzionale lineare. Indicheremo con
  $\| \cdot \|$ e con $\langle \cdot \mid \cdot \rangle$ la norma e il
  prodotto scalare in $L^2(\Omega)$.

\pg;

{\bf Lemma.} Siano $\Omega$ un aperto limitato di $\R^N$, $(A,D(A))$
un operatore ellittico del secondo ordine in forma di divergenza (con
le stesse ipotesi della definizione). Allora:
\begitems
\style n
* per ogni $f \in L^2(\Omega)$ esiste ed \`e unico $u \in D(A)$ tale
che: $u+Au=f$. Inoltre $(A+I)^{-1}$ \`e un operatore continuo di
$L^2(\Omega)$ in s\'e stesso e si ha $\|u\| \leq \|f\|$. Se $\Omega$
\`e limitato, $(A+I)^{-1}$ \`e un operatore compatto di $L^2(\Omega)$
in s\'e stesso.
* L'operatore $(A,D(A))$ \`e chiuso.
* $D(A)$ \`e denso in $L^2(\Omega)$.
\enditems

\pg;

* Pi\`u in generale, \`e possibile costruire operatori (differenziali)
  ellittici del secondo ordine in forma generale.

* Siano $\Omega$ un aperto di $\R^N$, $a = [a_{ij}]$ una matrice $N
  \times N$, $b=(b_i)$ e $\beta = (\beta_i)$ due vettori a $N$
  componenti, e $c$ una funzione.

* Definiamo due operatori del secondo ordine:
$$
\eqalign{%
Au &= - \sum_{i,j=1}^N \partial_i (a_{ij} \partial_j u) +
\sum_{j=1}^n \partial_j (\beta_j u) + b \cdot \nabla u + cu, \cr
Lu &= -\sum_{i,j=1}^N a_{ij} \partial_{ij} u + b \cdot \nabla u
+cu. \cr
}
$$

* Il secondo operatore non \`e in forma di divergenza, ed \`e pi\`u
  generale del primo.

\pg;

\sec Teoria spettrale

Per un operatore $A$ come sopra, supponiamo che $a_{ij}$, $\beta_j$,
$b_j$, $c$ siano funzioni di $L^\infty(\Omega)$. Supporremo sempre che
la matrice $[a_{ij}]$ sia coerciva con costante $\alpha>0$.

* Supponendo che $\Omega$ sia un aperto limitato e che
$$
D(A) = \left\{ u \in H_0^1(\Omega) \mid Au \in L^2(\Omega) \right\},
$$
si dimostra che esiste $\lambda_0 \in \R$ tale che per ogni $\lambda
\geq \lambda_0$ e ogni $u \in H_0^1(\Omega)$ si abbia (disuguaglianza
di G\aa rding)
$$
\langle Au+\lambda u \mid u \rangle \geq {\alpha \over 2} \|u\|^2.
$$

* Dunque l'operatore $(A+\lambda_0 I )^{-1}$ \`e continuo da
  $L^2(\Omega)$ in s\'e, e la sua immagine \`e $D(A)$.

\pg;

* Per il teorema di compattezza di Rellich, $(A+\lambda_0 I)^{-1}$ \`e
  un operatore compatto da $L^2(\Omega)$ in s\'e.

* Poich\'e $0$ non \`e un autovalore di $(A+\lambda_0 I)^{-1}$, esiste
  una successione $\{\mu_n\}_n$ si numeri complessi non nulli, tale
  che $\mu_n \to 0$, ogni $\mu_n$ \`e un autovalore di molteplicit\`a
  finita, e lo spettro di $(A+\lambda_0 I)^{-1}$ \`e $\{0\} \cup
  \{\mu_n \mid n \geq 1 \}$.

* Deduciamo infine che lo spettro di $A$ \`e costituito dai numeri
  $\lambda_n$, $n \geq 1$, definiti da
  $$
  \lambda_n = \frac{1}{\mu_n} - \lambda_0.
  $$

\pg;

\sec Un'applicazione del passo di montagna

{\bf Definizione.} Sia $\Omega$ un aperto di $\R^N$. Una funzione $f\colon \Omega \times \R \to \R$, $(x,s) \mapsto f(x,s)$, \`e una {\em funzione di Carath\'eodory}, se:
\begitems
* per ogni $s \in \R$, la funzione $f(\cdot,s)$ \`e misurabile su $\Omega$;
* per quasi ogni $x \in \Omega$, la funzione $f(x,\cdot)$ \`e continua su $\R$.
\enditems

\pg+

* Siano $\Omega$ un sottoinsieme aperto e limitato di $\R^N$, $g$ una funzione di Carath\'eodory e $Au=-\mathop{\rm div}(a(\cdot)\nabla u)$ un operatore ellittico, come visto nelle slide precedenti.

\pg+

* Ricordiamo che il primo autovalore di $A$ \`e caratterizzato da
$$
\eqalign{%}
\lambda_1 &= \inf \left\{ \int_\Omega a \nabla v \cdot \nabla v \mid v \in H_0^1(\Omega),\ \|v\|_2=1 \right\} \cr
&= \inf \left\{ \frac{\int_\Omega a \nabla v \cdot \nabla v}{\int_\Omega |v|^2} \mid v \in H_0^1(\Omega) \setminus \{0\}\right\} \cr}
$$

\pg;

Sia $G(x,s)=\int_0^s g(x,\sigma)\, d\sigma$, e definiamo il funzionale $J \colon H_0^1(\Omega) \to \R$ come
$$
\eqalign{%
J(u) &= {1 \over 2} \int_\Omega \left[ a \nabla u \cdot \nabla u + \lambda |u|^2 \right] dx \cr
&\qquad {}+\mu \int_\Omega G(x,u)\, dx - \langle f,u \rangle, \cr
}
$$
dove $\lambda \in \R$, $\mu \in \R$, $\mu \neq 0$, e $f \in H^{-1}(\Omega)$ sono fissati.

\pg+

* Vogliamo dimostrare che, sotto opportune ipotesi, $J$ soddisfa la condizione di Palais-Smale.

\pg;

{\bf Lemma.} Siano $\Omega$ un aperto limitato di $\R^N$ e $m \colon \Omega \to [0,+\infty)$ tale che $m>0$ quasi ovunque in $\Omega$. Supponiamo che esistano $b_1 \geq 0$ e $b_0 \in L^{p_0}(\Omega)$ con $p_0 > 2N/(N+2)$ se $N \geq 2$ e $p_0 =1$ se $N=1$, e infine $\theta$, $p \geq 1$ tali che $(N-2)p < N+2$, con la propriet\`a che
$$
m(x) |s|^\theta  \leq b_0(x) |s|+b_1 (1+|s|^{p+1})
$$
per ogni $s \in \R$ e per q.o. $x \in \Omega$. Allora per ogni $\ge>0$, esiste una costante $C(\ge)>0$ tale che per ogni $u \in H_0^1(\Omega)$:
$$
\int_\Omega |u|^2 \leq \ge \int_\Omega |\nabla u|^2 + C(\ge) \left( \int_\Omega m(x) |u|^\theta \, dx \right)^{2/\theta}.
$$

\pg;

{\em Dim.} La disuguaglianza da dimostrare \`e omogenea di grado
due. Ragionando per assurdo, supponiamo che esista $\ge >0$ tale che,
per ogni $n \geq 1$, esista $u_n \in H_0^1(\Omega)$ t.c. $\int_\Omega
|u_n|^2=1$ e
$$
1 > \ge \int_\Omega |\nabla u_n|^2 + n \left( \int_\Omega m(x)
|u_n|^{\theta}\, dx \right)^{2/\theta}.
$$
\pg+
La successione $\{u_n\}_n$ \`e limitata in $H_0^1(\Omega)$, e
passando ad una sottosuccessione possiamo d'ora in poi immaginare che
$u_n \rightharpoonup u$ in $H_0^1(\Omega)$ e $u_n \to u$ in
$L^2(\Omega)$ e puntualmente quasi ovunque.

Ne deduciamo che $\int_\Omega |u|^2=1$. L'ipotesi assurda garantisce
che
$$
\lim_{n \to +\infty} \int_\Omega m(x) |u_n|^\theta \, dx =0.
$$

\pg;

Per $N \geq 2$ abbiamo $p_0>2N/(N+2)$, sicch\'e $p_0' = p_0/(p_0-1) < 2^\*$; per il teorema di compattezza di Rellich, $u_n$ tende a $u$ in $L^{p_0'}(\Omega)$. Inoltre
$$
\|b_0(u-u_n)\|_{L^1} \leq \|b_0\|_{L^{p_0}} \| u-u_n \|_{L^{p_0'}},
$$
e dunque $b_0|u_n|$ tende a $b_0|u|$ in $L^1(\Omega)$ e puntualmente quasi ovunque. Allo stesso modo, $|u_n|^{p+1}$ tende a $|u|^{p+1}$ in $L^{1}(\Omega)$.

\pg+

Se
$$
\eqalign{%
Z_n (x) &= b_0(x) |u_n(x)| +b_1 (1+|u_n(x)|^{p+1}) \cr
Z(x) &= b_0(x) |u(x)| +b_1 (1+|u(x)|^{p+1}), \cr
}
$$
deduciamo che $Z_n$ tende a $Z$ in $L^1(\Omega)$ e puntualmente quasi
ovunque. Essendo $m |u_n| \leq Z_n$, da una versione generalizzata
della convergenza dominata deduciamo che $m|u_n|^\theta$ converge a
$m|u|^\theta$ in $L^1(\Omega)$ e che:
$$
\lim_{n \to +\infty} \int_\Omega m(x) |u_n|^\theta \, dx = \int_\Omega m(x) |u|^\theta \, dx.
$$

\pg;

Concludiamo che $\int_\Omega m(x) |u|^\theta \, dx=0$, cio\`e $u =0$
q.o. contro l'ipotesi di normalizzazione $\int_\Omega |u|^2 =1$.

\pg+

{\bf Proposizione.} Sia $\Omega$ un aperto limitato di $\R^N$. Se $N
\geq 2$, sia $p_0 = \ge+N/2$ per $\ge>0$, e $p_0=1$ se $N=1$. Sia
$p>1$ tale che $(N-2)p<N+2$; supponiamo che $g$ soddisfi le condizioni
\begitems
* esistono $b_0 \in L^{p_0}(\Omega)$, $b_1 \geq 0$, tali che $|g(x,s)|
\leq b_0(x) + b_1 |s|^{p}$;
* esistono $\theta >2$, $R>0$ tali che se $|s|\geq R$, $0 < \theta
G(x,s) \leq s g(x,s)$.
\enditems
Allora se $\lambda\in\R$, $\mu \neq 0$, $f \in H^{-1}(\Omega)$, il
funzionale $J$ soddisfa la condizione (PS) su $H_0^1(\Omega)$. Se
$\Omega$ \`e un aperto di classe $C^1$ e $f \in L^q(\Omega)$ con $q
\geq 2N/(N+2)$, il funzionale $J$ soddisfa la condizione (PS) su
$H^1(\Omega)$.

\pg;

{\em Dim.} Cominciamo da una conseguenza della seconda ipotesi su
$g$. Se $s \geq R$ allora $\theta s^{-1} \leq g(x,s)/G(x,s)$ e di
conseguenza $G(x,s) \geq G(x,R) R^{-\theta} s^\theta$ sull'intervallo
$[R,+\infty)$. Deduciamo che per qualche funzione $c_1 \in
L^{p_0}(\Omega)$, per ogni $(x,s) \in \Omega \times \R$:
$$
G(x,s) \geq \min \left\{ G(x,-R),G(x,R) \right\} R^{-\theta}
|s|^\theta - c_1(x). \eqno(1)
$$
\pg+
La prima condizione su $g$ implica ormai $\theta < 2^\*$. Consideriamo
ora una successione $\{u_n\}_n$ in $H_0^1(\Omega)$ tale che $J(u_n)
\to c$ e $J'(u_n) \to 0$.
\pg+
Riscriviamo quest'ultima condizione nella forma
$$
h_n = J'(u_n) = A u_n + \lambda u_n +\mu g(\cdot,u_n) -f \to 0
$$
in $H^{-1}(\Omega)$.

\pg;

* Mostriamo che la successione $\{u_n\}_n$ \`e limitata. Infatti, da
  $\theta>2$ e usando il fatto che esiste $c_2 \in L^{p_0}(\Omega)$
  tale che $\theta G(x,s) \leq s g(x,s) + c_2(x)$, per una costante
  $C>0$ indipendente da $n$ otteniamo:
  $$
  \eqalign{% \
  \int_\Omega G(x,u_n(x))\, dx &\leq {1 \over \theta-2} \int_\Omega
  [u_n(x)g(x,u_n(x))-2G(x,u_n(x))]\, dx + C \cr
  &= {1 \over \mu (\theta-2)} [\langle h_n-f,u_n \rangle - 2J(u_n)]+C
  \cr
  &\leq C (1+\|\nabla u_n\|).
  }
  $$
  \pg+
  D'altronde, ponendo $m(x) = R^{-\theta}\min\{1,G(x,R),G(x,-R)\}$, il
  Lemma precedente e l'equazione (1) garantiscono che
  $$
  \int_\Omega |u_n|^2 \leq \ge \|\nabla u_n\|^2 + C(\ge) (1+\| \nabla
  u_n\| )^{2\theta}.
  $$

\pg;

Poich\'e $a(x)\xi \cdot \xi \geq \alpha |\xi|^2$ con $\alpha>0$, e
poich\'e $J(u_n)$ resta limitata, se $\mu<0$
$$
\eqalign{%
\alpha \|\nabla u_n\|^2 &\leq 2 J(u_n) + \|f\|_{H^{-1}} \|\nabla u_n\|
+ |\lambda| \|u_n\|^2 \cr
&\qquad {}-\mu \int_\Omega G(x,u_n) \cr
&\leq C + \ge \|\nabla u_n\|^2 + C(\ge) (1+\|\nabla u_n\|)^{2/\theta}.
}
$$
\pg+
Se $\mu>0$, ricordiamo che (1) implica $\int_\Omega G(x,u_n)\, dx \geq
-C$, e verifichiamo facilmente che l'ultima disuguaglianza continua ad
essere soddisfatta. A questo punto, la scelta di $\ge>0$
sufficientemente piccolo (ad esempio $\ge=\alpha/2$) garantisce che
$\{u_n\}_n$ \`e una successione limitata in $H_0^1(\Omega)$.

\pg+

* Esiste pertanto una sottosuccessione $\{u_{n_j}\}_j$ convergente
  verso $u$ in senso debole in $H_0^1(\Omega)$, in senso forte in ogni
  $L^r(\Omega)$ con $(N-2)r<N+2$, e puntualmente quasi ovunque.

\pg;

Come nella dimostrazione del precedente Lemma, se $r_0 =
\min\{p_0,(p+1)/p\}$, allora $g(\cdot, u_{n_j})$ converge q.o. e in
$L^{r_0}(\Omega)$ verso $g(\cdot,u)$. Per interpolazione, lo stesso
accade in $L^{(p+1)/p}(\Omega)$, dunque a maggior ragione in
$H^{-1}(\Omega)$.

\pg+

Come visto sopra, denotando con $\tilde{A}$ l'estensione di $A$ a
$H_0^1(\Omega)$, $\tilde{A}$ risulta essere un isomorfismo di
$H_0^1(\Omega)$ su $H^{-1}(\Omega)$, e pertanto:
$$
\displaylines{%
u_{n_j} = \tilde{A}^{-1} [ \ge_{n_j}+f+\lambda u_{n_j} + \mu
g(\cdot,u_{n_j})] \cr
\to \tilde{A}^{-1} [f+\lambda u + \mu g(\cdot,u)].
}
$$
La dimostrazione \`e ormai conclusa.

\pg;

\sec Il passo di montagna e l'esistenza di soluzioni.

Per semplicit\`a, rafforziamo leggermente le ipotesi sulla funzione di
Carath\'eodory $g$:
\begitems
* esistono $b_0 \in L^\infty(\Omega)$, $b_1 \geq 0$, tali che $|g(x,s)|
\leq b_0(x) + b_1 |s|^{p}$;
* esistono $\theta >2$, $R>0$ tali che se $|s|\geq R$, $0 < \theta
G(x,s) \leq s g(x,s)$.
\enditems
Supponiamo inoltre che per ogni $\ge>0$, esista $\delta>0$ tale che
$|G(\cdot,s)| \leq \ge |s|^2$ per ogni $|s| \leq \delta$.

\pg+

{\bf Teorema.} se $\lambda<\lambda_1$, allora il problema di Dirichlet
$$
\cases{%
Au = \lambda u + g(\cdot,u) &in $\Omega$ \cr
u=0 &su $\partial\Omega$ \cr
}
$$
possiede una soluzione non identicamente nulla.

\pg;

{\em Dim.} Consideriamo il funzionale
$$
J(u) = {1 \over 2} \left[ \langle Au \mid u \rangle - \lambda \|u\|^2
\right] - \int_\Omega G(x,u(x))\, dx,
$$
funzionale dell'energia associato al nostro problema di Dirichlet in
$H_0^1(\Omega)$. Dimostriamo che il teorema del passo di montagna \`e
applicabile.

\pg+

Per le ipotesi su $g$, dato $\ge>0$ esiste una costante $C_\ge>0$ tale
che
$$
\eqalign{%
\left| \int_\Omega G(x,u(x))\, dx \right| &\leq \int_\Omega
|G(x,u(x))| \, dx \cr
&\leq \ge \|u\|^2 + C_\ge \int_\Omega |u(x)|^{p+1}\, dx \cr
&\leq \ge \|u\|^2 + C_\ge \|\nabla u \|^{p+1}.
}
$$

\pg;

Non appena $\lambda+2 \ge < \lambda_1$,
$$
\eqalign{%
J(u) &\geq \frac12 \left[ \langle Au\mid u \rangle - \lambda \|u\|^2
\right] - \ge \|u\|^2 - C_\ge \|\nabla u\|^{p+1} \cr
&\geq \frac{\alpha(\lambda_1-\lambda -2\ge)}{2 \lambda_1} \|\nabla
u\|^2 - C_\ge \|\nabla u\|^{p+1}.
}
$$
\pg+
Scegliendo $R_0^{p-1} = \alpha(\lambda_1-\lambda -2\ge) / (2 \lambda_1
C_\ge)$, se $\|\nabla u\|=R<R_0$, allora esiste $b=b(R)>0$ tale che
$J(u) \geq b$. L'origine \`e pertanto un minimo locale di $J$ su
$H_0^1(\Omega)$.

\pg+

Per trovare un punto $u_0 \in H_0^1(\Omega)$ tale che $J(u_0)<0$,
ricordiamo innanzitutto che $G(x,s) \geq m(x) |s|^\theta -C_1$ per
qualche funzione $m \in L^\infty(\Omega)$ con $m>0$ q.o., e
$\theta>2$.

\pg+

Sia $\varphi_1 \in H_0^1(\Omega)$ (con $\|\varphi_1\|_2=1$)
un'autofunzione associata all'autovalore $\lambda_1$ di $A$. 

\pg;

Per ogni $t>0$ otteniamo
$$
J(t\varphi_1) \leq \frac12 (\lambda_1 - \lambda) t^2 + C_2 - t^\theta
\int_\Omega m(x) |\varphi_1(x)|^\theta \, dx.
$$
Essendo $\theta > 2$, \`e sufficiente scegliere $t>0$ abbastanza
grande per dedurre che $J(t \varphi_1)<0$.

\pg+

Sappiamo gi\`a (anche sotto ipotesi pi\`u deboli) che $J$ soddisfa la
condizione (PS) su $H_0^1(\Omega)$. Quindi un'applicazione del Teorema
del Passo di Montagna fornisce un punto critico di $J$ a livello $c
\geq b >0$. In particolare, questo punto critico \`e una soluzione non
identicamente nulla del nostro problema di Dirichlet.

\pg;

\sec L'identit\`a di Pohozaev e assenza di soluzioni

* L'esistenza di soluzioni per i problemi di Dirichlet su domini
  limitati non \`e un fatto scontato. 

* In questa sezione introdurremo una notevole {\em identit\`a
  variazionale} e ne dedurremo una {\em ostruzione} all'esistenza di
  soluzioni non nulle.

* Consideriamo il problema di Dirichlet
$$
\cases{ -\Delta u = f(u) &in $\Omega$ \cr
u=0 &su $\partial\Omega$. \cr
} \eqno(P)
$$
Sia $F(u) = \int_0^u f(s) \, ds$.

\pg;

{\bf Teorema.} Sia $\Omega$ un dominio limitato di $\R^N$ con
frontiera regolare, e sia $\nu$ la normale esterna a
$\partial\Omega$. Se $u$ \`e una soluzione classica del problema (P)
(ma $u \in H_0^1(\Omega) \cap H^2(\Omega)$ basterebbe per ottenere la
stessa conclusione), allora \`e soddisfatta l'identit\`a di Pohozaev
$$
N \int_\Omega F(u(x))\, dx - {N-2 \over 2} \int_\Omega u(x)f(u(x))\,
dx = \frac12 \int_{\partial\Omega} \left| {\partial u \over \partial
\nu} \right|^2 (x \cdot \nu) \, d\sigma.
$$

\pg+

* Questo risultato \`e essenzialmente dovuto a S. Pohozaev, ed \`e
  stato ampiamente generalizzato da P. Pucci e J. Serrin negli anni
  '80.

\pg;

{\em Dim.} Definiamo $\Theta(x) = (x \cdot \nabla u(x))\nabla
u(x)$. Un calcolo diretto mostra che
$$
\eqalign{%
\mathop{\rm div} \Theta (x) &= \Delta u (x \cdot \nabla u) + \sum_k
\frac{\partial u}{\partial x_k} \frac{\partial}{\partial x_k} \left(
\sum_i x_i \frac{\partial u}{\partial x_i} \right) \cr
&= \Delta u (x \cdot \nabla u) + \sum_k \left| \frac{\partial
u}{\partial x_k} \right|^2  + \sum_{i,k} \frac{\partial u}{\partial
x_k} x_i \frac{\partial^2 u}{\partial x_i \partial x_k} \cr
&= \Delta u (x \cdot \nabla u) +|\nabla u|^2 + \frac12 \sum_i
\frac{\partial}{\partial x_i} |\nabla |^2.
}
$$
\pg+
Il teorema della divergenza mostra allora che 
$$
\eqalign{%
&\int_\Omega \Delta u (x \cdot \nabla u) \, dx +\int_\Omega |\nabla
u|^2 \, dx + \frac12 \int_\Omega \sum_i \frac{\partial}{\partial x_i}
|\nabla |^2 \, dx \cr
&= \int_{\partial\Omega} (x \cdot \nabla u)(\nabla u \cdot \nu)\,
d\sigma. \cr
}
$$
\pg;

Poich\'e $u=0$ su $\partial\Omega$, risulta che $\nabla u = (\partial
u / \partial \nu)\nu$ su $\partial\Omega$, sicch\'e
$$
\eqalign{%
\int_\Omega \Delta u (x \cdot \nabla u) \, dx +\int_\Omega |\nabla
u|^2 \, dx &+ \frac12 \int_\Omega \sum_i \frac{\partial}{\partial x_i}
|\nabla |^2 \, dx \cr
&= \int_{\partial \Omega} (x \cdot \nu) \left| \frac{\partial
u}{\partial \nu} \right|^2 \, d\sigma. \cr
}
$$
\pg+
Sia ora $\Theta_1(x) = \frac12 |\nabla u|^2 x$. Poich\'e
$$
\mathop{\rm div} \Theta_1 = \frac{N}{2} |\nabla u|^2 + \frac12 \sum_i
x_i \frac{\partial}{\partial x_i} |\nabla u|^2,
$$
una seconda applicazione del teorema della divergenza mostra che
$$
\frac{N}{2} \int_\Omega |\nabla u|^2 \, dx + \frac12 \int_\Omega
\sum_i x_i \frac{\partial}{\partial x_i} |\nabla u|^2 \, dx = \frac12
\int_{\partial\Omega} (x\cdot \nu) \left| \frac{\partial u}{\partial
\nu} \right|^2 \, d\sigma.
$$

\pg;

Confrontando queste due identit\`a, troviamo
$$
\int_\Omega \Delta u (x\cdot \nabla u)\, dx + \left(1-\frac{N}{2}
\right) \int_\Omega |\nabla u|^2 \, dx = \frac{1}{2}
\int_{\partial\Omega} (x \cdot \nu) \left| \frac{\partial u}{\partial
\nu} \right|^2 \, d\sigma.
$$
\pg+
\`E venuto il momento di inserire l'informazione che $u$ risolve (P):
$$
\eqalign{%
\int_\Omega \Delta u (x\cdot \nabla u)\, dx &= \int_\Omega f(u(x))
(x\cdot \nabla u)\, dx \cr
&= \int_\Omega f(u(x)) \sum_i x_i \frac{\partial u}{\partial x_i} \,
dx \cr
&= \int_\Omega \sum_i x_i \frac{\partial F(u(x))}{x_i} \, dx. \cr
}
$$

\pg;

Un'integrazione per parti nell'ultimo integrale mostra che
$$
\int_\Omega \sum_i x_i \frac{\partial F(u(x))}{x_i} \, dx. = -N
\int_\Omega F(u(x))\, dx,
$$
e pertanto
$$
\int_\Omega \Delta u (x \cdot \nabla u)\, dx = N \int_\Omega F(u(x))\,
dx.
$$
\pg+
Sempre da (P), moltiplicando per $u$ e integrando:
$$
\int_\Omega |\nabla u|^2 \, dx = \int_\Omega u f(u)\, dx.
$$
Con queste uguaglianze, otteniamo finalmente l'identit\`a di Pohozaev.

\pg;

* Come applicazione di questa identit\`a variazionale, mostriamo che
  l'ipotesi di crescita {\em sottocritica} di $f$ ($f(u) \sim |u|^p$
  con $p<(N+2)/(N-2)$) \`e realmente necessaria per l'esistenza di
  soluzioni non banali.

\pg+

* Consideriamo il problema di Dirichlet ``modello''
$$
\cases{%
-\Delta u = |u|^{p-1}u &in $\Omega$ \cr
u=0 &su $\partial\Omega$ \cr
}
$$

\pg+

{\bf Corollario.} Se $\Omega$ \`e {\em stellato} rispetto all'origine
di $\R^N$, cio\`e se $\nu \cdot x > 0$ per ogni $x
\in \partial\Omega$, allora ogni soluzione classica del problema
modello soddisfa l'identit\`a
$$
\left( {N \over p+1} - {N-2 \over 2} \right) \int_\Omega |u|^{p+1} \,
dx >0,
$$
e dunque $u \neq 0$ implica $p<(N+2)/(N-2)$.

\pg+

* Questi risultati sono molto sensibili alle piccole perturbazioni del
  problema. \`E noto (Brezis e Nirenberg, 1983) che l'equazione
  $-\Delta u = \lambda u + u^{{N+2 \over N-2}}$ possiede soluzioni
  positive non banali in qualunque dominio $\Omega$, almeno per
  opportuni valori di $\lambda \in \R$.

\pg;

\sec E se la condizione (PS) fallisce?

* Uno dei problemi pi\`u interessanti della Teoria dei Punti Critici
  \`e l'analisi della {\em perdita di compattezza}.

\pg+

* Abbiamo presentato risultati di esistenza di punti critici sotto
  ipotesi di compattezza globale dello spazio. Poi siamo passati ad
  ipotesi di compattezza per opportune successioni (di Palais-Smale).

\pg+

* Ma se un funzionale non soddisfa la condizione (PS), \`e inevitabile
  gettare la spugna?

\pg;

* Ricordiamo la notazione $A_\alpha = \{x \in A \mid d(x,A) \leq \alpha\}$.

\sec Un lemma di deformazione senza (PS)

{\bf Lemma (M. Willem).} Sia $J \in C^1(X)$ un funzionale sullo spazio
di Banach $X$.  Siano $S \subset X$, $c \in \R$, $\epsilon>0$,
$\delta>0$ tali che
$$
\|J'(u)\| \geq {4 \epsilon \over \delta}
$$
per ogni $u \in [c-2\epsilon \leq J \leq c+2\epsilon] \cap
S_{2\delta}$. Allora esiste un mappa continua $\eta \in C([0,1] \times
X,X)$ tale che, per ogni $u \in X$ e $t \in [0,1]$, si abbia:
\begitems
\style n
* $\eta(0,u)=u$
* $\eta(t,u)=u$ se $u \notin [c-2\epsilon \leq J \leq c+2 \epsilon]
\cap S_{2 \delta}$
* $\eta(1,[J \leq c+\epsilon]\cap S) \subset [J\leq c-\epsilon] \cap
S_\delta$
* $\eta(t,\cdot)$ \`e un omeomorfismo di $X$ in $X$.
\enditems

\pg;

{\em Dim.} Siano
$$
\eqalign{%
A &= [c-2\epsilon \leq J \leq c+2\epsilon] \cap S_{2\delta} \cr
B &= [c-\epsilon \leq J \leq c+\epsilon] \cap S_{\delta} \cr
}
$$
in modo tale che $B \subset A \subset X_r$. Sia $V \colon X_r \to X$
un campo vettoriale pseudogradiente per $J$. Definiamo infine una
mappa lipschitziana $\rho \colon X \to \R$ tale che $0 \leq \rho \leq
1$, $\rho=1$ in $B$ e $\rho=0$ in $X \setminus A$.

\pg+

Poniamo $f \colon X \to X$,
$$
f(u) = -\rho(u) {V(u) \over \|V(u)\|}.
$$
\pg+
Poich\'e $\|f(u)\|\leq 1$ per ogni $u$, il problema di Cauchy
$$
\cases{%
{dw \over dt} = f(w) \cr
w(0)=u \cr
}
$$
possiede, per ogni $u \in X$ soluzione unica definita per tutti i
tempi $t \geq 0$.

\pg;

Definiamo $\eta \colon [0,1] \times X \to X$, $\eta(t,u)=w(\delta
t,u)$. Le propriet\`a 1., 2. e 4. si dimostrano come nel lemma di
deformazione gi\`a visto.

\pg+

Per dimostrare 3, cominciamo ad osservare che per $t \geq 0$,
$$
\|w(t,u)-u\| \leq \int_0^t \| f(w(\tau,u))\| \, d\tau \leq t,
$$
sicch\'e $w(t,S) \subset S_\delta$ per ogni $t \in [0,\delta]$. Quindi
$\eta(t,S) \subset S_\delta$ per ogni $t \in [0,1]$.

Notiamo inoltre che, per ogni $u \in X$ fissato, la funzione $t
\mapsto J(w(t,u))$ \`e monotona decrescente.

\pg+

Sia $u \in [J \leq c+\epsilon] \cap S$. distinguiamo due casi:
\begitems
* se $J(w(\hat t,u)) <c-\epsilon$ per qualche $\hat t \in [0,\delta)$,
allora $J(\eta(1,u))=J(w(\delta,u)) \leq J(w(\hat t,u)) <c-\epsilon$.
\enditems

\pg;

\begitems
* Altrimenti abbiamo $w(t,u) \in [c-\epsilon\leq J \leq c+\epsilon]
\cap S_\delta = B$ per ogni $t \in [0,\delta]$, pertanto
$$
\eqalign{%
J(w(t,u)) &= J(u) + \int_0^\delta {d \over dt} J(w(t,u)) \, dt \cr
&\leq J(u) - \int_0^\delta {1 \over 2} \|J'(w(t,u))\| \, dt \cr
&\leq c+\epsilon -{1 \over 2} {4\epsilon \over \delta} \delta =
c-\epsilon. \cr
}
$$
\pg+
In entrambi i casi, $\eta(1,u) \in [J \leq c-\epsilon] \cap S_\delta$
se $u \in [J \leq c+\epsilon] \cap S_{2\delta}$. La dimostrazione \`e
completa.

\pg+

* Il senso di questo Lemma (per $S=X$) \`e il seguente: se la derivata del
  funzionale $J$ \`e ben discosta da zero nella striscia $[c-2\epsilon
  \leq J \leq c+2\epsilon]$, allora $[J \leq c+\epsilon]$ pu\`o essere
  deformato con continuit\`a in $[J \leq c-\epsilon]$.

* La stessa dimostrazione gi\`a vista del teorema del passo di
  montagna permette di concludere che {\em esiste una successione (PS)
  al livello} $c$ definito in quell'enunciato.

\pg;

\sec Un approccio alternativo

* L'uso dei teoremi di minimax non \`e l'unico approccio possibile
  alla risoluzione di equazioni semilineari ellittiche.

\pg+

* Affronteremo ora lo studio di un problema di Dirichlet, su un
  dominio limitato $\Omega$, mediante una tecnica di {\em
  minimizzazione vincolata}.

\pg+

* \`E possibile mostrare, sotto ipotesi abbastanza generiche, che la
  risoluzione con il Passo di Montagna \`e {\em equivalente} alla
  risoluzione per minimizzazione vincolata.

\pg;

* Sia $2<p<2^\*=\frac{2N}{N-2}$ un numero reale.

* Cerchiamo una funzione $u$ che soddisfi il problema {\em
  superlineare} e {\em sottocritico}
  $$
  \cases{-\Delta u + q(x) u = |u|^{p-2}u &in $\Omega$ \cr
  u=0 &su $\partial\Omega$. \cr
  }\eqno(P)
  $$

* Supporremo valide, fino a diversa indicazione, le seguenti ipotesi:
  $\Omega$ \`e un aperto limitato di $\R^N$ con $N \geq 3$, $q \in
  L^\infty(\Omega)$ e $q(x) \geq 0$ per q.o. $x \in \Omega$.

\pg+

{\bf Teorema.} Sotto le predette ipotesi, il problema (P) possiede
almeno una soluzione non banale e non negativa in $\Omega$.

\pg+

* L'ipotesi che $q\geq 0$ in $\Omega$ garantisce che il primo
  autovalore dell'operatore $-\Delta + q(\cdot)$ sia strettamente
  positivo. Ci\`o implica che la norma standard di $H_0^1(\Omega)$ sia
  equivalente a quella indotta dal prodotto scalare
  $$
  \langle u\mid v \rangle = \int_\Omega \nabla u \cdot \nabla v \, dx
  + \int_\Omega q(x)uv\, dx.
  $$
  D'ora in avanti, lo spazio $H_0^1(\Omega)$ sar\`a normato in questo
  modo.

\pg;

* Le soluzioni (deboli) di (P) corrispondono ai punti critici del
  funzionale $I \colon H_0^1(\Omega) \to \R$,
  $$
  I(u) = \frac12 \int_\Omega |\nabla u|^2 \, dx + \frac12 \int_\Omega
  q(x) |u|^2 \, dx - \frac1p \int_\Omega |u|^{p}\, dx.
  $$

\pg+

* Sappiamo che questo funzionale \`e derivabile secondo Fr\'echet, per
  gli esempi visti in precedenza.

\pg+

* Il funzionale $I$ non \`e limitato dal basso:
$$
\lim_{t \to +\infty} I(tu) = \lim_{t \to +\infty} \frac{t^2}{2}
\|u\|^2 - \frac{t^p}{p} \|u\|_p^p = -\infty
$$
per ogni $u \neq 0$, dal momento che $p>2$.

\pg;

\sec Minimizzazione sulle sfere

* Per ogni $\beta>0$, definiamo la {\em sfera}
$$
\Sigma_\beta = \left\{ u \in H_0^1(\Omega) \mid \int_\Omega |u|^p \,
dx = \beta \right\}.
$$

\pg+

* Se $u \in \Sigma_\beta$, allora $I(u) = \frac{1}{2} \|u\|^2 -
  \frac{\beta}{p}$. In particolare, $I$ \`e limitato dal basso su
  $\Sigma_\beta$.

\pg+

* Introduciamo il problema di minimo
$$
m_\beta = \inf_{u \in \Sigma_\beta} \|u\|^2.
$$

\pg;

{\bf Lemma.} Per ogni $\beta>0$, il livello $m_\beta$ \`e raggiunto da
una funzione non negativa. Esiste pertanto $u \in \Sigma_\beta$, $u
\geq 0$ quasi ovunque in $\Omega$, tale che $\|u\|^2 = m_\beta$.

\pg+

{\em Dim.} Sia $\{u_k\}_k$ una successione minimizzante per
$m_\beta$. Evidentemente anche la successione $\{|u|_k\}_k$ \`e
minimizzante per $m_\beta$: possiamo d'ora in poi supporre che $u_k
\geq 0$ quasi ovunque in $\Omega$.

\pg+

Poich\'e questa successione minimizzante \`e limitata (perch\'e?),
possiamo estrarre una sottosuccessione e supporre che: $u_k
\rightharpoonup u$ in $H_0^1(\Omega)$, $u_k \to u$ in $L^p(\Omega)$ e
puntualmente quasi ovunque.

\pg+ 

\`E noto dall'analisi funzionale lineare che ogni norma \`e
semicontinua inferiormente rispetto alla topologia debole, dunque
$$
\|u\|^2 \leq \liminf_{k \to +\infty} \|u_k\|^2 = m_\beta.
$$

\pg+

Per la convergenza in $L^p$, risulta
$$
\int_\Omega |u|^p \, dx = \beta.
$$

\pg;

La convergenza puntuale quasi ovunque implica che $u \geq 0$ quasi
ovunque in $\Omega$, e tutto questo dimostra che $u$ \`e una soluzione
non negativa del problema $m_\beta$.

\pg+

* L'ipotesi che $p<2^\*$ ha permesso di passare al limite nella
  convergenza forte di $L^p(\Omega)$, e di dimostrare che il limite
  {\em debole} appartiene al vincolo $\Sigma_\beta$.

\pg+

{\bf Lemma.} Sia $u$ il minimo per $m_\beta$ costruito nel lemma
precedente. Allora $u$ soddisfa
$$
\int_\Omega \nabla u \cdot \nabla v \, dx + \int_\Omega q(x)uv \, dx =
{m_\beta \over \beta} \int_\Omega |u|^{p-2}uv \, dx
$$
per ogni $v \in H_0^1(\Omega)$.

\pg+

{\em Dim.} Cominciamo da un'osservazione fondamentale. Sebbene $u$
minimizzi il funzionale $N(u)=\|u\|^2$ su $\Sigma_\beta$, non possiamo
dedurre che la derivata di $N$ si annulla in $u$. Infatti
$\Sigma_\beta$ {\em non} \`e un sottospazio vettoriale: non \`e lecito
confrontare i valori di $N(u)$ e $N(u+v)$, poich\'e in generale $u+v
\notin \Sigma_\beta$.

\pg;

Fissiamo $v \in H_0^1(\Omega)$. Per $s \in \R$ abbastanza piccolo,
diciamo $s \in (-\ge,\ge)$, la funzione $u+sv$ non \`e identicamente
nulla. Pertanto esiste $t \colon (-\ge,\ge) \to (0,+\infty)$ tale che
$$
\int_\Omega |t(s)(u+sv)|^p\, dx = \beta.
$$
Esplicitamente,
$$
t(s) = \left( \frac{\beta}{\int_\Omega |u+sv|^p \, dx} \right)^{1/p}.
$$
\pg+

La funzione $s \mapsto t(s)(u+sv)$ definisce una curva in
$\Sigma_\beta$ che passa per $u$ al ``tempo'' $s=0$. La funzione $t$
\`e derivabile in $(-\ge,\ge)$, e risulta
$$
t'(s)=-\beta^{1/p} \left( \int_\Omega |u+sv|^p\, dx \right)^{-{1 \over
p}-1} \int_\Omega |u+sv|^{p-2}(u+sv)v \, dx.
$$
Inoltre $t(0)=1$, $t'(0)=-\beta^{-1}\int_\Omega |u|^{p-2}uv\, dx$.

\pg;

Definiamo $\gamma \colon (-\ge,\ge) \to \R$,
$$
\gamma(s) = \left\| t(s)(u+sv) \right\|^2.
$$
Poich\'e $t(s)(u+sv) \in \Sigma_\beta$ per ogni $s$, il punto $s=0$
\`e un minimo per $\gamma$. Essendo
$$
\gamma'(s) = 2 \langle t(s)(u+sv)\mid t'(s)(u+sv)+t(s)v \rangle,
$$
abbiamo
$$
0 = \gamma'(0)= -2 \frac{m_\beta}{\beta} \int_\Omega |u|^{p-2}uv\, dx+
2 \langle u\mid v \rangle.
$$
Quindi
$$
\langle u \mid v \rangle = \frac{m_\beta}{\beta} \int_\Omega
|u|^{p-2}uv \, dx
$$
per ogni $v \in H_0^1(\Omega)$, e la dimostrazione \`e conclusa.

\pg+

* Chi ha qualche nozione di geometria differenziale, probabilmente ha
  notato che abbiamo utilizzato il concetto di derivata su variet\`a
  (immerse), e pi\`u precisamente la condizione che $dN(u)=0$ su $T_u
  \Sigma_\beta$.

\pg;

Per concludere la dimostrazione del teorema di esistenza di una
soluzione al problema (P), dobbiamo in qualche modo... sbarazzarci del
fattore moltiplicativo $m_\beta/\beta$. Questo numero appare come un
{\em moltiplicatore di Lagrange} subordinato al vincolo
$\Sigma_\beta$. \`E a questo punto che entra in giorno, in maniera
cruciale l'omogeneit\`a del vincolo.

\pg+

Sia $u$ il minimo di $N$ su $\Sigma_\beta$. Poniamo $u=cw$, con $c \in
\R$ da determinare. Sappiamo che $u$ soddisfa
$$
c \langle w \mid v \rangle = {m_\beta \over \beta} c^{p-1} \int_\Omega
|w|^{p-2}wv\, dx
$$
per ogni $v \in H_0^1(\Omega)$. Possiamo allora scegliere
$c=(\beta/m_\beta)^{\frac{1}{p-2}}$ e dedurre che
$$
\langle w \mid v \rangle = \int_\Omega
|w|^{p-2}wv\, dx
$$
per ogni $v \in H_0^1(\Omega)$. La funzione $w$ \`e pertanto la
soluzione cercata.
 







\pg. %------------------------------FINE-------------------------------
