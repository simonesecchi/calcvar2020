\input ctuslides2


\def\R{{\bbchar R}}

\slideshow 

\tit Calcolo delle Variazioni\nl 
     a.a. 2019-2020\nl

\subtit\bf Simone Secchi\nl simone.secchi@unimib.it

\subtit\rm \url{http://elearning.unimib.it}

\pg;

\sec Prerequisiti e strumenti

* Calcolo differenziale in spazi euclidei di dimensione finita
* Teoria della misura e dell'integrazione secondo Lebesgue
* Principi di Analisi Funzionale Lineare
* Teoria elementare degli spazi di Sobolev (almeno il caso hilbertiano $p=2$)


\pg;

\sec Strumenti: il calcolo differenziale in dimensione infinita

{\bf Notazione.} Se $X$ \`e uno spazio di Banach (reale), il suo duale
topologico sar\`a denotato con il simbolo $X^\*$. Se $A \in X^\*$, il
simbolo $A[v]$ indicher\`a il valore di $A$ nel punto $v$; talvolta
semplificheremo la notazione e scriveremo $Av$ al posto di $A[v]$.

\bigskip

{\bf Definizione.} Siano $X$ uno spazio di Banach, e $U \subset X$ un
suo aperto. Un funzionale su $U$ \`e un'applicazione $I \colon U \to
\R$. Si noti che i ``nostri'' funzionali {\bf non} sono
necessariamente {\bf lineari}!

\pg;

{\bf Definizione.} Sia $I \colon U \to \R$ un funzionale. Diremo che
$I$ \`e derivabile secondo Fr\'echet nel punto $u \in U$ se esiste un
elemento $A \in X^\*$ tale che
$$
\lim_{\|v\| \to 0} {I(u+v)-I(u)-Av \over \|v\|} =0, \eqmark
$$
o, equivalentemente, se
$$
I(u+v)=I(u)+Av+o(\|v\|) \quad\hbox{per $v \to 0$}.
$$

\medskip

Si osservi che questa \`e la definizione di funzione differenziabile
quando $X=\R^n$.

\pg;

{\bf Lemma.} Se $I$ \`e derivabile nel punto $u \in U$, allora
l'elemento $A$ che soddisfa (1) \`e univocamente determinato.

\smallskip

{\em Dim.} Infatti, supponiamo che $A$ e $B$ siano due elementi di
$X^\*$ che soddisfano (1). Per sottrazione,
$$
\lim_{\|v\| \to 0} {(A-B)v \over \|v\|} =0.
$$
Fissiamo $u \in X$ con $\|u\|=1$, e scegliamo $v=tu$, $t \to
0^+$. Allora
$$
(A-B)u=\lim_{t \to 0+} {t(A-B)u \over t\|u\|} =0.
$$
Per l'arbitrariet\`a di $u$, concludiamo che $A=B$.

\pg;

{\bf Definizione.} Se $I$ \`e un funzionale derivabile secondo
Fr\'echet nel punto $u \in U$, la derivata (talvolta: il
differenziale) di Fr\'echet di $I$ in $u$ \`e l'unico elemento $I'(u)
\in X^\*$ (talvolta: $dI(u)$) tale che
$$
I(u+v)=I(u)+I'(u)[v]+o(\|v\|)
$$
per $v \to 0$.

\bigskip

{\bf Definizione.} Se $I \colon U \to \R$ \`e derivabile secondo
Fr\'echet in ogni punto $u \in U$, diremo che $I$ \`e
Fr\'echet-derivabile in $U$. La derivata di Fr\'echet di $I$ \`e
allora la mappa $I' \colon U \to X^\*$ che ad $u \in U$ associa $I'(u)
\in X^\*$. Si tratta --- in generale --- di una mappa {\em non
  lineare}.

Se $I'$ \`e una mappa continua da $U$ in $X^\*$, diremo che $I \in C^1(U)$.

\pg;

\sec Il caso hilbertiano

Se $H$ \`e uno spazio di Hilbert (reale), \`e noto che gli elementi
del duale $H^\*$ sono isometricamente identificati con vettori di $H$
attraverso l'isomorfismo di Riesz. In particolare, un funzionale $I$
definito su $U \subset H$ \`e derivabile in $u \in U$ se e solo esiste
un vettore, detto d'ora in poi {\em gradiente} di $I$ in $u$ e
denotato $\nabla I(u)$, tale che
$$
I(u+v) = I(u) + \langle \nabla I(u) \mid v \rangle + o(\|v\|)
$$
per $v\to 0$.

\pg;

{\bf Proposizione.} Siano $I$ e $J$ due funzionali derivabili nel
punto $u \in X$. Allora valgono le seguenti affermazioni.
\begitems
\style n
* Se $a$ e $b$ sono numeri reali, allora $aI+bJ$ \`e derivabile in
$u$, e vale $(aI+bJ)'(u)=aI'(u)+bJ'(u)$.
* Il prodotto $IJ$ \`e derivabile in $u$, e vale $(IJ)'(u) =
J(u)I'(u)+I(u)J'(u)$.
* Se $\gamma \colon \R \to U$ \`e una curva derivabile in $t_0$ e
$u=\gamma(t_0)$, allora la composizione $\eta \colon \R \to \R$
definita da $\eta(t)=I(\gamma(t))$ \`e derivabile in $t_0$, e vale
$\eta'(t_0)=I'(u)[\gamma'(t_0)]$.
* Se $A \subset R$ \`e un aperto, $f \colon A \to R$ \`e derivabile in
$I(u) \in A$, allora la composizione $K(u)=f(I(u))$ \`e definita in un
intorno $V$ di $u$, \`e derivabile in $u$ e vale $K'(u)=f'(I(u))I'(u)$.
\enditems

\pg;

{\em Dim.} La prima affermazione \`e banale (esercizio!). Per quanto
riguarda la seconda, quando $v \to 0$ in $X$, abbiamo
$$
\eqalign{%
I(u+v)J(u+v) &= \left( I(u)+I'(u)[v]+o(\|v\|) \right) \left(
  J(u)+J'(u)[v]+o(\|v\|) \right) \cr
&= I(u)J(u) + J(u)I'(u)[v] + I(u)J'(u)[v] + I'(u)[v] J'(u)[v] \cr
&\quad {} + o(\|v\|) \left( I(u)+I'(u)[v]+J(u)+J'(u)[v]+o(\|v\|)
\right). \cr
}%
$$
Concludiamo osservando che
$$
I'(u)[v] J'(u)[v]
+ o(\|v\|) \left( I(u)+I'(u)[v]+J(u)+J'(u)[v]+o(\|v\|) \right)
$$
\`e $o(\|v\|)$ per $v \to 0$. La terza affermazione \`e simile,
infatti per $h \to 0$ in $\R$
$$
\eqalign{%
  \eta(t_0+h) &= I(\gamma (t_0+h)) = I(\gamma(t_0)+\gamma'(t_0)h +
  o(|h|)) \cr
  &= I(u) + I'(u)[\gamma'(t_0)h+o(|h|)] + o(\|\gamma'(t_0)h+o(|h|)\|)
  \cr
  &= \eta(t_0) +I'(u)[\gamma'(t_0)h] +I'(u)[o(|h|)] +
  o(\|\gamma'(t_0)h+o(|h|)\|). \cr
}%
$$
Poich\'e gli ultimi due addendi sono $o(|h|)$, otteniamo che
$$
\eta(t_0+h) = \eta(t_0)+I'(u)[\gamma'(t_0)h]+o(|h|).
$$

\pg;

Infine, quando $v \to 0$ in $X$, si verifica come prima che
$$
\eqalign{%
  K(u+v)&= f(I(u+v)) = f(I(u)+I'(u)[v]+o(\|v\|)) \cr
  &= f(I(u))+f'(I(u))(I'(u)[v]+o(\|v\|))+o(I'(u)[v]+o(\|v\|)) \cr
  &= f(I(u)) + f'(I(u))I'(u)[v]+o(\|v\|).
}%
$$

\bigskip

{\bf Osservazione.} \`E possibile introdurre il concetto di derivata
per applicazioni tra due spazi di Banach $X$ e $Y$. Solo in questo
contesto pu\`o essere enunciata una formulazione completa della regola
di derivazione delle funzioni composte.

Poich\'e non ne faremo uso in queste lezioni, rimandiamo al testo di
Ambrosetti e Prodi per ulteriori approfondimenti.

\pg;

{\bf Definizione.} Sia $I$ un funzionale definito nell'aperto $U$ di
$X$, e sia $u \in U$. Diremo che $I$ \`e derivabile secondo G\^ateaux
in $u$ se esiste un elemento $A \in X^\*$ tale che
$$
\lim_{t \to 0} {I(u+tv)-I(u) \over t} = Av \eqmark
$$
per ogni $v \in X$. In tal caso, l'unico (esercizio!) elemento
siffatto prende il nome di derivata secondo G\^ateaux di $I$ in $u$, e
si denota con $I'_G(u)$ o con $d_GI(u)$.

\bigskip

Osserviamo che questa nuova derivata riprende la cosiddetta {\em
  derivata direzionale} gi\`a nota nell'ambito del calcolo
differenziale in dimensione finita.

In particolare, ricordando i ``soliti'' esempi in $\R^2$, deduciamo
che esistono funzionali (non lineari) derivabili secondo G\^ateaux ma
non derivabili secondo Fr\'echet.

\pg;

\sec Condizione sufficiente per la derivabilit\`a secondo Fr\'echet

{\bf Proposizione.} Supponiamo che $U \subset X$ sia un aperto, che
$I$ sia G\^ateaux-derivabile in $U$, e che $I'_G$ sia continua in un
punto $u \in U$. Allora $I$ \`e Fr\'echet-derivabile in $u$, e
(ovviamente) $I'(u)=I'_G(u)$.

\bigskip

Omettiamo la dimostrazione, che \`e probabilmente stata proposta nel
caso $X=\R^2$ nel corso di Analisi Matematica 2.

\pg;

\sec Punti critici

{\bf Definizione.} Siano $X$ uno spazio di Banach, $U$ un aperto di
$X$, e $I$ un funzionale definito su $U$. Diremo che $u \in U$ \`e un
punto critico di $I$ se $I$ \`e derivabile in $u$ e
$$
I'(u)=0.
$$
Pi\`u esplicitamente, questo significa che $I'(u)[v]=0$ per ogni $v
\in  X$.

Se $u$ \`e un punto critico di $I$ e $I(u)=c$, diremo che $u$ \`e un
punto critico (di $I$) al livello $c$. Se, per qualche $c \in \R$,
l'insieme $I^{-1}(\{c\})\subset X$ contiene almeno un elemento, diremo
che $c$ \`e un valore critico per $I$.

\bigskip

L'equazione $I'(u)=0$ \`e nota come equazione di Eulero (o di
Eulero-Lagrange) associata al funzionale $I$.

\pg;

\sec Esempi

{\bf Esempio 1.} Ogni $A \in X^\*$ \`e derivabile. Infatti, basta
scrivere
$$
A[u+v]=Au+Av
$$
per dedurre che $A'(u)=A$ per qualsiasi $u \in X$.

\medskip

{\bf Esempio 2.} Sia $X$ uno spazio di Banach, e sia $a \colon X
\times X \to \R$ una forma bilineare continua. Denotiamo con $J \colon
X \to \R$ il funzionale definito da $J(u)=a(u,u)$ per ogni $u \in
X$. Allora $J$ \`e derivabile in $X$. Infatti
$$
\eqalign{%
J(u+v) &=a(u+v,u+v) = a(u,u)+a(u,v)+a(v,u)+a(v,v) \cr
&= J(u) +
a(u,u)+a(u,v)+a(v,u)+a(v,v).
}%
$$
Poich\'e $|a(v,v)| \leq M \|v\|^2$ per l'ipotesi di continuit\`a di
$a$ come forma bilineare, deduciamo che $a(v,v)=o(\|v\|)$ per $v \to
0$, e dunque che
$$
J'(u)[v]=a(u,v)+a(v,u).
$$

\pg;

{\bf Esempio 3.} (esercizio) Sia $H$ uno spazio di Hilbert con norma $\| \cdot
\|$. Il funzionale $J(u)=\|u\|$ \`e derivabile in ogni punto $u \neq
0$, e risulta
$$
\nabla J(u) = {u \over \|u\|}.
$$

\medskip

{\bf Esempio 4.} Sia $X$ uno spazio di Banach, e siano $I$, $J$ due
funzionali derivabili in $X$. Definiamo
$$
Q(u) = {I(u) \over J(u)}
$$
sul sottoinsieme (aperto) $\{u \in  X \mid J(u) \neq 0\}$. Per la
Proposizione sulle regole di calcolo dimostrata sopra, possiamo
affermare che $Q$ \`e derivabile e che
$$
Q'(u) = {J(u) I'(u)[v] - I(u) J'(u)[v] \over J(u)^2}
$$
per ogni $u\ in X$ tale che $J(u) \neq 0$.

\pg;

\sec Esempi in spazi concreti

{\bf Esempio 5.} Sia $\Omega \subset \R^N$, $N \geq 1$, un insieme
aperto e limitato. Definiamo i funzionali
$$
I \colon L^2(\Omega) \to \R, \quad I(u) = \int_\Omega |u(x)|^2 \, dx,
$$
$$
J \colon H_0^1(\Omega) \to \R, \quad J(u) = \int_\Omega |\nabla
u(x)|^2 \, dx,
$$
$$
K \colon H^1(\Omega) \to \R, \quad K(u) = \int_\Omega |\nabla
u(x)|^2 \, dx,
$$
$$
L \colon H^1(\Omega) \to \R, \quad L(u) = \int_\Omega |\nabla
u(x)|^2 \, dx + \int_\Omega |u(x)|^2 \, dx.
$$
Trattandosi di forme quadratiche associate a forme bilineari continue,
sappiamo gi\`a che i quattro funzionali sono derivabili.

\pg;

Esplicitamente, valgono le relazioni
$$
\eqalign{%
\nabla I(u) &= 2u \cr
\nabla L(u) &= 2u \cr
\nabla J(u) &= 2u.
}
$$
Un
calcolo diretto mostra che
$$
K'(u)[v] = 2 \int_\Omega \nabla u(x) \cdot \nabla v(x) \, dx
$$
per ogni $u$, $v \in H^1(\Omega)$, ma non siamo autorizzati ad
affermare che $\nabla K(u)=2u$ (perch\'e?)

\pg;

\sec Inversione della Convergenza Dominata

{\bf Teorema di Lebesgue.} Sia $\Omega$ un aperto di $\R^N$, e sia
$\{u_k\}_k$ una successione in $L^1(\Omega)$ tale che
\begitems
\style n
* $u_k(x) \to u(x)$ per q.o $x \in \Omega$;
* esiste $v \in L^1(\Omega)$ tale che $|u_k(x)| \leq v(x)$ per q.o. $x
\in \Omega$ e ogni $k$.
\enditems
Allora $u \in L^1(\Omega)$ e $u_k \to u$ nella norma di $L^1(\Omega)$.

\bigskip

Questo risultato fondamentale di Teoria della Misura pu\`o essere {\em
parzialmente} invertito, come mostra il seguente teorema. Per la
dimostrazione, rimandiamo al libro di H. Brezis, Analisi funzionale.

\pg;

{\bf Teorema.} Sia $\Omega$ un aperto di $\R^N$, e sia $\{u_k\}_k$ una
successione di $L^p(\Omega)$, $p \in [1,+\infty]$, tale che $u_k \to
u$ in $L^p(\Omega)$. Allora esistono una sottosuccessione
$\{u_{k_j}\}_j$ ed una funzione $v \in L^p(\Omega)$ tali che
\begitems
\style n
* $u_{k_j}(x) \to u(x)$ per q.o. $x \in \Omega$;
* per ogni $j$, $|u_{k_j}(x)| \leq v(x)$ per q.o. $x \in \Omega$.
\enditems

\bigskip

Questo teorema mostra che la convergenza forte in $L^p$ implica --- a
meno di sottosuccessioni --- l'esistenza di una funzione dominante.

\pg;

\sec Operatori di Nemitskii

Siano $\Omega$ un aperto limitato di $\R^N$, $N \geq 3$, con frontiera
regolare, e sia $f \colon \R \to \R$ una funzione continua. Supponiamo
che esistano $a>0$ e $b>0$ tali che
$$
|f(t)| \leq a+b|t|^{2^\*-1},
$$
dove $2^\* = 2N/(N-2)$ \`e l'esponente critico di Sobolev. Definiamo
$$
F(t) = \int_0^t f(x) \, dx
$$
e consideriamo il funzionale $J \colon H^1(\Omega) \to \R$ dato da
$$
J(u) = \int_\Omega F(u(x))\, dx.
$$

\pg;

{\bf Proposizione.} Sotto le ipotesi precedenti, $J$ \`e un funzionale
derivabile in $H^1(\Omega)$, e vale
$$
J'(u)[v] = \int_\Omega f(u(x))v(x)\, dx
$$
per ogni $u$, $v \in H^1(\Omega)$.

\bigskip

La dimostrazione non \`e immediata: mostriamo prima che $J$ \`e
G\^a\-te\-aux-derivabile, e poi che la derivata di G\^ateaux \`e
continua. Come abbiamo visto sopra, ci\`o implica che $J$ \`e
Fr\'echet-derivabile.

\pg;

* Derivata di G\^ateaux

Per q.o. $x\in\Omega$, risulta
$$
\lim_{t \to 0} {F(u(x)+t(v(x)) - F(u(x)) \over t} = f(u(x))v(x).
$$
Per il teorema di Lagrange, esiste un numero reale $\theta$ tale che
$|\theta| \leq |t|$ e
$$
\eqalign{%
\left| {F(u(x)+t(v(x)) - F(u(x)) \over t} \right| &= \left|
f(u(x)+\theta v(x))v(x) \right| \cr
&\leq \left( a+b|u(x)+\theta v(x)|^{2^\*-1} \right) |v(x)| \cr
&\leq C \left( |v(x)|+|u(x)|^{2^\*-1}|v(x)|+|v(x)|^{2^\*} \right).
}
$$
Per Convergenza Dominata,
$$
\lim_{t \to 0} \int_\Omega {F(u(x)+t(v(x)) - F(u(x)) \over t}\, dx =
\int_\Omega f(u(x))v(x)\, dx.
$$

\pg;

Poich\'e $v \mapsto \int_\Omega f(u(x))v(x)\, dx$ \`e un operatore
lineare e continuo  in $H^1(\Omega)$ (disuguaglianza di H\"older e di
Sobolev), abbiamo individuato la derivata secondo G\^ateaux di $J$:
$$
J'_G(u)[v] = \int_\Omega f(u(x))v(x)\, dx.
$$

* Derivata di Fr\'echet

Mostriamo che $J'_G \colon H^1(\Omega) \to (H^1(\Omega))^\*$ \`e
un'applicazione continua. A tal fine, sia $\{u_k\}_k$ una successione
che converge a $u$ in $H^1(\Omega)$. Per il teorema di convergenza
dominata inversa, possiamo supporre che --- a meno di sottosuccessioni
---
\begitems
* $u_k \to u$ in $L^{2^\*}(\Omega)$;
* $u_k(x) \to u(x)$ per q.o. $x \in \Omega$;
* esiste $w \in L^{2^\*}(\Omega)$ tale che $|u_k(x)| \leq w(x)$ per
q.o. $x \in \Omega$ e ogni $k$.
\enditems

\pg;

Usiamo la disuguaglianza di H\"older:
$$
\eqalign{%
\left| (J'_G(u_k)-J'_G(u))[v] \right| &\leq \int_\Omega
|f(u_k(x))-f(u(x))| |v(x)| \, dx \cr
&\leq \left( \int_\Omega |f(u_k(x))-f(u(x))|^{2^\* \over 2^\*-1}\, dx
\right)^{2^\*-1 \over 2^\*} \times \cr
&\quad \times \left( \int_\Omega |v(x)|^{2^\*}\, dx
\right)^{1/2^\*}.
}
$$
La continuit\`a di $f$ implica $\lim_{k \to +\infty}
|f(u_k(x))-f(u(x))|=0$ per q.o. $x\in\Omega$, e inoltre
$$
\eqalign{%
|f(u_k(x))-f(u(x))|^{2^\* \over 2^\* -1} &\leq C \left(
1+|u_k(x)|^{2^\*-1} + |u(x)|^{2^\* -1} \right)^{2^\* \over 2^\* -1} \cr
&\leq C \left(
1+|w(x)|^{2^\*-1} + |w(x)|^{2^\* -1} \right)^{2^\* \over 2^\* -1} \cr
&\leq C \left(
1+|w(x)|^{2^\*} + |w(x)|^{2^\* } \right) \in L^1(\Omega).  \cr
}
$$
Per Convergenza Dominata,
$$
\lim_{k \to +\infty} \int_\Omega |f(u_k(x))-f(u(x))|^{2^\* \over 2^\*
-1} \, dx =0.
$$

\pg;

Perci\`o
$$
\eqalign{%
\|J'_G(u_k)-J'_G(u)\| &= \sup \{ (J'_G(u_k)-J'_G(u))[v] \mid v \in
H^1(\Omega),\ \|v\|=1 \} \cr
&\leq C \left( \int_\Omega |f(u_k(x))-f(u(x))|^{2^\* \over 2^\*
-1} \, dx \right)^{2^\* -1 \over 2^\*} \to 0.
}
$$
Riassumendo: abbiamo dimostrato che da ogni successione $\{u_k\}_k$
convergente a $u$ \`e possibile estrarre una sottosuccessione tale che
$J'_G(u_k) \to J'_G(u)$ in $(H^1(\Omega))^\*$. \`E ora un esercizio di
Topologia Generale dedurre che l'intera successione $\{u_k\}_k$ gode
di questa propriet\`a (perch\'e il limite \`e indipendente dalla
sottosuccessione scelta).


\pg. %------------------------------FINE-------------------------------

